\documentclass{reference_card}
\begin{document}
\renewcommand{\footrulewidth}{0.4pt}
\fancyhead[LE,LO]{git -- quick reference (page \thepage/\pageref{LastPage})}
\fancyhead[RO,RE]{source: \url{https://github.com/remigiusz-suwalski/.../releases}}
\fancyfoot[RF]{author: Remigiusz Suwalski,\\ date: \today}
\fancyfoot[LF]{}

\begin{multicols*}{3}
\textbf{git} is a stupid content tracker.

\begin{compactenum}
\item [\texttt{git}] \texttt{blame} shows which author and revision last modified each line of a file.
\end{compactenum}

\begin{compactenum}
\item [\texttt{git}] \texttt{branch} lists local branches.
\item [\texttt{-a}] \emph{all} branches instead.
\item [\texttt{-r}] \emph{remote} branches instead.
\item [\texttt{-v}] show sha1 and commit subject for each head.
\item [\texttt{-d}] \emph{deletes} a local branch (\texttt{D}: with force).
\item [\texttt{-m}] \emph{moves} a local branch (\texttt{M}: with force).
\end{compactenum}

\begin{compactenum}
\item [\texttt{git}] \texttt{clean} removes untracked files.
\item [\texttt{-d}] untracked \emph{directories} as well.
\item [\texttt{-f}] with \emph{force}.
\item [\texttt{-x}] ignored files too.
\end{compactenum}

\begin{compactenum}
\item [\texttt{git}] \texttt{clone} \emph{clones} a repository.
\item [---] \texttt{recurse-submodules}: recursively.
\end{compactenum}

\begin{compactenum}
\item [\texttt{git}] \texttt{checkout} updates working tree.
\item [\texttt{-b}] creates a new branch.
\item [\texttt{-f}] throws away local changes.
\item [\texttt{-p}] selectively discards edits from working tree.
\item [\texttt{.}] reverts local uncommitted changes (?).
\end{compactenum}

\begin{compactenum}
\item [\texttt{git}] \texttt{cherry-pick} applies changes introduced by other commit.
\end{compactenum}

\begin{compactenum}
\item [\texttt{git}] \texttt{diff} shows changes between commits.
\end{compactenum}

\begin{compactenum}
\item [\texttt{git}] \texttt{fetch} downloads objects and refs.
\item [\texttt{-p}] \emph{prunes} refs that no longer exist on the remote.
\item [\texttt{-t}] fetches all \emph{tags} from the remote.
\item [---] \texttt{all}: fetches all remotes.
\end{compactenum}

\begin{compactenum}
\item [\texttt{git}] \texttt{gc} cleanups unnecessary files.
\end{compactenum}

\begin{compactenum}
\item [\texttt{git}] \texttt{init} creates an empty Git repository.
\item [---] \texttt{bare}: creates a bare repository.
\end{compactenum}

\begin{compactenum}
\item [\texttt{git}] \texttt{log} shows commit logs.
\item [\texttt{-G}] looks for diffs whose patch contains added or removed lines matching regex.
\item [\texttt{-S}] looks for diffs that change \# of occurrences of given string.
\item [\texttt{-U}] with different than usual three lines of context.
\item [\texttt{-n}] limits the \emph{number} of commits to output.
\item [\texttt{-p}] generates a patch.
\item [---] \texttt{author}, \texttt{committer}: limits to commits with author/committer matching (any) pattern.
\item [---] \texttt{oneline}: shows abbrev. commits, one per line.
\item [---] \texttt{format}: pretty-prints in a given format.
\begin{compactenum}
\item [\texttt{H}] commit hash (\texttt{h}: abbreviated),
\item [\texttt{an}] author name,
\item [\texttt{ae}] author email,
\item [\texttt{ad}] author date (\texttt{ar} relative, \texttt{ai} ISO 8601, \texttt{at} UNIX timestamp), commiter: \texttt{a} $\to$ \texttt{c},
\item [\texttt{d}] ref name,
\item [\texttt{s}, \texttt{b}] subject, body.
\end{compactenum}
\item [---] \texttt{graph}: draws a representation of history.
\item [---] \texttt{no-merges}: skip commits with more than one parent.
\end{compactenum}

\begin{compactenum}
\item [\texttt{git}] \texttt{merge} join 2+ development histories together.
\end{compactenum}

\begin{compactenum}
\item [\texttt{git}] \texttt{mv} moves files, directories, symlinks.
\item [\texttt{-f}] \emph{forces} move even if target exists (required on OSX when changing file name's capitalization).
\item [\texttt{-n}] performs a dry run.
\end{compactenum}

\begin{compactenum}
\item [\texttt{git}] \texttt{pull} fetches and merges changes from remote repository.
\item [\texttt{-r}] \emph{rebases} current branch instead of merging into.
\end{compactenum}

\begin{compactenum}
\item [\texttt{git}] \texttt{rebase} reapplies commits on top of smth.
\item [\texttt{-i}] \emph{interactively}: lists commits to be rebased.
\item [---] \texttt{onto}: rebases starting from given point.
\end{compactenum}

\begin{compactenum}
\item [\texttt{git}] \texttt{remote} manages set of tracked repositories.
\item [\texttt{-v}] be more \emph{verbose} (showiung remote URLs).
\item [\scriptsize \texttt{add}] adds a remote.
\item [\scriptsize \texttt{rename}] renames the remote.
\item [\scriptsize \texttt{remove}] removes the remote.
\end{compactenum}

\begin{compactenum}
\item [\texttt{git}] \texttt{reset} resets current HEAD to specified state.
\item [---] \texttt{hard}: resets both index and working tree.
\item [---] \texttt{mixed}: resets index but not the working tree.
\item [---] \texttt{soft}: doesn't touch the index file, working tree.
\end{compactenum}

\begin{compactenum}
\item [\texttt{git}] \texttt{revert} records new commit undoing effect of earlier commits.
\end{compactenum}

\begin{compactenum}
\item [\texttt{git}] \texttt{rm} removes files from working tree and index.
\item [\texttt{-n}] performs a dry run instead.
\item [\texttt{-r}] allows \emph{recursive} removal.
\item [---] \texttt{cached}: unstages and remove paths only from index. Working tree will be left alone.
\end{compactenum}

\begin{compactenum}
\item [\texttt{git}] \texttt{shortlog} summarizes output of git log.
\item [\texttt{-n}] sorts output by \emph{number} of commits per author.
\item [\texttt{-s}] provides a commit count \emph{summary} only.
\item [\texttt{-e}] prints \emph{email} addresses.
\end{compactenum}

\begin{compactenum}
\item [\texttt{git}] \texttt{show} shows blobs, trees, tags, commits.
\end{compactenum}

\begin{compactenum}
\item [\texttt{git}] \texttt{stash} stashes changes in dirty directory away.
\item [\scriptsize \texttt{apply}] applies single stash.
\item [\scriptsize \texttt{list}] lists stashes.
\item [\scriptsize \texttt{push}] default behaviour.
\item [\scriptsize \texttt{pop}] removes single stash and applies it.
\item [\scriptsize \texttt{show}] shows changes recorded in stash entry.
\end{compactenum}

\begin{compactenum}
\item [\texttt{git}] \texttt{status} shows the working tree status.
\item [\texttt{-b}] shows the \emph{branch} even in short-format.
\item [\texttt{-s}] gives output in \emph{short} format.
\end{compactenum}

\begin{compactenum}
\item [\texttt{git}] \texttt{tag} lists tags.
\item [\texttt{-a}] makes an \emph{annotated} tag.
\item [\texttt{-d}] \emph{deletes} existing tags.
\item [\texttt{-f}] \emph{forces} replacement of existing tags.
\item [\texttt{-m}] uses given tag \emph{message} instead of prompting.
\item [\texttt{-s}] makes a GPG-\emph{signed} tag.
\item [\texttt{-u}] the same as \texttt{-s} but with custom key.
\item [\texttt{-v}] \emph{verifies} GPG signature of tags.
\item [---] \texttt{contains}: only tags containing given commit.
\item [---] \texttt{sort}: sorts by keys from \texttt{git for-each-ref}.
\end{compactenum}

\section*{Snippets}
To remove a commit:
\begin{minted}{bash}
git rebase --onto ~<sha>
\end{minted}

To remove a submodule:
\begin{minted}{bash}
git submodule deinit <name>
git rm <name>
git rm --cached <name>
rm -rf .git/modules/<name>
\end{minted}

To remove a tag:
\begin{minted}{bash}
git tag -d <tag>
git push <remote> :refs/tags/<tag>
\end{minted}

To undo a change:
\begin{minted}{bash}
git reset HEAD -- # unstage
git reset --mixed HEAD~ # uncommit
git checkout -- # discard
\end{minted}

\section*{git config}
\subsection*{core}
\texttt{fileMode false} don't honor the executable bit of files in the working tree.
\end{multicols*}
\end{document}
%

\textbf{git commit}\git{commit} creates a commit, which is like a snapshot of the repository.

\mintinline{bash}{git commit --amend}: instead of creating new commit, combines staged changes with the previous commit.

\mintinline{bash}{git commit --reset-author}: when amending or committing a conflicting cherry-pick, declare authorship of the resulting commit to the committer.
See also \mintinline{bash}{git commit --author=...}

\mintinline{bash}{git commit --no-edit}: to keep current commit message without launching an editor (useful during amends).

Useful short options:
\begin{compactenum}
\item [\texttt{-m}] with specified commit message (equivalent to \mintinline{bash}{--message=...}),
\item [\texttt{-v}] with diff between HEAD and what would be committed at the bottom of commit message template.
Diff will not be part of the commit message!.
(equivalent to \mintinline{bash}{--verbose}),
\end{compactenum}

%
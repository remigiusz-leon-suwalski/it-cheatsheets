%

\textbf{git config}\git{config} gets or sets local (repository) or global options.

\mintinline{bash}{git config --list} lists all variables and their values.

\mintinline{bash}{git config --show-origin} augments the output with the origin type (file, stdin, blob, command line) and the actual origin (path, ref, blob).

\mintinline{bash}{git config option-name option-value} sets an option repository wide, exactly like with \mintinline{bash}{--local} switch.
By adding \mintinline{bash}{--global} flag the same happens for all repositories of current (Unix) user.
There is also a rarily used flag \mintinline{bash}{--system} that is applied system-wide.

Useful options include:
\begin{minted}{bash}
git config user.email 'john@doe.com'
git config user.name 'John Doe'

# "reuse recorded resolution"; tells Git to remember how
# a hunk conflict has been resolved so that the next time
# it sees the same conflict, Git resolves it automatically.
git config rerere.enabled true

# keep HTTPS credentials using Unix sockets, protected with
# file permissions, so generally speaking, they are secure
git config credential.helper cache
\end{minted}

%
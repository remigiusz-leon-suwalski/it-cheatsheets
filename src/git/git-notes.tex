\documentclass{charun}
\title{Git notes, version 0.2.0}
\author{Leon Suwalski}
\begin{document}
\begin{multicols*}{2}
\maketitle
\raggedright

\section{git add}
\textbf{git add} adds new or changed files to the staging area.
Useful options:
\begin{compactenum}
\item [\texttt{-f}] allow adding otherwise ignored files (equivalent to \mintinline{bash}{--force}),
\item [\texttt{-p}] interactively choose hunks of patch to be added (equivalent to \mintinline{bash}{--patch}),
\item [\texttt{-v}] be verbose.
\end{compactenum}

\section{git commit}
\textbf{git commit} creates a commit, which is like a snapshot of the repository.

\mintinline{bash}{git commit --reset-author}: when amending or committing a conflicting cherry-pick, declare authorship of the resulting commit to the committer.
See also \mintinline{bash}{git commit --author=...}

\mintinline{bash}{git commit --no-edit}: to keep current commit message without launching an editor (useful during amends).

Useful short options:
\begin{compactenum}
\item [\texttt{-m}] with specified commit message,
\item [\texttt{-v}] with diff between HEAD and what would be committed at the bottom of commit message template (diff will not be part of the commit message!).
\end{compactenum}


\end{multicols*}
\end{document}
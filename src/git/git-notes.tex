\documentclass{charun}
\title{Git notes, version 0.12.8}
\author{Leon Suwalski}

\usepackage{xcolor}
% copied from https://www.color-hex.com/color-palette/85828
\definecolor{gitcolor}{HTML}{ffecec}
\definecolor{gitcolor2}{HTML}{990000}

% copied from https://tex.stackexchange.com/questions/132849/how-can-i-change-the-font-size-of-the-number-in-minted-environment
\BeforeBeginEnvironment{minted}{\begin{tcolorbox}[breakable,boxsep=5pt,left=0pt,right=0pt,top=0pt,bottom=0pt,boxrule=0.5pt,arc=0pt,outer arc=0pt,colback=gitcolor,colframe=gitcolor2]\small}%
\AfterEndEnvironment{minted}{\end{tcolorbox}}%
    
\newcommand{\git}[1]{\marginpar[\textbf{\color{blue}{#1}}]{\textbf{\color{blue}{#1}}}}

\usepackage{marginnote}

\begin{document}
\maketitle
\raggedright

\textbf{git} (v2.45.2) is a stupid content tracker.

\section{Git commands -- configuration}
%

\textbf{git config}\git{config} gets or sets local (repository) or global options.

\mintinline{bash}{git config --list} lists all variables and their values.

\mintinline{bash}{git config --show-origin} augments the output with the origin type (file, stdin, blob, command line) and the actual origin (path, ref, blob).

\mintinline{bash}{git config option-name option-value} sets an option repository wide, exactly like with \mintinline{bash}{--local} switch.
By adding \mintinline{bash}{--global} flag the same happens for all repositories of current (Unix) user.
There is also a rarily used flag \mintinline{bash}{--system} that is applied system-wide.

Useful options include:
\begin{minted}{bash}
git config user.email 'john@doe.com'
git config user.name 'John Doe'

# "reuse recorded resolution"; tells Git to remember how
# a hunk conflict has been resolved so that the next time
# it sees the same conflict, Git resolves it automatically.
git config rerere.enabled true

# keep HTTPS credentials using Unix sockets, protected with
# file permissions, so generally speaking, they are secure
git config credential.helper cache
\end{minted}

%

\section{Git commands -- setting up a repository}
%

\section{git init}
\textbf{git init} turns current directory into a Git repository (by creating a hidden .git subdirectory).

\mintinline{bash}{git init --bare} creates a bare repository, that is used as a remote, and not for active development.

%
%

\textbf{git clone}\git{clone} downloads an already existing repository.
It's more or less equivalent to init, remote add, fetch and checkout.

\mintinline{bash}{git clone --bare} clones a bare repository which serves as an authoritative focal point for collaborative development.

\mintinline{bash}{git clone --recurse-submodules -j8} clones repository including its submodules, up to eight of them at a time in parallel.

%

\section{Git commands -- tracking changes}
%

\textbf{git add}\git{add} adds new or changed files to the staging area.
Useful options:
\begin{compactenum}
\item [\texttt{-f}] allow adding otherwise ignored files (equivalent to \mintinline{bash}{--force}),
\item [\texttt{-p}] interactively choose hunks of patch to be added (equivalent to \mintinline{bash}{--patch}),
\item [\texttt{-v}] be verbose (equivalent to \mintinline{bash}{--verbose}).
\end{compactenum}

%
%

\subsection{git commit}
\textbf{git commit} creates a commit, which is like a snapshot of the repository.

\mintinline{bash}{git commit --reset-author}: when amending or committing a conflicting cherry-pick, declare authorship of the resulting commit to the committer.
See also \mintinline{bash}{git commit --author=...}

\mintinline{bash}{git commit --no-edit}: to keep current commit message without launching an editor (useful during amends).

Useful short options:
\begin{compactenum}
\item [\texttt{-m}] with specified commit message (equivalent to \mintinline{bash}{--message=...}),
\item [\texttt{-v}] with diff between HEAD and what would be committed at the bottom of commit message template.
Diff will not be part of the commit message!.
(equivalent to \mintinline{bash}{--verbose}),
\end{compactenum}

%
%

\textbf{git stash}\git{stash} stashes the changes in a dirty directory away.
It's a bit differs a bit from other Git commands as it has no short options.

\begin{minted}{bash}
# the same as 'git stash'
git stash push

# list the stash entries
git stash list

# inverse operation of 'git stash push':
# remove a single stashed state,
# apply it to the working tree.
git stash pop

# keep a single stashed state,
# apply it to the working tree.
git stash apply
\end{minted}

%
%

\textbf{git mv}\git{mv} moves files, directories, symlinks.
Since Git doesn't explicitly track file movements, this is equivalent to renaming followed by \mintinline{bash}{git rm ...} and \mintinline{bash}{git add ...}.
Useful short options:
\begin{compactenum}
\item [\texttt{-f}] forces move even if target exists (equivalent to \mintinline{bash}{--force}),
\item [\texttt{-n}] performs a dry run (equivalent to \mintinline{bash}{--dry-run}).
\end{compactenum}

On OSX, \texttt{-f} flag is mandatory when changing capitalization of file name.

%
%

%


\section{Git commands -- branching}
%

%

%

\textbf{git checkout}\git{checkout} updates the working tree (usually to switch between branches).
Useful short options:
\begin{compactenum}
\item [\texttt{-b}] creates a new branch instead (equivalent to \mintinline{bash}{git branch ...} followed by \mintinline{bash}{git checkout ...}),
\item [\texttt{-f}] throws away local changes (equivalent to \mintinline{bash}{--force}),
\item [\texttt{-p}] selectively discards edits from working tree (equivalent to \mintinline{bash}{--patch}).
\end{compactenum}

TODO: --track

%
%

%

%

\textbf{git merge-base}\git{merge-base} finds common ancestor(s) between branches to use in a three-way merge.
If there are at least three branches, use \mintinline{bash}{--octopus} flag!

%
%

%

% TODO: reflog
%

\begin{compactenum}
\item [\texttt{git}] \texttt{cherry-pick} applies changes introduced by other commit.
\end{compactenum}

%

\section{Git commands -- tags}
%

%


\section{Git commands -- history}
%

%

%

%

%

%

%

%

%

\textbf{git blame}\git{blame} shows which author and revision last modified each line of a file.

%
%

%


\section{Git commands -- undoing changes}
%

\textbf{git revert}\git{revert} creates a commit that undoes one or more existing commits.

%
%

\textbf{git clean}\git{clean} removes untracked files.
Useful short options:
\begin{compactenum}
\item [\texttt{-d}] recursing into untracked directories too,
\item [\texttt{-f}] removes untracked files even if clean.requireForce is not set to false,
\item [\texttt{-x}] without using standard ignore rules (.gitignore).
\end{compactenum}

%
%

%

%

%


\section{Git commands -- collaborating}
%

%

%

\textbf{git push}\git{push} uploads local branch commits to the corresponding remote branch.

\mintinline{bash}{git push origin master} pushes branch \emph{master} to remote \emph{origin}.

Useful short options:
\begin{compactenum}
\item [\texttt{-f}] disables overwrite checks (equivalent to \mintinline{bash}{--force}).
\end{compactenum}

%
%
\subsection{git pull}
\textbf{git pull} updates current local working branch  and all remote tracking branches.
It's more or less equivalent to fetch followed by a merge.

\mintinline{bash}{git pull --rebase origin master} rewrites history so any local commits occur after all new commits from the remote.
Useful to avoid merge commits.

%
%

%


\section{Specifying revisions}
\begin{minted}{bash}
# @{u} refers to the upstream branch which the
# current branch is tracking, for example:
git merge @{u}
# @{upstream} means the same.
\end{minted}

\section{To be done}
To remove a commit:
\begin{minted}{bash}
git rebase --onto ~<sha>
\end{minted}

To remove a submodule:
\begin{minted}{bash}
git submodule deinit <name>
git rm <name>
git rm --cached <name>
rm -rf .git/modules/<name>
\end{minted}

To remove a tag:
\begin{minted}{bash}
git tag -d <tag>
git push <remote> :refs/tags/<tag>
\end{minted}

To undo a change:
\begin{minted}{bash}
git reset HEAD -- # unstage
git reset --mixed HEAD~ # uncommit
git checkout -- # discard
git restore
\end{minted}

% git checkout . undoes unstaged local modification
% git reset undoes staged modifications

\texttt{fileMode false} don't honor the executable bit of files in the working tree.

\section{.gitignore (work in progress)}
\begin{enumerate}
\item A blank line matches no files, so it can serve as a separator for readability.
\item A line starting with \# serves as a comment. Put a backslash ("\") in front of the first hash for patterns that begin with a hash.
\item Trailing spaces are ignored unless they are quoted with backslash ("\").
\item An optional prefix "!" which negates the pattern; any matching file excluded by a previous pattern will become included again. It is not possible to re-include a file if a parent directory of that file is excluded. Git doesn’t list excluded directories for performance reasons, so any patterns on contained files have no effect, no matter where they are defined. Put a backslash ("\") in front of the first "!" for patterns that begin with a literal "!", for example, "\!important!.txt".
\item The slash "/" is used as the directory separator. Separators may occur at the beginning, middle or end of the .gitignore search pattern.
\item If there is a separator at the beginning or middle (or both) of the pattern, then the pattern is relative to the directory level of the particular .gitignore file itself. Otherwise the pattern may also match at any level below the .gitignore level.
\item If there is a separator at the end of the pattern then the pattern will only match directories, otherwise the pattern can match both files and directories.
\item For example, a pattern doc/frotz/ matches doc/frotz directory, but not a/doc/frotz directory; however frotz/ matches frotz and a/frotz that is a directory (all paths are relative from the .gitignore file).
\item An asterisk "*" matches anything except a slash. The character "?" matches any one character except "/". The range notation, e.g. [a-zA-Z], can be used to match one of the characters in a range. See fnmatch(3) and the \mintinline{bash}{FNM_PATHNAME} flag for a more detailed description.
\item Two consecutive asterisks ("**") in patterns matched against full pathname may have special meaning:
\item A leading "**" followed by a slash means match in all directories. For example, "**/foo" matches file or directory "foo" anywhere, the same as pattern "foo". "**/foo/bar" matches file or directory "bar" anywhere that is directly under directory "foo".
\item A trailing "/**" matches everything inside. For example, "abc/**" matches all files inside directory "abc", relative to the location of the .gitignore file, with infinite depth.
\item A slash followed by two consecutive asterisks then a slash matches zero or more directories. For example, "a/**/b" matches "a/b", "a/x/b", "a/x/y/b" and so on.
\item Other consecutive asterisks are considered regular asterisks and will match according to the previous rules.
\end{enumerate}

Todo: git ls-remote

\section{To be done later}
Distributed git: git request-pull, git format-patch -M, git send-email.
git diff \$(git merge-base contrib master) equals git diff master...contrib.
% „gpg -a --export F721C45A | git hash-object -w --stdin”
git describe.


\end{document}

%

% https://www.postgresql.org/docs/15/queries-table-expressions.html
% 7.2 Table expressions
\color{red}
\subsection{Table expressions}
The \textbf{INNER JOIN} keyword selects records that have matching values in both tables.
You can join a table to itself!
\begin{minted}[linenos,firstnumber=last]{sql}
SELECT bks.starttime FROM cd.bookings bks
INNER JOIN cd.members mems ON mems.memid = bks.memid;
\end{minted}

\textbf{LEFT INNER JOIN} produces an output row even if a given row on the left hand table doesn't match anything.
Similarly \textbf{RIGHT INNER JOIN} produces at least one row for each row in right hand table.
Finally we have \textbf{FULL OUTER JOIN} which treats both sides of the expression as optional and is rarely used.

%
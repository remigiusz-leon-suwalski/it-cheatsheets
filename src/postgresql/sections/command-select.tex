%

\color{black}
\subsection{SELECT}
\textbf{SELECT}\footnote{\url{https://www.postgresql.org/docs/current/sql-select.html}} retrieves specified columns (or all columns if \mintinline{sql}{*} is used) from table mentioned after \mintinline{sql}{FROM} keyword:
\begin{minted}{sql}
SELECT * FROM cd.facilities;
SELECT name, membercost FROM cd.facilities;
\end{minted}

\textbf{SELECT DISTINCT} eliminates duplicate rows from the result:
\begin{minted}{sql}
SELECT DISTINCT surname FROM cd.members
\end{minted}

\textbf{AS} gives temporary aliases to the columns:
\begin{minted}{sql}
SELECT name AS facilityname FROM cd.facilities;
\end{minted}

% https://pgexercises.com/questions/basic/where.html
The \textbf{WHERE} clause indicates the condition(s) that rows must satisfy to be selected. 
Usual logical operators are available: \mintinline{sql}{AND}, \mintinline{sql}{OR}, \mintinline{sql}{NOT}.
SQL uses a three-valued logic system with true, false and null (representing unknowns).
See \textbf{\ref{subsection_comparison}} for comparison operators.
\begin{minted}{sql}
SELECT * FROM cd.facilities WHERE membercost > 0;
SELECT facid, name FROM cd.facilities
WHERE membercost > 0 AND (membercost < monthlymaintenance);
\end{minted}

If '\textbf{ORDER BY} + column(s)' is used, returned rows are sorted in the specified order, otherwise rows are returned in whatever rorder the system finds fastest to produce.
Here \mintinline{sql}{ASC} is the default mode.

If '\textbf{LIMIT} + count' is used, no more rows than given count are returned.
If '\textbf{OFFSET} + start' is also used, that many initial rows will be skipped.
(This produces unpredictable output if rows are not sorted with \mintinline{sql}{ORDER BY}!).
Standard SQL uses FETCH instead, LIMIT is Postgresql specific.

\begin{minted}{sql}
SELECT surname, firstname FROM cd.members
ORDER BY surname DESC LIMIT 10 OFFSET 5;
\end{minted}

Operators \textbf{UNION}, \textbf{INTERSECT} and \textbf{INTERSECT} are used to combine outputs of multiple \mintinline{sql}{SELECT} statements, corresponding to $\cup$, $\cap$ and $\setminus$.
In all three cases, duplicate rows are removed unless \textbf{ALL} is used!
\begin{minted}{sql}
SELECT surname FROM cd.members
UNION
SELECT name FROM cd.facilities;
\end{minted}

%
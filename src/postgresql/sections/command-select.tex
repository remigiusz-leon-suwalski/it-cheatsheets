%

\color{black}
\subsection{SELECT}
\textbf{SELECT} retrieves specified columns (or all columns if \mintinline{sql}{*} is used) from table mentioned after \mintinline{sql}{FROM} keyword:
\begin{minted}{sql}
SELECT * FROM cd.facilities;
SELECT name, membercost FROM cd.facilities;
\end{minted}

\textbf{SELECT DISTINCT} eliminates duplicate rows from the result:
\begin{minted}{sql}
SELECT DISTINCT surname FROM cd.members
\end{minted}

The DISTINCT clause is unpredictable unless ORDER BY is used too:
\begin{minted}{sql}
SELECT DISTINCT ON (firstname) firstname, surname
FROM cd.members ORDER BY firstname;
\end{minted}

\textbf{AS} gives temporary aliases to the columns:
\begin{minted}{sql}
SELECT name AS facilityname FROM cd.facilities;
\end{minted}

% https://pgexercises.com/questions/basic/where.html
The \textbf{WHERE} clause indicates the condition(s) that rows must satisfy to be selected. 
Usual logical operators are available: \mintinline{sql}{AND}, \mintinline{sql}{OR}, \mintinline{sql}{NOT}.
SQL uses a three-valued logic system with true, false and null (representing unknowns).
\begin{minted}{sql}
SELECT facid, name FROM cd.facilities
WHERE membercost > 0 AND (membercost < monthlymaintenance);
\end{minted}

With '\textbf{ORDER BY} + column(s)', returned rows are sorted (without: in whatever order the system finds fastest to produce).
Here \mintinline{sql}{ASC} is the default mode.

With '\textbf{LIMIT} + count', no more rows than given count are returned.
{\color{gray}(Standard SQL uses FETCH instead).}
With '\textbf{OFFSET} + start', that many initial rows will be skipped.
Both produce unpredictable output without \mintinline{sql}{ORDER BY}!

\begin{minted}{sql}
SELECT surname, firstname FROM cd.members
ORDER BY surname DESC LIMIT 10 OFFSET 5;
\end{minted}

Operators \textbf{UNION}, \textbf{INTERSECT} and \textbf{EXCEPT} are used to combine outputs of multiple \mintinline{sql}{SELECT} statements, corresponding to $\cup$, $\cap$ and $\setminus$.
In all three cases, duplicate rows are removed unless \textbf{ALL} (like in: \textbf{UNION ALL}) is used.
\begin{minted}{sql}
SELECT surname FROM cd.members UNION SELECT name FROM cd.facilities;
\end{minted}

%
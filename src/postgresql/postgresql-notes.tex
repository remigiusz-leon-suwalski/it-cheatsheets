\documentclass{charun}
\title{PostgreSQL 15.3 notes, version 0.3.2}
\author{Leon Suwalski}

\usepackage{xcolor}
% copied from https://www.color-hex.com/color-palette/85828
\definecolor{sqlcolor}{HTML}{e6ffef}
\definecolor{sqlcolor2}{HTML}{038700}

% copied from https://tex.stackexchange.com/questions/132849/how-can-i-change-the-font-size-of-the-number-in-minted-environment
\BeforeBeginEnvironment{minted}{
    \begin{tcolorbox}[
        breakable,
        boxsep=5pt,
        left=0pt,
        right=0pt,
        top=0pt,
        bottom=0pt,
        boxrule=1pt,
        arc=2pt,
        outer arc=2pt,
        colback=sqlcolor,
        colframe=sqlcolor2
    ]
        \small
}%
\AfterEndEnvironment{minted}{\end{tcolorbox}}%

\begin{document}
\begin{multicols*}{2}
\maketitle
\raggedright

See: \url{https://pgexercises.com}.
Quickstart:
\begin{minted}{bash}
$ wget https://pgexercises.com/dbfiles/clubdata.sql
$ sudo -u postgres psql
# CREATE USER leo;
# ALTER USER leo SUPERUSER CREATEDB;
$ psql -U leo -f clubdata.sql -d postgres -x -q
$ sudo -u postgres psql
# \l
# \c exercises
# SELECT * FROM cd.facilities;
\end{minted}

\section{SQL Commands}
%

\color{black}
\subsection{SELECT}
\textbf{SELECT} retrieves specified columns (or all columns if \mintinline{sql}{*} is used) from table mentioned after \mintinline{sql}{FROM} keyword:
\begin{minted}{sql}
SELECT * FROM cd.facilities;
SELECT name, membercost FROM cd.facilities;
\end{minted}

\textbf{SELECT DISTINCT} eliminates duplicate rows from the result:
\begin{minted}{sql}
SELECT DISTINCT surname FROM cd.members
\end{minted}

The DISTINCT clause is unpredictable unless ORDER BY is used too:
\begin{minted}{sql}
SELECT DISTINCT ON (firstname) firstname, surname
FROM cd.members ORDER BY firstname;
\end{minted}

\textbf{AS} gives temporary aliases to the columns:
\begin{minted}{sql}
SELECT name AS facilityname FROM cd.facilities;
\end{minted}

% https://pgexercises.com/questions/basic/where.html
The \textbf{WHERE} clause indicates the condition(s) that rows must satisfy to be selected. 
Usual logical operators are available: \mintinline{sql}{AND}, \mintinline{sql}{OR}, \mintinline{sql}{NOT}.
SQL uses a three-valued logic system with true, false and null (representing unknowns).
\begin{minted}{sql}
SELECT facid, name FROM cd.facilities
WHERE membercost > 0 AND (membercost < monthlymaintenance);
\end{minted}

With '\textbf{ORDER BY} + column(s)', returned rows are sorted (without: in whatever order the system finds fastest to produce).
Here \mintinline{sql}{ASC} is the default mode.

With '\textbf{LIMIT} + count', no more rows than given count are returned.
{\color{gray}(Standard SQL uses FETCH instead).}
With '\textbf{OFFSET} + start', that many initial rows will be skipped.
Both produce unpredictable output without \mintinline{sql}{ORDER BY}!

\begin{minted}{sql}
SELECT surname, firstname FROM cd.members
ORDER BY surname DESC LIMIT 10 OFFSET 5;
\end{minted}

Operators \textbf{UNION}, \textbf{INTERSECT} and \textbf{EXCEPT} are used to combine outputs of multiple \mintinline{sql}{SELECT} statements, corresponding to $\cup$, $\cap$ and $\setminus$.
In all three cases, duplicate rows are removed unless \textbf{ALL} (like in: \textbf{UNION ALL}) is used.
\begin{minted}{sql}
SELECT surname FROM cd.members UNION SELECT name FROM cd.facilities;
\end{minted}

%

% Chapter 7. Queries
\section{Queries}
% 7.1
%

% https://www.postgresql.org/docs/15/queries-table-expressions.html
% 7.2 Table expressions
\subsection{Table expressions}
The \textbf{INNER JOIN} keyword selects records that have matching values in both tables.
You can join a table to itself!
\begin{minted}[linenos,firstnumber=last]{sql}
SELECT bks.starttime FROM cd.bookings bks
INNER JOIN cd.members mems ON mems.memid = bks.memid;
\end{minted}

\textbf{LEFT INNER JOIN} produces an output row even if a given row on the left hand table doesn't match anything.
Similarly \textbf{RIGHT INNER JOIN} produces at least one row for each row in right hand table.
Finally we have \textbf{FULL OUTER JOIN} which treats both sides of the expression as optional and is rarely used.

%
% 7.3
% 7.4
% 7.5
% 7.6
% 7.7
% 7.8

% Chapter 9. Functions and Operators
\section{Functions and operators}
% 01
% https://www.postgresql.org/docs/current/functions-comparison.html
% 9.2 Comparison
\color{red}
\subsection{Comparison}
\label{subsection_comparison}%
Comparison operators are \mintinline{sql}{<}, \mintinline{sql}{<=}, \mintinline{sql}{>}, \mintinline{sql}{>=}, \mintinline{sql}{=} (equal), \mintinline{sql}{<>} (not equal), \mintinline{sql}{!=} (alias for not equal).
For example, below query returns members who joined after the start of September 2012:
\begin{minted}[linenos,firstnumber=last]{sql}
SELECT memid FROM cd.members WHERE joindate >= '2012-09-01';
\end{minted}

See also \ref{9_9_datetime} for date/time functions and operators.

% BETWEEN, BETWEEN SYMMETRIC, IS DISTINCT FROM, IS NULL, IS UNKNOWN, IS TRUE, IS FALSE
% 03
% https://www.postgresql.org/docs/15/functions-string.html
% 9.4. String Functions and Operators
\subsection{String functions/operators}
To concatenate two strings, use pipes:
\begin{minted}[linenos,firstnumber=last]{sql}
'Post' || 'greSQL'
\end{minted}
%
% 05
% 06
%

% https://www.postgresql.org/docs/15/functions-matching.html
% 9.7. Pattern Matching
\color{black}
\subsection{Pattern matching}
% There's other ways to accomplish this task: Postgres supports regular expressions with the ~ operator, for example. Use whatever makes you feel comfortable, but do be aware that the LIKE operator is much more portable between systems.
\textbf{LIKE}\footnote{\url{https://www.postgresql.org/docs/current/functions-matching.html\#FUNCTIONS-LIKE}} provides pattern matching: underscore matches any single character, percentage matches any string (possible empty).
\begin{minted}{sql}
SELECT * FROM cd.facilities WHERE name LIKE '%Tennis%';
\end{minted}
% SIMILAR TO

%
% 08
%

% https://www.postgresql.org/docs/16/functions-datetime.html
% 9.9. Date/Time Functions and Operators
\color{red}
\subsection{Date/Time Functions and Operators}
\label{9_9_datetime}

%
% 10
% 11
% 12
% 13
% 14
% 15
% 16
% 17
%

% https://www.postgresql.org/docs/current/functions-conditional.html
% 9.18. Conditional Expressions
\color{black}
\subsection{Conditional expressions}
\textbf{CASE} is a generic conditional expression, similar to if/else:
\begin{minted}[linenos,firstnumber=last]{sql}
SELECT name,
    CASE WHEN (monthlymaintenance > 100) THEN 'expensive'
         ELSE 'cheap'
    END AS cost
FROM cd.facilities;
\end{minted}
It has a simpler form:
\begin{minted}[linenos,firstnumber=last]{sql}
...
CASE a WHEN 1 THEN 'one'
       WHEN 2 THEN 'two'
       ELSE 'other'
END
...
\end{minted}

%
% 19
% 20
%

% https://www.postgresql.org/docs/15/functions-aggregate.html
% 9.21. Aggregate Functions
\subsection{Aggregate functions}
Aggregate functions can't be used in the \mintinline{sql}{WHERE} clauses due to the order of evaluation of clauses. Therefore this works:
\begin{minted}[linenos,firstnumber=last]{sql}
SELECT * FROM cd.members
WHERE joindate = (SELECT MAX(joindate) FROM cd.members);
\end{minted}
And this does not:
\begin{minted}[linenos,firstnumber=last]{sql}
SELECT * FROM cd.members
WHERE joindate = MAX(joindate);
\end{minted}

Common aggregate functions are
\textbf{COUNT},
\textbf{MAX} (maximum),
\textbf{MIN} (minimum),
\textbf{SUM}.

%
% 22
%

% https://www.postgresql.org/docs/current/functions-subquery.html
% 9.23. Subquery Expressions
\subsection{Subquery expressions}
\textbf{IN} is a shorthand for multiple OR statements.
Right-hand side is a subquery returning one column or a list of scalar expressions; result is true if left-hand side is equal to any of the right-hand rows/expressions.
\begin{minted}[linenos,firstnumber=last]{sql}
SELECT * FROM cd.facilities WHERE facid in (1, 5);
\end{minted}

%
% 24
% 25
% 26
% 27
% 28
% 29
% 30

\newpage
\section{Old stuff}
Only unique values:
\begin{minted}{sql}
SELECT DISTINCT name FROM customer;

SELECT DISTINCT ON (column1) column_alias, column2
FROM table_name ORDER BY column1, column2;
\end{minted}

It's a good idea to specify order when using DISTINCT ON.

\hrule

Concatenation of two strings:
\begin{minted}{sql}
SELECT name || ' ' || email FROM customer;
\end{minted}

\hrule

There are also \mintinline{sql}{NULLS FIRST} and \mintinline{sql}{NULLS LAST} options.

Sorting by expression:
\begin{minted}{sql}
SELECT name, LENGTH(name) len
FROM customer ORDER BY len DESC;
\end{minted}

\hrule

Filtering:
\begin{minted}{sql}
SELECT email FROM customer WHERE name = 'Julia';
\end{minted}

\hrule

Joins:
\begin{minted}{sql}
SELECT * FROM bookings
INNER JOIN members ON members.id = bookings.id;
-- or one from:
LEFT JOIN members ON members.id = bookings.id;
RIGHT JOIN members ON members.id = bookings.id;
FULL OUTER JOIN members ON members.id = bookings.id;
\end{minted}

There are self-joins and Cartesian product (cross join) as well.

\hrule

Aggregating result form multiple rows:
\begin{minted}{sql}
SELECT customer_id, SUM (amount) FROM payment
GROUP BY customer_id;
\end{minted}

HAVING is to groups what WHERE is to rows:
\begin{minted}{sql}
SELECT customer_id, SUM (amount) FROM payment
GROUP BY customer_id HAVING SUM (amount) > 200;
\end{minted}

\hrule

\section{To be done}

Group rows using an aggregate function
\begin{minted}{sql}
SELECT c1, aggregate(c2) FROM t GROUP BY c1;
\end{minted}

Filter groups using HAVING clause
\begin{minted}{sql}
SELECT c1, aggregate(c2) FROM t GROUP BY c1 HAVING condition;
\end{minted}
    
% to be done: page 2 of https://www.postgresqltutorial.com/wp-content/uploads/2018/03/PostgreSQL-Cheat-Sheet.pdf
% to be done: page 3 of https://www.postgresqltutorial.com/wp-content/uploads/2018/03/PostgreSQL-Cheat-Sheet.pdf

\end{multicols*}
\end{document}

% https://pgexercises.com/questions/basic/ DONE
% https://pgexercises.com/questions/joins/ NOT DONE
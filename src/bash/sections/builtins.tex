%

\subsection{Shell builtins}
\begin{compactenum}
\item [\symbolbash] \commandbash{.}: evaluates a script.

\item [\symbolbash] \commandbash{[}: defines conditionals.

\item [\symbolbash] \commandbash{alias}: substitutes a string for a word.
\item [\symbolbash] \commandbash{unalias}: removes defined aliases.

\item [\symbolbash] \commandbash{bg}/\commandbash{fg}: resumes suspended job in the background/brings a background or suspended process to the foreground.

\item [\symbolbash] \commandbash{bind}: binds keys to readline function/macro.

\item [\symbolbash] \commandbash{builtin}: executes the specified shell builtin.

\item [\symbolbash] \commandbash{caller}: inspects the call stack.

\item [\symbolbash] \commandbash{cd}: changes the shell working directory.
\item [\texttt{-}] to the previous directory.

\item [\symbolbash] \commandbash{command -v}: tells how shell will invoke the specified command (like '{\small \mintinline{bash}{command -v grep}}').

\item [\symbolbash] \commandbash{compgen}: generates completions for: \texttt{a}~aliases, \texttt{b}~shell builtins, \texttt{c}~all commands, \texttt{d}~directories, \texttt{e}~exported shell variables, \texttt{f}~files, \texttt{g}~groups, \texttt{j}~jobs, \texttt{k}~shell reserved words, \texttt{s}~services, \texttt{u}~users, \texttt{v}~shell variables.
\item [\symbolbash] \commandbash{complete}: handles how commands are completed when pressing Tab.
\item [\symbolbash] \commandbash{compopt}: modifies completion options (?).

\item [\symbolbash] \commandbash{declare}: declares variables and/or gives them attributes: \texttt{a} indexed array, \texttt{A} associative array, \texttt{i} integer, \texttt{l} lowercase chars only.

\item [\symbolbash] \commandbash{dirs}: lists remembered (for popd) directories.

\item [\symbolbash] \commandbash{disown}: \dotfill ????

\item [\symbolbash] \commandbash{echo}: displays a line of text:
\item [\texttt{e}] enables interpretation of backslash escapes,
\item [\texttt{n}] does not output the trailing newline.

\item [\symbolbash] \commandbash{enable}: enables/disables shell builtins.

\item [\symbolbash] \commandbash{eval}: \dotfill ????

\item [\symbolbash] \commandbash{exec}: \dotfill ????

\item [\symbolbash] \commandbash{exit}: exits the shell with a status (0..255) or, if omitted, exit code of the last command.

\item [\symbolbash] \commandbash{export}: \dotfill ????

\item [\symbolbash] \commandbash{fc}: \dotfill ????

\item [\symbolbash] \commandbash{getopts}: \dotfill ????

\item [\symbolbash] \commandbash{hash}: \dotfill ????

\item [\symbolbash] \commandbash{help}: displays info about shell builtins.

\item [\symbolbash] \commandbash{history}: \dotfill ????

\item [\symbolbash] \commandbash{jobs}: \dotfill ????

\item [\symbolbash] \commandbash{kill}: \dotfill ????

\item [\symbolbash] \commandbash{let}: \dotfill ????

\item [\symbolbash] \commandbash{local}: \dotfill ????

\item [\symbolbash] \commandbash{logout}: exits a login shell.

\item [\symbolbash] \commandbash{mapfile}: \dotfill ????

\item [\symbolbash] \commandbash{pushd}: adds a directory to the top of the stack and performs a cd to new top directory.
\item [\symbolbash] \commandbash{popd}: removes entries from the directory stack and performs a cd to new top directory.

\item [\symbolbash] \commandbash{printf}: \dotfill ????

\item [\symbolbash] \commandbash{pwd}: prints name of current directory.

\item [\symbolbash] \commandbash{read}: \dotfill ????

\item [\symbolbash] \commandbash{readarray}: \dotfill ????

\item [\symbolbash] \commandbash{readonly}: \dotfill ????

\item [\symbolbash] \commandbash{return}: stops function and returns a value.

\item [\symbolbash] \commandbash{set}: \dotfill ????

\item [\symbolbash] \commandbash{shift}: renames positional parameters from n+1, n+2, ... to 1, 2, ... (by default, $n = 1$).

\item [\symbolbash] \commandbash{shopt}: \dotfill ????

\item [\symbolbash] \commandbash{source}: read and execute commands from filename in the current shell environmen.

\item [\symbolbash] \commandbash{suspend}: \dotfill ????

\item [\symbolbash] \commandbash{test}: checks file types and compares values.

\item [\symbolbash] \commandbash{times}: \dotfill ????

\item [\symbolbash] \commandbash{trap}: \dotfill ????

\item [\symbolbash] \commandbash{false}/\commandbash{true}: returns an un/successful result.

\item [\symbolbash] \commandbash{type}: \dotfill ????

\item [\symbolbash] \commandbash{typeset}: \dotfill ????

\item [\symbolbash] \commandbash{ulimit}: \dotfill ????

\item [\symbolbash] \commandbash{umask}: gets/sets file mode creation mask.


\item [\symbolbash] \commandbash{unset}: unsets a shell variable, removing it from memory and the shell's exported environment.

\item [\symbolbash] \commandbash{wait}: waits for process to change state.
\end{compactenum}

Useful when writing scripts/functions:

\begin{compactenum}
\item [\symbolbash] \commandbash{break}: exits from a loop.
\item [\symbolbash] \commandbash{continue}: resumes next iteration of a loop.

\item [\symbolbash] \commandbash{local}: restricts variable scope to a function and its children; without operands lists them.

\item [\symbolbash] \commandbash{return}: stops function and returns a value.

\item [\symbolbash] \commandbash{shift}: renames positional parameters from n+1, n+2, ... to 1, 2, ... (by default, $n = 1$).

\item [\symbolbash] \commandbash{true}: does nothing, successfully.
\item [\symbolbash] \commandbash{:} (colon): like true, does nothing successfully.
\item [\symbolbash] \commandbash{false}: does nothing, unsuccessfully.
\end{compactenum}

%
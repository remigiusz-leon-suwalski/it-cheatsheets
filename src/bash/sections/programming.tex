\section{Programming in Bash}
\subsection{Shebang}
The shebang (\mintinline{bash}{#!}) at the head of a script indicates an interpreter for execution, as in \mintinline{bash}{#!/bin/bash}.
(Even better is \mintinline{bash}{#!/usr/bin/env bash}, see stackoverflow question \href{https://stackoverflow.com/questions/16365130/what-is-the-difference-between-usr-bin-env-bash-and-usr-bin-bash}{\#16365130}).
Other lines starting with a \mintinline{bash}{#} are comments and thus won't be executed.


\subsection{Quoting and literals}
Inside \textbf{single quotes} \mintinline{bash}{' '} nothing is interpreted: they preserve literal values of characters enclosed within them.
A single (strong) quote may not appear between single quotes, even when escaped, but it may appear between \textbf{double (weak) quotes} \mintinline{bash}{""}.

They work quite similarly, with an exception that the shell expands any variables that appear within them (pathname expansion, process substitution and word splitting are disabled, everything else works!).

It is a very important concept: without them Bash splits lines into words at whitespace characters -- tabs and spaces.
See also: \mintinline{bash}{IFS} variable and \mintinline{bash}{$'...'} and \mintinline{bash}{$"..."} (both are Bash extensions, to be done).

Avoid backticks \mintinline{bash}{`code`} at all cost!

\subsection{Variables}
\textbf{Variable} names are case sensitive.
They can contain digits and underscores as well,
but a name starting with a digit is not allowed.
Example:
\begin{minted}{bash}
var="kind"
echo ${var}ness # kindness
\end{minted}

Special variables:
\begin{compactenum}
    \item \mintinline{bash}{$0}: name of the script itself.
    \item \mintinline{bash}{$1}, \mintinline{bash}{$2}, \ldots: the first, second, etc. argument.
    \mintinline{bash}{shift} removes first argument and advances rest of them forward.
    \item \mintinline{bash}{$*} and \mintinline{bash}{$@} denote all positional parameters (\mintinline{bash}{$*}: as a single word, \mintinline{bash}{$@}: as separate strings).
    \item \mintinline{bash}{$#}: the number of positional parameters.
    \item \mintinline{bash}{$?}: exit status of last executed command.
    \item \mintinline{bash}{$$}: the process ID of the shell.
    \item \mintinline{bash}{$!}: the process ID of last executed command.
\end{compactenum}

\subsection{Expansions}
Eight expansions happen after splitting command into tokens, in the order that they are explained.

\subsubsection{Brace expansion}
Brace expansion is used when we need to generate all possible string combinations.
Both of the commands produce the same output:
\begin{minted}{bash}
echo {I,really,love,dots}.
echo I. really. love. dots.
\end{minted}

\textbf{Warning}: it does not expand the variables (\texttt{\$var}), which is done later, but supports ranges (sequences) of characters:
\begin{minted}{bash}
echo {a..r}
a b c d e f g h i j k l m n o p q r
\end{minted}

and (maybe zero paded or with an increment rate) integers, assuming the Bash version is 4 or newer:

\begin{minted}{bash}
echo {01..09..1}.
01. 02. 03. 04. 05. 06. 07. 08. 09.
\end{minted}

\subsubsection{Tilde expansion}
There is a \textbf{tilde expansion} as well.
The expression \mintinline{bash}{~[user]} expands to the home directory of the current (or given) user.

\subsubsection{Parameter expansion}
\mintinline{bash}{${var^}}, \mintinline{bash}{${var,}}, \mintinline{bash}{${var~}} convert first character to upper, lower or reverse case (every, when caret/comma/tilde are doubled).
This has an impact on every array element.

\mintinline{bash}{${var#pattern}}, \mintinline{bash}{${var%pattern}} remove the pattern from the beginning, end of variable (greedy when hash, percent sign are doubled).
Application: extracting parts of a filename.

\mintinline{bash}{${var/pattern/string}} performs a single (when first slash is doubled: full) search and replace operation.

\mintinline{bash}{${#var}} returns length of the string.

\mintinline{bash}{${var:offset:length}} skips first \texttt{offset} characters from \texttt{var} and truncates the output to given length.
\texttt{:length} may be skipped.

Negative values separated with extra space are accepted.

\mintinline{bash}{${var:-value}} uses a default value, if \texttt{var} is empty or unset.

\mintinline{bash}{${var:=value}} does the same, but performs an assignment as well.

\mintinline{bash}{${var:+value}} uses an alternative value if \texttt{var} isn't empty or unset!

\subsubsection{Command substitution}
To execute commands in a subshell and then pass their stdout (but not stderr!), use \mintinline{bash}{$( commands )}.

(To group them without new shell, use \mintinline{bash}{{ cmds }}.)

\subsubsection{Arithmetic expansion}
The arithmetic expression \mintinline{bash}{$(( ... ))} is evaluated and expands to the result.
Bash guarantees that the output will be a one-word integer.
Exit code is zero $\iff$ output is nonzero.

\subsubsection{Process substitution}
This kind of substitution: \mintinline{bash}{<( ... )} and \mintinline{bash}{>( ... )} (not specified by POSIX!), where input or output of a command appears as a temporary file, is performed simultaneously with the following: arithmetic and parameter expansions, command substitution.

\subsubsection{Word splitting}

\subsubsection{File name expansion}
Glob patterns (composed of normal characters and \texttt{*}, \texttt{?}, \texttt{[...]}) are not regular expressions!
Bash supports extended globs too (\texttt{X(list)} where X is one from \texttt{?*+@!}).

\subsection{Streams}
There are always three default files open:
\begin{compactenum}
\item \emph{stdin} (the keyboard, file descriptor 0),
\item \emph{stdout} (the screen, file descriptor 1) and
\item \emph{stderr} (error messages, file descriptor 2).
\end{compactenum}

These \textbf{streams} can be \textbf{redirected}:
\begin{compactenum}
\item \mintinline{bash}{cmd > file} redirects to a file (overwrites),
\item \mintinline{bash}{cmd >> file} appends instead,
\item \mintinline{bash}{m>n} (\mintinline{bash}{m>&n}) redirects a file descriptor to a file
(or another file descriptor),
\item \mintinline{bash}{&>file} redirects stdout and stderr to a file,
% \item \texttt{:> file} truncates file to zero length,
\item \mintinline{bash}{|} (pipe) serves as a command chaining tool.
\end{compactenum}

\textbf{Here document} is a section of a source code file that is treated as if it were a separate file:

\begin{minted}{bash}
cat <<EOF > /path/to/your/file
   Line to be written in your file.
EOF
\end{minted}

Using \mintinline{bash}{'EOF'} instead prevents variable expansion.

\subsection{Control flow statements}
The one-line constructs \texttt{\&\&} and \texttt{||} work not like and, or ($\wedge$, $\vee$), but the \emph{if -- then -- else} statement.
Nonzero exit status denotes (usually) a failure.

\subsubsection{Loops}
\begin{minted}{bash}
while read myline; do
  echo "It says ${myline}"
done < some_file

for var in "the 1st" "the 2nd"; do
  echo "${var}"
done
for (( i = 1; i <= 10; i++ )); do
  echo "i = ${i}."
done # C-style
\end{minted}
\subsubsection{Conditionals}
Here at least one statement must be specified inside every block,
but one can use a single colon (:) as a null statement to avoid
rewriting the code.

\begin{minted}{bash}
if condition; then
  commands
elif second_condition; then
  some_commands
else
  other_commands
fi

select word in "pl" "de" "fi"
do
  echo "Your language is $word".
done
\end{minted}

There is also a case instruction:
\begin{minted}{bash}
case $language in
  bash)
    echo "Bourne Again Shell!"
  ;;
  python|haskell)
    echo "Python or Haskell!"
  ;;
  *)
    echo "Unknown language!"
  ;; # optional
esac
\end{minted}

\subsubsection{Single versus double square brackets}
The \texttt{[} command is defined by POSIX, \texttt{]} prevents its further arguments from being used.
Double brackets, the \texttt{[[} command, are a Bash extension that changes the way text ist being parsed:
\begin{compactenum}
\item \texttt{<} and \texttt{>} compare lexicographically instead of redirecting streams (inside \texttt{[ ]} they have to be escaped: \texttt{\textbackslash{}<}, \texttt{\textbackslash{}>})
\item \texttt{||} and \texttt{\&\&} are true logical operators.
\item \texttt{( )} groups commands (not a subshell!)
\item \texttt{==} without quotes does pattern matching
\item \texttt{=$\sim$} matches extended regex (no \texttt{[]}-alternative)
\end{compactenum}

\subsubsection{Testing conditions}
All of these tests but \texttt{-h} follow symlinks (!), beware.

\begin{compactenum}
\item \textbf{File tests}:
\begin{compactenum}
    \item \texttt{-e} file exists,
    \texttt{-s} file is nonempty,
    \item \texttt{-d} directory,
    \texttt{-f} regular file,
    \texttt{-h} symlink,
    \item \texttt{-b} block device,
    \texttt{-c} character device,
    \item \texttt{-p} named pipe,
    \texttt{-S} socket.
\end{compactenum}
\item \textbf{File permissions}:
\begin{compactenum}
    \item \texttt{-r} readable,
    \texttt{-w} writable,
    \texttt{-x} executable,
    \item \texttt{-u} setuid,
    \texttt{-g} setgid,
    \texttt{-k} sticky bit.
\end{compactenum}
\item \textbf{String tests}: \texttt{-z} empty, \texttt{-n} nonempty.
\item \textbf{Arithmetic tests}:
  \texttt{-eq} $=$,
  \texttt{-lt} $<$,
  \texttt{-gt} $>$,
  \texttt{-ne} $\neq$,
  \texttt{-le} $\le$,
  \texttt{-ge} $\ge$.
\end{compactenum}

Both \texttt{break} and \texttt{continue} are accepted.

%

\subsection{Shell builtins}
\begin{compactenum}
\item [\symbolbash] \commandbash{.}: evaluates a script.

\item [\symbolbash] \commandbash{[}: defines conditionals, \commandbash{test}: checks file types and compares values. Both are discouraged, use \commandbash{[[ ]]} instead.

\item [\symbolbash] \commandbash{alias}: substitutes a string for a word,\\\commandbash{unalias}: removes defined aliases.

\item [\symbolbash] \commandbash{bg}: resumes suspended job in the background; \commandbash{fg}: brings a background or suspended process to the foreground.

\item [\symbolbash] \commandbash{bind}: binds keys to readline function/macro.

\item [\symbolbash] \commandbash{builtin}: executes the specified shell builtin.

\item [\symbolbash] \commandbash{caller}: inspects the call stack.

\item [\symbolbash] \commandbash{cd}: changes the shell working directory.
\item [\texttt{-}] to the previous directory.

\item [\symbolbash] \commandbash{command -v}: tells how shell will invoke the specified command (like '{\small \mintinline{bash}{command -v grep}}').

\item [\symbolbash] \commandbash{compgen}: generates completions for: \texttt{a}~aliases, \texttt{b}~shell builtins, \texttt{c}~all commands, \texttt{d}~directories, \texttt{e}~exported shell variables, \texttt{f}~files, \texttt{g}~groups, \texttt{j}~jobs, \texttt{k}~shell reserved words, \texttt{s}~services, \texttt{u}~users, \texttt{v}~shell variables.
\item [\symbolbash] \commandbash{complete}: handles how commands are completed when pressing Tab.
\item [\symbolbash] \commandbash{compopt}: modifies completion options (?).

\item [\symbolbash] \commandbash{declare}: declares variables and/or gives them attributes: \texttt{a} indexed array, \texttt{A} associative array, \texttt{i} integer, \texttt{l} lowercase chars only.

\item [\symbolbash] \commandbash{dirs}: lists remembered (for popd) directories.

\item [\symbolbash] \commandbash{disown}: removes entries from the active jobs table; they are still connected to the terminal!

\item [\symbolbash] \commandbash{echo}: displays a line of text:
\item [\texttt{e}] enables interpretation of backslash escapes,
\item [\texttt{n}] does not output the trailing newline.

\item [\symbolbash] \commandbash{enable}: enables/disables shell builtins.

\item [\symbolbash] \commandbash{eval}: executes a command by the shell

\item [\symbolbash] \commandbash{exec}: replaces current shell with a command.

\item [\symbolbash] \commandbash{exit}: exits the shell with a status (0..255) or, if omitted, exit code of the last command.

\item [\symbolbash] \commandbash{export}: marks variables to be passed to subshells; without args prints such variables.

\item [\symbolbash] \commandbash{fc}: \dotfill ????

\item [\symbolbash] \commandbash{getopts}: \dotfill ????

\item [\symbolbash] \commandbash{hash}: \dotfill ????

\item [\symbolbash] \commandbash{help}: displays info about shell builtins.

\item [\symbolbash] \commandbash{history}: lists previously ran commands.

\item [\symbolbash] \commandbash{jobs}: \dotfill ????

\item [\symbolbash] \commandbash{kill}: \dotfill ????

\item [\symbolbash] \commandbash{let}: evaluates arithmetic expressions.

\item [\symbolbash] \commandbash{local}: \dotfill ????

\item [\symbolbash] \commandbash{logout}: exits a login shell.

\item [\symbolbash] \commandbash{mapfile}: \dotfill ????

\item [\symbolbash] \commandbash{pushd}: adds a directory to the top of the stack and performs a cd to new top directory.
\item [\symbolbash] \commandbash{popd}: removes entries from the directory stack and performs a cd to new top directory.

\item [\symbolbash] \commandbash{printf}: \dotfill ????

\item [\symbolbash] \commandbash{pwd}: prints name of current directory.

\item [\symbolbash] \commandbash{read}: \dotfill ????

\item [\symbolbash] \commandbash{readarray}: \dotfill ????

\item [\symbolbash] \commandbash{readonly}: \dotfill ????

\item [\symbolbash] \commandbash{return}: stops function and returns a value.

\item [\symbolbash] \commandbash{set}: \dotfill ????

\item [\symbolbash] \commandbash{shift}: renames positional parameters from n+1, n+2, ... to 1, 2, ... (by default, $n = 1$).

\item [\symbolbash] \commandbash{shopt}: \dotfill ????

\item [\symbolbash] \commandbash{source}: read and execute commands from filename in the current shell environmen.

\item [\symbolbash] \commandbash{suspend}: suspends current shell until receiving of a SIGCONT signal.

\item [\symbolbash] \commandbash{times}: \dotfill ????

\item [\symbolbash] \commandbash{trap}: \dotfill ????

\item [\symbolbash] \commandbash{false}/\commandbash{true}: returns an un/successful result.

\item [\symbolbash] \commandbash{type}: \dotfill ????

\item [\symbolbash] \commandbash{typeset}: \dotfill ????

\item [\symbolbash] \commandbash{ulimit}: modifies shell resource limits.

\item [\symbolbash] \commandbash{umask}: gets/sets file mode creation mask.

\item [\symbolbash] \commandbash{unset}: unsets a shell variable, removing it from memory and the shell's exported environment.

\item [\symbolbash] \commandbash{wait}: waits for process to change state.
\end{compactenum}

Useful when writing scripts/functions:

\begin{compactenum}
\item [\symbolbash] \commandbash{break}: exits from a loop.
\item [\symbolbash] \commandbash{continue}: resumes next iteration of a loop.

\item [\symbolbash] \commandbash{local}: restricts variable scope to a function and its children; without operands lists them.

\item [\symbolbash] \commandbash{return}: stops function and returns a value.

\item [\symbolbash] \commandbash{shift}: renames positional parameters from n+1, n+2, ... to 1, 2, ... (by default, $n = 1$).

\item [\symbolbash] \commandbash{true}: does nothing, successfully.
\item [\symbolbash] \commandbash{:} (colon): like true, does nothing successfully.
\item [\symbolbash] \commandbash{false}: does nothing, unsuccessfully.
\end{compactenum}

%

%
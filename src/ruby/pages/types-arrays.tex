
\subsection{Arrays}
Arrays are zero-indexed, can contain all kinds of objects and have a defined order.
% \begin{minted}{ruby}
% words = ["one", "two", "three"]
% words << "four"
% puts words[10] # retrieving an element that does not exist gives nil
% [1, 2] + [5, 6] # [1, 2, 5, 6]
% ["A", "Be"] * 2 # ["A", "Be", "A", "Be"]
% & # intersection
% # vvv methods vvv
% .first .last .length .sort .compact .index(...) .rotate(...), .transpose
% \end{minted}

\textbf{count} returns a count of specified elements.
With no argument and no block, returns the count of all elements.
With argument \mintinline{ruby}{obj}, returns the count of elements \mintinline{ruby}{==} to \mintinline{ruby}{obj}.
With no argument and a block given, calls the block with each element; returns the count of elements for which the block returns a truthy value:
\begin{minted}{ruby}
[0, 1, 2].count # => 3
[0, 1, 2, 0.0].count(0) # => 2
[0, 1, 2, 3].count {|element| element > 1} # => 2
\end{minted}

\textbf{sum}.
When a block is given, it is called with each element and the block’s return value (instead of the element itself) is used as the addend.
When no block is given, returns the object equivalent to:
\begin{minted}{ruby}
sum = init
array.each {|element| sum += element }
\end{minted}
Note: \mintinline{ruby}{join} (\mintinline{ruby}{flatten}) may be faster for an array of strings (of arrays).

\textbf{uniq} returns a new Array containing those elements from self that are not duplicates, the first occurrence always being retained.
With no block given, identifies and omits duplicates using method \mintinline{ruby}{eql?} to compare:
With a block given, calls the block for each element; identifies (using method eql?) and omits duplicate values, that is, those elements for which the block returns the same value:
\begin{minted}{ruby}
['a', 'aa', 'aaa', 'b', 'bb', 'bbb'].uniq {|element| element.size } # => ["a", "aa", "aaa"]
[0, 0, 1, 1, 2, 2].uniq # => [0, 1, 2]
\end{minted}

% return true if [2,3,5,7,11,17].include? num

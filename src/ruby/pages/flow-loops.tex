
\subsection{Loops}
Ruby has for loops (but you should almost never use them), while/until loops and an infinite \mintinline{ruby}{loop} loop (which is rarely used).
\mintinline{ruby}{break} and \mintinline{ruby}{next} have usual meaning.
\mintinline{ruby}{redo} goes to the start of current iteration and does it again.

\begin{multicols}{3}
\begin{minted}{ruby}
while applies?
  #...
  break
end

do_something while applies?
\end{minted}
\columnbreak

\begin{minted}{ruby}
until applies?
  do_something
  break
end

do_something until applies?
\end{minted}
\columnbreak

\begin{minted}{ruby}
loop do
  do_something
  break
end
\end{minted}
\end{multicols}

(Nie wiem, jak to się nazywa)
\begin{minted}{ruby}
[1, 2, 3].each do |num|
  do_something
end

[1, 2, 3].each.with_index do |num, idx|
  do_something
end
\end{minted}
Instead of giving a definition of what iterator is, we present several examples:
\begin{itemize}
  \item \mintinline{ruby}{5.times { puts "Hi!" }}
  \item \mintinline{ruby}{1.upto(5) { puts "Hi!" }}
  \item \mintinline{ruby}{5.down­to(1) { puts "Hi!" }}
  \item \mintinline{ruby}{(1..5).each { puts "Hi!" }}
  \item \mintinline{ruby}{xs.each { |x| puts 1+x }}
\end{itemize}  
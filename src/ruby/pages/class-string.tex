\subsection{class String}
% https://ruby-doc.org/3.3.6/String.html#method-i-split
A \mintinline{ruby}{String} object has an arbitrary sequence of bytes, typically representing text or binary data.
Class \mintinline{ruby}{String} includes module \mintinline{ruby}{Comparable}.
In general, if there exist both bang and non-bang version of method, the bang! mutates and the non-bang! does not. However, a method without a bang can also mutate, such as \mintinline{ruby}{String#replace}.
\mintinline{ruby}{String} objects differ from \mintinline{ruby}{Symbol} objects in that Symbol objects are designed to be used as identifiers, instead of text or data.

\subsubsection{Methods for converting to new String}
Each of these methods returns a new String based on self, often just a modified copy of self.

Extension

*: Returns the concatenation of multiple copies of self,

+: Returns the concatenation of self and a given other string.

center: Returns a copy of self centered between pad substring.

concat: Returns the concatenation of self with given other strings.

prepend: Returns the concatenation of a given other string with self.

ljust: Returns a copy of self of a given length, right-padded with a given other string.

rjust: Returns a copy of self of a given length, left-padded with a given other string.

Encoding

b: Returns a copy of self with ASCII-8BIT encoding.

scrub: Returns a copy of self with each invalid byte replaced with a given character.

unicode\_normalize: Returns a copy of self with each character Unicode-normalized.


Substitution

dump: Returns a copy of self with all non-printing characters replaced by xHH notation and all special characters escaped.

undump: Returns a copy of self with all \\xNN notation replace by \\uNNNN notation and all escaped characters unescaped.

sub: Returns a copy of self with the first substring matching a given pattern replaced with a given replacement string;.

gsub: Returns a copy of self with each substring that matches a given pattern replaced with a given replacement string.

succ, next: Returns the string that is the successor to self.

\textbf{reverse} zwraca wszystkie znaki od tyłu (podobnie jak dla tablic).
Ma wersję z wykrzyknikiem.
\begin{minted}{ruby}
'stressed'.reverse # => "desserts"
\end{minted}

tr: Returns a copy of self with specified characters replaced with specified replacement characters.

tr\_s: Returns a copy of self with specified characters replaced with specified replacement characters, removing duplicates from the substrings that were modified.

%: Returns the string resulting from formatting a given object into self

Casing.
\begin{minted}{ruby}
'hello World!'.capitalize # => "Hello world!"
'hello World!'.downcase   # => "hello world!"
'hello World!'.upcase     # => "HELLO WORLD!"
'hello World!'.swapcase   # => "HELLO wORLD!"
\end{minted}

Deletion

delete: Returns a copy of self with characters removed

delete\_prefix: Returns a copy of self with a given prefix removed.

delete\_suffix: Returns a copy of self with a given suffix removed.

lstrip: Returns a copy of self with leading whitespace removed.

rstrip: Returns a copy of self with trailing whitespace removed.

strip: Returns a copy of self with leading and trailing whitespace removed.

chomp: Returns a copy of self with a trailing record separator removed, if found.

chop: Returns a copy of self with trailing newline characters or the last character removed.

squeeze: Returns a copy of self with contiguous duplicate characters removed.

[], slice: Returns a substring determined by a given index, start/length, or range, or string.

byteslice: Returns a substring determined by a given index, start/length, or range.

chr: Returns the first character.

Duplication

to\_s, to\_str: If self is a subclass of String, returns self copied into a String; otherwise, returns self.


\subsubsection{Methods for converting to non-String}
Each of these methods converts the contents of self to a non-String.

Characters, Bytes, and Clusters

bytes: Returns an array of the bytes in self.

chars: Returns an array of the characters in self.

codepoints: Returns an array of the integer ordinals in self.

getbyte: Returns an integer byte as determined by a given index.

% grapheme_clusters: Returns an array of the grapheme clusters in self.

lines: Returns an array of the lines in self, as determined by a given record separator.

partition: Returns a 3-element array determined by the first substring that matches a given substring or regexp,

rpartition: Returns a 3-element array determined by the last substring that matches a given substring or regexp,

\textbf{split} returns an array of substrings determined by a given delimiter – regexp or string.
\begin{minted}{ruby}
'abc def ghi'.split(' ') # => ["abc", "def", "ghi"]
"abc \n\tdef\t\n  ghi".split(' ') # => ["abc", "def", "ghi"]
'abc  def   ghi'.split(' ') # => ["abc", "def", "ghi"]
''.split(' ') # => []
'abracadabra'.split('ab') # => ["", "racad", "ra"]
'aaabcdaaa'.split('a') # => ["", "", "", "bcd"]
''.split('a') # => []
'3.14159'.split('1') # => ["3.", "4", "59"]
'!@#$%^$&*($)_+'.split('$') # => ["!@#", "%^", "&*(", ")_+"]

# regexp delimeter
'abracadabra'.split(/ab/) # => ["", "racad", "ra"]
'aaabcdaaa'.split(/a/) # => ["", "", "", "bcd"]
'aaabcdaaa'.split(//) # => ["a", "a", "a", "b", "c", "d", "a", "a", "a"]
'1 + 1 == 2'.split(/\W+/) # => ["1", "1", "2"]

# with a limit - a positive integer
'aaabcdaaa'.split('a', 1) # => ["aaabcdaaa"]
'aaabcdaaa'.split('a', 2) # => ["", "aabcdaaa"]
'aaabcdaaa'.split('a', 5) # => ["", "", "", "bcd", "aa"]
'aaabcdaaa'.split('a', 7) # => ["", "", "", "bcd", "", "", ""]
'aaabcdaaa'.split('a', 8) # => ["", "", "", "bcd", "", "", ""]

# with a block
'abc def ghi'.split(' ') {|substring| p substring }
\end{minted}

Matching

scan: Returns an array of substrings matching a given regexp or string, or, if a block given, passes each matching substring to the block.

unpack: Returns an array of substrings extracted from self according to a given format.

unpack1: Returns the first substring extracted from self according to a given format.

Numerics

hex: Returns the integer value of the leading characters, interpreted as hexadecimal digits.

oct: Returns the integer value of the leading characters, interpreted as octal digits.

ord: Returns the integer ordinal of the first character in self.

% to_i: Returns the integer value of leading characters, interpreted as an integer.

% to_f: Returns the floating-point value of leading characters, interpreted as a floating-point number.

Strings and Symbols

inspect: Returns copy of self, enclosed in double-quotes, with special characters escaped.

% to_sym, intern: Returns the symbol corresponding to self.
\subsection{moduł Enumerable}
% https://ruby-doc.org/3.3.6/Enumerable.html#method-i-inject
Rdzenne klasy \mintinline{ruby}{ARGF}, \mintinline{ruby}{Array}, \mintinline{ruby}{Dir}, \mintinline{ruby}{Enumerator}, \mintinline{ruby}{ENV}, \mintinline{ruby}{Hash}, \mintinline{ruby}{IO}, \mintinline{ruby}{Range}, \mintinline{ruby}{Struct} oraz klasy biblioteki podstawowej \mintinline{ruby}{CSV}, \mintinline{ruby}{Set} rozszerzają \mintinline{ruby}{Enumerable}.

\subsubsection{Methods for Querying}
These methods return information about the Enumerable other than the elements themselves:

include?, member?: Returns true if self == object, false otherwise.

\textbf{all} (\textbf{any}) sprawdza, czy każdy (pewne) elementy są prawdziwe:
\begin{minted}{ruby}
# bez argumentu, bez bloku
(1..4).all? # => true
(1..4).any? # => true
%w[a b c d].all? # => true
%w[a b c d].any? # => true
[1, 2, nil].all?
['a', 'b', false].all?
[1, false, nil].any? # => true
[].all? # => true
[].any?

# z argumentem
(1..4).all?(Integer) # => true
(1..4).all?(Numeric) # => true
(1..4).all?(Float)
[nil, false, 0].any?(Integer) # => true
[nil, false, 0].any?(Numeric) # => true
[nil, false, 0].any?(Float)
%w[bar baz bat bam].all?(/ba/) # => true
%w[bar baz bat bam].all?(/bar/)
%w[bar baz bat bam].all?('ba')
%w[bar baz bat bam].any?(/m/) # => true
%w[bar baz bat bam].any?(/foo/)
%w[bar baz bat bam].any?('ba')
{foo: 0, bar: 1, baz: 2}.all?(Array) # => true
{foo: 0, bar: 1, baz: 2}.all?(Hash)
{foo: 0, bar: 1, baz: 2}.any?(Array) # => true
{foo: 0, bar: 1, baz: 2}.any?(Hash)
[].all?(Integer) # => true
[].any?(Integer)

# z blokiem
(1..4).all? {|element| element < 5 } # => true
(1..4).all? {|element| element < 4 }
(1..4).any? {|element| element < 2 } # => true
(1..4).any? {|element| element < 1 }
{foo: 0, bar: 1, baz: 2}.all? {|key, value| value < 3 } # => true
{foo: 0, bar: 1, baz: 2}.all? {|key, value| value < 2 }
{foo: 0, bar: 1, baz: 2}.any? {|key, value| value < 1 } # => true
{foo: 0, bar: 1, baz: 2}.any? {|key, value| value < 0 }
\end{minted}

none?: Returns true if no element meets a specified criterion; false otherwise.

one?: Returns true if exactly one element meets a specified criterion; false otherwise.

count: Returns the count of elements, based on an argument or block criterion, if given.

tally: Returns a new Hash containing the counts of occurrences of each element.

Methods for Fetching
These methods return entries from the Enumerable, without modifying it:

Leading, trailing, or all elements:

entries, to\_a: Returns all elements.

first: Returns the first element or leading elements.

take: Returns a specified number of leading elements.

drop: Returns a specified number of trailing elements.

take\_while: Returns leading elements as specified by the given block.

drop\_while: Returns trailing elements as specified by the given block.

\subsubsection{Minimum and maximum value elements:
}
\textbf{max}, \textbf{min}, \textbf{minmax} zwracają najmniejszy i/lub największy element(y).
\begin{minted}{ruby}
[0, 1, 2].max # => 2
[0, 1, 2, 3].max(3) # => [3, 2, 1]
[0, 1, 2, 3].max(6) # => [3, 2, 1, 0]
['2', '33', '000'].min {|a, b| a.size <=> b.size } # => "2"
[0, 1, 2].minmax # => [0, 2]
\end{minted}

min\_by: Returns the smallest element, as determined by the given block.

max\_by: Returns the largest element, as determined by the given block.

minmax\_by: Returns the smallest and largest elements, as determined by the given block.

Groups, slices, and partitions:

group\_by: Returns a Hash that partitions the elements into groups.

partition: Returns elements partitioned into two new Arrays, as determined by the given block.

slice\_after: Returns a new Enumerator whose entries are a partition of self, based either on a given object or a given block.

slice\_before: Returns a new Enumerator whose entries are a partition of self, based either on a given object or a given block.

slice\_when: Returns a new Enumerator whose entries are a partition of self based on the given block.

chunk: Returns elements organized into chunks as specified by the given block.

chunk\_while: Returns elements organized into chunks as specified by the given block.

\subsubsection{Methods for Searching and Filtering}
These methods return elements that meet a specified criterion:

\textbf{find} (znane też jako \textbf{detect}) zwraca pierwszy element, dla którego blok jest prawdziwy.
\begin{minted}{ruby}
(0..9).find {|element| element > 2} # => 3
\end{minted}

find\_all, filter, select: Returns elements selected by the block.

find\_index: Returns the index of an element selected by a given object or block.

reject: Returns elements not rejected by the block.

uniq: Returns elements that are not duplicates.

Methods for Sorting
These methods return elements in sorted order:

sort: Returns the elements, sorted by <=> or the given block.

sort\_by: Returns the elements, sorted by the given block.

Methods for Iterating
each\_entry: Calls the block with each successive element (slightly different from each).

each\_with\_index: Calls the block with each successive element and its index.

each\_with\_object: Calls the block with each successive element and a given object.

each\_slice: Calls the block with successive non-overlapping slices.

each\_cons: Calls the block with successive overlapping slices. (different from each\_slice).

reverse\_each: Calls the block with each successive element, in reverse order.

Other Methods
map, collect: Returns objects returned by the block.

filter\_map: Returns truthy objects returned by the block.

flat\_map, collect\_concat: Returns flattened objects returned by the block.

grep: Returns elements selected by a given object or objects returned by a given block.

grep\_v: Returns elements selected by a given object or objects returned by a given block.

reduce, inject: Returns the object formed by combining all elements.

sum: Returns the sum of the elements, using method +.

zip: Combines each element with elements from other enumerables; returns the n-tuples or calls the block with each.

cycle: Calls the block with each element, cycling repeatedly.


none?: Returns true if no element meets a specified criterion; false otherwise.
\begin{minted}{ruby}
#none?
\end{minted}

one?: Returns true if exactly one element meets a specified criterion; false otherwise.
\begin{minted}{ruby}
#one?
\end{minted}

count: Returns the count of elements, based on an argument or block criterion, if given.
\begin{minted}{ruby}
#count
\end{minted}

tally: Returns a new Hash containing the counts of occurrences of each element.
Method ...
\begin{minted}{ruby}
#tally
\end{minted}

Method ...
\begin{minted}{ruby}
#chain
\end{minted}

Method ...
\begin{minted}{ruby}
#chunk
\end{minted}

Method ...
\begin{minted}{ruby}
#chunk_while
\end{minted}

Method ...
\begin{minted}{ruby}
#collect
\end{minted}

Method ...
\begin{minted}{ruby}
#collect_concat
\end{minted}

Method ...
\begin{minted}{ruby}
#compact
\end{minted}



Method ...
\begin{minted}{ruby}
#cycle
\end{minted}

Method ...
\begin{minted}{ruby}
#detect
\end{minted}

Method ...
\begin{minted}{ruby}
#drop
\end{minted}

Method ...
\begin{minted}{ruby}
#drop_while
\end{minted}

Method ...
\begin{minted}{ruby}
#each_cons
\end{minted}

Method ...
\begin{minted}{ruby}
#each_entry
\end{minted}

Method ... % TODO
\begin{minted}{ruby}
#each_slice
\end{minted}

Method ...
\begin{minted}{ruby}
#each_with_index
\end{minted}

Method ...
\begin{minted}{ruby}
#each_with_object
\end{minted}

Method ...
\begin{minted}{ruby}
#entries
\end{minted}

Method ...
\begin{minted}{ruby}
#filter
\end{minted}

Method ...
\begin{minted}{ruby}
#filter_map
\end{minted}

Method ...
\begin{minted}{ruby}
#find_all
\end{minted}

Method ...
\begin{minted}{ruby}
#find_index
\end{minted}

Method ...
\begin{minted}{ruby}
#first
\end{minted}

Method ...
\begin{minted}{ruby}
#flat_map
\end{minted}

Method ...
\begin{minted}{ruby}
#grep
\end{minted}

Method ...
\begin{minted}{ruby}
#grep_v
\end{minted}

Method ...
\begin{minted}{ruby}
#group_by
\end{minted}

\textbf{include?} zwraca, czy którykolwiek element wyliczanki jest taki sam.
\begin{minted}{ruby}
(1..4).include?(2) # => true
(1..4).include?(5)
(1..4).include?('2')
%w[a b c d].include?('b') # => true
%w[a b c d].include?('2')
{foo: 0, bar: 1, baz: 2}.include?(:foo) # => true
{foo: 0, bar: 1, baz: 2}.include?('foo')
{foo: 0, bar: 1, baz: 2}.include?(0)
\end{minted}


Method ...
\begin{minted}{ruby}
#lazy
\end{minted}

Method ...
\begin{minted}{ruby}
#map
\end{minted}

Method ...
\begin{minted}{ruby}
#max
\end{minted}

Method ...
\begin{minted}{ruby}
#max_by
\end{minted}

Method ...
\begin{minted}{ruby}
#member?
\end{minted}

Method ...
\begin{minted}{ruby}
#min
\end{minted}

Method ...
\begin{minted}{ruby}
#min_by
\end{minted}

Method ...
\begin{minted}{ruby}
#minmax
\end{minted}

Method ...
\begin{minted}{ruby}
#minmax_by
\end{minted}

Method ...
\begin{minted}{ruby}
#partition
\end{minted}


Method ...
\begin{minted}{ruby}
#reject
\end{minted}

Method ...
\begin{minted}{ruby}
#reverse_each
\end{minted}

Method ...
\begin{minted}{ruby}
#select
\end{minted}

Method ...
\begin{minted}{ruby}
#slice_after
\end{minted}

Method ...
\begin{minted}{ruby}
#slice_before
\end{minted}

Method ...
\begin{minted}{ruby}
#slice_when
\end{minted}

Method ...
\begin{minted}{ruby}
#sort
\end{minted}

Method ...
\begin{minted}{ruby}
#sort_by
\end{minted}

Method ...
\begin{minted}{ruby}
#sum
\end{minted}

Method ...
\begin{minted}{ruby}
#take
\end{minted}

Method ...
\begin{minted}{ruby}
#take_while
\end{minted}

Method ...
\begin{minted}{ruby}
#to_a
\end{minted}

Method ...
\begin{minted}{ruby}
#to_h
\end{minted}

Method ...
\begin{minted}{ruby}
#to_set
\end{minted}

Method ...
\begin{minted}{ruby}
#uniq
\end{minted}

Method ...
\begin{minted}{ruby}
#zip
\end{minted}

\textbf{inject} returns the object formed by combining all elements.
\textbf{reduce} is an alias.
\begin{minted}{ruby}
# with method-name argument
(1..4).inject(:+) # => 10
(1..4).inject(10, :+) # => 20

# with a block
(1..4).inject {|sum, n| sum + n*n } # => 30
(1..4).inject(2) {|sum, n| sum + n*n } # => 32

# with initial operand
(1..4).inject(2, :+) # => 12
(1..4).inject(2.0, :+) # => 12.0
('a'..'d').inject('foo', :+) # => "fooabcd"
%w[a b c].inject(['x'], :push) # => ["x", "a", "b", "c"]
(1..4).inject(Complex(2, 2), :+) # => (12+2i)
(1..4).inject do |memo, element|
  p "Memo: #{memo}; element: #{element}"
  memo + element
end # => 10
\end{minted}
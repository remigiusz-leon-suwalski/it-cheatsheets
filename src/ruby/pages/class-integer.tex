\subsection{klasa Integer}
Obiekt tej klasy przedstawia liczbę całkowitą.
Klasa dziedziczy z klasy \mintinline{ruby}{Numeric}.

% Querying
% allbits?: Returns whether all bits in self are set.
% anybits?: Returns whether any bits in self are set.
% nobits?: Returns whether no bits in self are set.

% Comparing
% #<: Returns whether self is less than the given value.
% #<=: Returns whether self is less than or equal to the given value.
% #<=>: Returns a number indicating whether self is less than, equal to, or greater than the given value.
% == (aliased as ===): Returns whether self is equal to the given
% value.
% #>: Returns whether self is greater than the given value.
% #>=: Returns whether self is greater than or equal to the given value.

% Converting
% ::sqrt: Returns the integer square root of the given value.
% ::try_convert: Returns the given value converted to an Integer.
% % (aliased as modulo): Returns self modulo the given value.
% #&: Returns the bitwise AND of self and the given value.
% *: Returns the product of self and the given value.
% **: Returns the value of self raised to the power of the given value.
% +: Returns the sum of self and the given value.
% -: Returns the difference of self and the given value.
% #/: Returns the quotient of self and the given value.
% <<: Returns the value of self after a leftward bit-shift.
% >>: Returns the value of self after a rightward bit-shift.
% []: Returns a slice of bits from self.
% #^: Returns the bitwise EXCLUSIVE OR of self and the given value.
% ceil: Returns the smallest number greater than or equal to self.
% chr: Returns a 1-character string containing the character represented by the value of self.

\textbf{digits} zwraca tablicę cyfr, domyślnie przy podstawie 10.
Nie ma sensu dla ujemnych liczb i podstaw mniejszych niż 2.
\begin{minted}{ruby}
12345.digits      # => [5, 4, 3, 2, 1]
12345.digits(7)   # => [4, 6, 6, 0, 5]
12345.digits(100) # => [45, 23, 1]
\end{minted}

% div: Returns the integer result of dividing self by the given value.
% divmod: Returns a 2-element array containing the quotient and remainder results of dividing self by the given value.
% fdiv: Returns the Float result of dividing self by the given value.
% floor: Returns the greatest number smaller than or equal to self.
% pow: Returns the modular exponentiation of self.
% pred: Returns the integer predecessor of self.
% remainder: Returns the remainder after dividing self by the given value.
% round: Returns self rounded to the nearest value with the given precision.
% succ (aliased as next): Returns the integer successor of self.
% to_f: Returns self converted to a Float.
% to_s (aliased as inspect): Returns a string containing the place-value representation of self in the given radix.
% truncate: Returns self truncated to the given precision.
% #|: Returns the bitwise OR of self and the given value.

% Other
% downto: Calls the given block with each integer value from self down to the given value.
% times: Calls the given block self times with each integer in (0..self-1).
% upto: Calls the given block with each integer value from self up to the given value.

% ::sqrt
% ::try_convert
% #%
% #&
% #*
% #**
% #+
% #-
% #-@
% #/
% #<
% #<<
% #<=
% #<=>
% #==
% #===
% #>
% #>=
% #>>
% #[]
% #^
% #abs
% #allbits?
% #anybits?
% #bit_length
% #ceil
% #ceildiv
% #chr
% #coerce
% #denominator
% #digits
% #div
% #divmod
% #downto
% #even?
% #fdiv
% #floor
% #gcd
% #gcdlcm
% #inspect
% #integer?
% #lcm
% #magnitude
% #modulo
% #next
% #nobits?
% #numerator
% #odd?
% #ord
% #pow
% #pred
% #rationalize
% #remainder
% #round
% #size
% #succ
% #times
% #to_f
% #to_i
% #to_int
% #to_r
% #to_s
% #truncate
% #upto
% #zero?
% #|
% #~
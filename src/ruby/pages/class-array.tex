\subsection{class Array}
Array -- tablica.
% https://ruby-doc.org/3.3.6/Array.html#method-i-join

\subsubsection{Methods for Creating an Array}
Tekst.

\subsubsection{Methods for Querying}
Tekst.

\subsubsection{Methods for Comparing}
Tekst.

\subsubsection{Methods for Fetching}
{\footnotesize
[]: Returns one or more elements.
fetch: Returns the element at a given offset.
first: Returns one or more leading elements.
last: Returns one or more trailing elements.
}

\textbf{max}, \textbf{min}, \textbf{minmax} zwracają najmniejszy i/lub największy element(y).
\begin{minted}{ruby}
[0, 1, 2].max # => 2
[0, 1, 2, 3].max(3) # => [3, 2, 1]
[0, 1, 2, 3].max(6) # => [3, 2, 1, 0]
['2', '33', '000'].min {|a, b| a.size <=> b.size } # => "2"
[0, 1, 2].minmax # => [0, 2]
\end{minted}
    
{\footnotesize
assoc: Returns the first element that is an array whose first element == a given object.
rassoc: Returns the first element that is an array whose second element == a given object.
at: Returns the element at a given offset.
values\_at: Returns the elements at given offsets.
dig: Returns the object in nested objects that is specified by a given index and additional arguments.
drop: Returns trailing elements as determined by a given index.
take: Returns leading elements as determined by a given index.
drop\_while: Returns trailing elements as determined by a given block.
take\_while: Returns leading elements as determined by a given block.
slice: Returns consecutive elements as determined by a given argument.
sort: Returns all elements in an order determined by <=> or a given block.
}

\textbf{reverse} zwraca wszystkie elementy od tyłu.
Ma wersję z wykrzyknikiem.
\begin{minted}{ruby}
['foo', 'bar', 'two'].reverse # => ["two", "bar", "foo"]
\end{minted}

{\footnotesize
compact: Returns an array containing all non-nil elements.
select, filter: Returns an array containing elements selected by a given block.
uniq: Returns an array containing non-duplicate elements.
rotate: Returns all elements with some rotated from one end to the other.
bsearch: Returns an element selected via a binary search as determined by a given block.
bsearch\_index: Returns the index of an element selected via a binary search as determined by a given block.
sample: Returns one or more random elements.
shuffle: Returns elements in a random order.
}

\subsubsection{Methods for Assigning}
Tekst.

\subsubsection{Methods for Deleting}
Tekst.

\subsubsection{Methods for Combining}
Tekst.

\subsubsection{Methods for Iterating}
Tekst.

\subsubsection{Methods for Converting}
\textbf{map}, \textbf{collect} zwraca nową tablicę wartości funkcji dla każdego elementu starej tablicy.
\textbf{collect} jest aliasem.
Obydwie metody mają swoje wersje z wykrzyknikiem (\textbf{map!}, \textbf{collect!}).
\begin{minted}{ruby}
a = [:foo, 'bar', 2].map { |element| element.class } # => [Symbol, String, Integer]
\end{minted}

{\footnotesize
flatten: Returns an array that is a recursive flattening of self.
flatten!: Replaces each nested array in self with the elements from that array.
inspect, to\_s: Returns a new String containing the elements.
}

\textbf{join} łączy argumenty (domyślnie: niczym) i zwraca nowy łańcuch znaków (String).
\begin{minted}{ruby}
[:foo, 'bar', 2].join # => "foobar2"
[:foo, 'bar', 2].join(" ") # => "foo bar 2"
[:foo, [:bar, [:baz, :bat]]].join # => "foobarbazbat"
\end{minted}

{\footnotesize
to\_a: Returns self or a new array containing all elements.
to\_ary: Returns self.
to\_h: Returns a new hash formed from the elements.
transpose: Transposes self, which must be an array of arrays.
zip: Returns a new array of arrays containing self and given arrays; follow the link for details.
}
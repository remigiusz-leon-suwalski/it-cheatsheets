\subsection{klasa Complex}
Objekt tej klasy jest liczbą zespoloną we współrzędnych kartezjańskich lub biegunowych.
Można je dodawać, odejmować, mnożyć, dzielić i potęgować i sprzęgać (\textbf{conj}, \textbf{conjugate}).
Porównywanie (\mintinline{ruby}{<=>}) ma sens tylko, gdy obydwie liczby są rzeczywiste; w przeciwnym razie zwraca \mintinline{ruby}{nil}.
Sprawdzanie równości (\mintinline{ruby}{==}) ma sens zawsze.
\begin{minted}{ruby}
3 + 4 * Complex::I
Complex.rect(3, 4)
3+4i # => (3+4i)

Complex.polar(1, Math::PI/3) # => (0.5000000000000001+0.8660254037844386i)

Complex.rect(1, 2).conj # => (1-2i)
\end{minted}

Liczby zespolone można zmieniać w zespolone (\textbf{to\_c}), rzeczywiste (\textbf{to\_f}), całkowite (\textbf{to\_i}), wymierne (\textbf{to\_r}) albo łańcuchy znaków (\textbf{to\_s}).

\textbf{real}, \textbf{imag} (znane też jako \textbf{imaginary}) zwraca część rzeczywistą i urojoną.
\textbf{rect} (znane też jako \textbf{rectangular}) zwraca obydwie części.
\begin{minted}{ruby}
Complex(9, -4).real # => 9
Complex(9, -4).imag # => -4
Complex(9, -4).rect # => [9, -4]
\end{minted}

\textbf{abs} (znane też jako \textbf{magnitude}) zwraca moduł, wartość bezwzględną; \textbf{abs2} zwraca jego kwadrat.
\textbf{arg} (znane też jako \textbf{angle}, \textbf{phase}) zwraca argument (kąt) w radianach.
\textbf{polar} zwraca moduł i argument.
\begin{minted}{ruby}
Complex.polar(3, Math::PI/2).abs  # => 3.0
Complex.polar(3, Math::PI/2).arg  # => 1.57079632679489660
\end{minted}

% TODO * ** + - / <=> ==

\textbf{Do zrobienia później}.
denominator, numerator.
fdiv.
hash.
finite?, infinite?
inspect.
rationalize.

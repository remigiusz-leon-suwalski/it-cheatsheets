\documentclass{a5charun}
\title{Ruby 3.3.0 notes, version 0.2.0}
\author{Leon Suwalski}

\setlength{\multicolsep}{6.0pt plus 1.0pt minus 1.0pt}% 50% of original values

\usepackage{xcolor}
% copied from https://www.color-hex.com/color-palette/85828
\definecolor{rubycolor}{HTML}{ffecec}
\definecolor{rubycolor2}{HTML}{990000}

% copied from https://tex.stackexchange.com/questions/132849/how-can-i-change-the-font-size-of-the-number-in-minted-environment
\BeforeBeginEnvironment{minted}{\begin{tcolorbox}[breakable,boxsep=5pt,left=0pt,right=0pt,top=0pt,bottom=0pt,boxrule=0.5pt,arc=0pt,outer arc=0pt,colback=rubycolor,colframe=rubycolor2]\small}%
\AfterEndEnvironment{minted}{\end{tcolorbox}}%

\begin{document}
\maketitle

\section{Modules}
\subsection{moduł Enumerable}
% https://ruby-doc.org/3.3.6/Enumerable.html#method-i-inject
Rdzenne klasy \mintinline{ruby}{ARGF}, \mintinline{ruby}{Array}, \mintinline{ruby}{Dir}, \mintinline{ruby}{Enumerator}, \mintinline{ruby}{ENV}, \mintinline{ruby}{Hash}, \mintinline{ruby}{IO}, \mintinline{ruby}{Range}, \mintinline{ruby}{Struct} oraz klasy biblioteki podstawowej \mintinline{ruby}{CSV}, \mintinline{ruby}{Set} rozszerzają \mintinline{ruby}{Enumerable}.

\mintinline{ruby}{all} (\mintinline{ruby}{any}) sprawdza, czy każdy (pewne) elementy są prawdziwe:
\begin{minted}{ruby}
# bez argumentu, bez bloku
(1..4).all? # => true
(1..4).any? # => true
%w[a b c d].all? # => true
%w[a b c d].any? # => true
[1, 2, nil].all?
['a', 'b', false].all?
[1, false, nil].any? # => true
[].all? # => true
[].any?

# z argumentem
(1..4).all?(Integer) # => true
(1..4).all?(Numeric) # => true
(1..4).all?(Float)
[nil, false, 0].any?(Integer) # => true
[nil, false, 0].any?(Numeric) # => true
[nil, false, 0].any?(Float)
%w[bar baz bat bam].all?(/ba/) # => true
%w[bar baz bat bam].all?(/bar/)
%w[bar baz bat bam].all?('ba')
%w[bar baz bat bam].any?(/m/) # => true
%w[bar baz bat bam].any?(/foo/)
%w[bar baz bat bam].any?('ba')
{foo: 0, bar: 1, baz: 2}.all?(Array) # => true
{foo: 0, bar: 1, baz: 2}.all?(Hash)
{foo: 0, bar: 1, baz: 2}.any?(Array) # => true
{foo: 0, bar: 1, baz: 2}.any?(Hash)
[].all?(Integer) # => true
[].any?(Integer)

# z blokiem
(1..4).all? {|element| element < 5 } # => true
(1..4).all? {|element| element < 4 }
(1..4).any? {|element| element < 2 } # => true
(1..4).any? {|element| element < 1 }
{foo: 0, bar: 1, baz: 2}.all? {|key, value| value < 3 } # => true
{foo: 0, bar: 1, baz: 2}.all? {|key, value| value < 2 }
{foo: 0, bar: 1, baz: 2}.any? {|key, value| value < 1 } # => true
{foo: 0, bar: 1, baz: 2}.any? {|key, value| value < 0 }
\end{minted}

none?: Returns true if no element meets a specified criterion; false otherwise.
\begin{minted}{ruby}
#none?
\end{minted}

one?: Returns true if exactly one element meets a specified criterion; false otherwise.
\begin{minted}{ruby}
#one?
\end{minted}

count: Returns the count of elements, based on an argument or block criterion, if given.
\begin{minted}{ruby}
#count
\end{minted}

tally: Returns a new Hash containing the counts of occurrences of each element.
Method ...
\begin{minted}{ruby}
#tally
\end{minted}

Method ...
\begin{minted}{ruby}
#chain
\end{minted}

Method ...
\begin{minted}{ruby}
#chunk
\end{minted}

Method ...
\begin{minted}{ruby}
#chunk_while
\end{minted}

Method ...
\begin{minted}{ruby}
#collect
\end{minted}

Method ...
\begin{minted}{ruby}
#collect_concat
\end{minted}

Method ...
\begin{minted}{ruby}
#compact
\end{minted}



Method ...
\begin{minted}{ruby}
#cycle
\end{minted}

Method ...
\begin{minted}{ruby}
#detect
\end{minted}

Method ...
\begin{minted}{ruby}
#drop
\end{minted}

Method ...
\begin{minted}{ruby}
#drop_while
\end{minted}

Method ...
\begin{minted}{ruby}
#each_cons
\end{minted}

Method ...
\begin{minted}{ruby}
#each_entry
\end{minted}

Method ...
\begin{minted}{ruby}
#each_slice
\end{minted}

Method ...
\begin{minted}{ruby}
#each_with_index
\end{minted}

Method ...
\begin{minted}{ruby}
#each_with_object
\end{minted}

Method ...
\begin{minted}{ruby}
#entries
\end{minted}

Method ...
\begin{minted}{ruby}
#filter
\end{minted}

Method ...
\begin{minted}{ruby}
#filter_map
\end{minted}

Method ...
\begin{minted}{ruby}
#find
\end{minted}

Method ...
\begin{minted}{ruby}
#find_all
\end{minted}

Method ...
\begin{minted}{ruby}
#find_index
\end{minted}

Method ...
\begin{minted}{ruby}
#first
\end{minted}

Method ...
\begin{minted}{ruby}
#flat_map
\end{minted}

Method ...
\begin{minted}{ruby}
#grep
\end{minted}

Method ...
\begin{minted}{ruby}
#grep_v
\end{minted}

Method ...
\begin{minted}{ruby}
#group_by
\end{minted}

Method ...
\begin{minted}{ruby}
#include?
\end{minted}


Method ...
\begin{minted}{ruby}
#lazy
\end{minted}

Method ...
\begin{minted}{ruby}
#map
\end{minted}

Method ...
\begin{minted}{ruby}
#max
\end{minted}

Method ...
\begin{minted}{ruby}
#max_by
\end{minted}

Method ...
\begin{minted}{ruby}
#member?
\end{minted}

Method ...
\begin{minted}{ruby}
#min
\end{minted}

Method ...
\begin{minted}{ruby}
#min_by
\end{minted}

Method ...
\begin{minted}{ruby}
#minmax
\end{minted}

Method ...
\begin{minted}{ruby}
#minmax_by
\end{minted}

Method ...
\begin{minted}{ruby}
#partition
\end{minted}


Method ...
\begin{minted}{ruby}
#reject
\end{minted}

Method ...
\begin{minted}{ruby}
#reverse_each
\end{minted}

Method ...
\begin{minted}{ruby}
#select
\end{minted}

Method ...
\begin{minted}{ruby}
#slice_after
\end{minted}

Method ...
\begin{minted}{ruby}
#slice_before
\end{minted}

Method ...
\begin{minted}{ruby}
#slice_when
\end{minted}

Method ...
\begin{minted}{ruby}
#sort
\end{minted}

Method ...
\begin{minted}{ruby}
#sort_by
\end{minted}

Method ...
\begin{minted}{ruby}
#sum
\end{minted}

Method ...
\begin{minted}{ruby}
#take
\end{minted}

Method ...
\begin{minted}{ruby}
#take_while
\end{minted}

Method ...
\begin{minted}{ruby}
#to_a
\end{minted}

Method ...
\begin{minted}{ruby}
#to_h
\end{minted}

Method ...
\begin{minted}{ruby}
#to_set
\end{minted}

Method ...
\begin{minted}{ruby}
#uniq
\end{minted}

Method ...
\begin{minted}{ruby}
#zip
\end{minted}

\textbf{inject} returns the object formed by combining all elements.
\textbf{reduce} is an alias.
\begin{minted}{ruby}
# with method-name argument
(1..4).inject(:+) # => 10
(1..4).inject(10, :+) # => 20

# with a block
(1..4).inject {|sum, n| sum + n*n } # => 30
(1..4).inject(2) {|sum, n| sum + n*n } # => 32

# with initial operand
(1..4).inject(2, :+) # => 12
(1..4).inject(2.0, :+) # => 12.0
('a'..'d').inject('foo', :+) # => "fooabcd"
%w[a b c].inject(['x'], :push) # => ["x", "a", "b", "c"]
(1..4).inject(Complex(2, 2), :+) # => (12+2i)
(1..4).inject do |memo, element|
  p "Memo: #{memo}; element: #{element}"
  memo + element
end # => 10
\end{minted}

\section{Classes}
\subsection{class Array}
% https://ruby-doc.org/3.3.6/Array.html#method-i-join

Methods for Converting
map, collect: Returns an array containing the block return-value for each element.

map!, collect!: Replaces each element with a block return-value.

flatten: Returns an array that is a recursive flattening of self.

flatten!: Replaces each nested array in self with the elements from that array.

inspect, to\_s: Returns a new String containing the elements.

\textbf{join} returns a new String containing the elements joined by the field separator.
\begin{minted}{ruby}
[:foo, 'bar', 2].join # => "foobar2"
[:foo, 'bar', 2].join(" ") # => "foo bar 2"
[:foo, [:bar, [:baz, :bat]]].join # => "foobarbazbat"
\end{minted}

to\_a: Returns self or a new array containing all elements.

to\_ary: Returns self.

to\_h: Returns a new hash formed from the elements.

transpose: Transposes self, which must be an array of arrays.

zip: Returns a new array of arrays containing self and given arrays; follow the link for details.
\subsection{klasa String}
% https://ruby-doc.org/3.3.6/String.html#method-i-split
Obiekt tej klasy jest ciągiem znaków, zazwyczaj przedstawiającym tekst.
Niektóre metody bez wykrzykników mutują argumenty, na przykład \mintinline{ruby}{String#replace}.
Klasa \mintinline{ruby}{String} zawiera moduł \mintinline{ruby}{Comparable}.

\subsubsection{Methods for Querying}
\textbf{length} (znane też jako \textbf{size}) liczy znaki, nie bajty (porównaj z klasą \mintinline{ruby}{Enumerable}).

empty?: Returns true if self.length is zero; false otherwise.

bytesize: Returns the count of bytes.

count: Returns the count of substrings matching given strings.

Substrings

% #=~: Returns the index of the first substring that matches a given Regexp or other object; returns nil if no match is found.

index: Returns the index of the first occurrence of a given substring; returns nil if none found.

rindex: Returns the index of the last occurrence of a given substring; returns nil if none found.

include?: Returns true if the string contains a given substring; false otherwise.

match: Returns a MatchData object if the string matches a given Regexp; nil otherwise.

match?: Returns true if the string matches a given Regexp; false otherwise.

\textbf{start\_with?} (\textbf{end\_with?}) zwraca, czy jeden łańcuch znaków zaczyna się (kończy się) pewnym innym łańcuchem.
\begin{minted}{ruby}
'hello'.start_with?('hell') # => true
'hello'.start_with?(/H/i) # => true
'hello'.start_with?('heaven', 'hell') # => true
'hello'.start_with?('heaven', 'paradise')
'hello'.end_with?('ello') # => true
'hello'.end_with?('heaven', 'ello') # => true
'hello'.end_with?('heaven', 'paradise')
\end{minted}

Encodings

encoding: Returns the Encoding object that represents the encoding of the string.

unicode\_normalized?: Returns true if the string is in Unicode normalized form; false otherwise.

valid\_encoding?: Returns true if the string contains only characters that are valid for its encoding.

ascii\_only?: Returns true if the string has only ASCII characters; false otherwise.

Other

sum: Returns a basic checksum for the string: the sum of each byte.

hash: Returns the integer hash code.

Methods for Comparing
==, ===: Returns true if a given other string has the same content as self.

eql?: Returns true if the content is the same as the given other string.

% #<=>: Returns -1, 0, or 1 as a given other string is smaller than, equal to, or larger than self.

casecmp: Ignoring case, returns -1, 0, or 1 as a given other string is smaller than, equal to, or larger than self.

casecmp?: Returns true if the string is equal to a given string after Unicode case folding; false otherwise.

Methods for Modifying a String
Each of these methods modifies self.

Insertion

insert: Returns self with a given string inserted at a given offset.

<<: Returns self concatenated with a given string or integer.

Substitution

sub!: Replaces the first substring that matches a given pattern with a given replacement string; returns self if any changes, nil otherwise.

gsub!: Replaces each substring that matches a given pattern with a given replacement string; returns self if any changes, nil otherwise.

succ!, next!: Returns self modified to become its own successor.

replace: Returns self with its entire content replaced by a given string.

reverse!: Returns self with its characters in reverse order.

setbyte: Sets the byte at a given integer offset to a given value; returns the argument.

tr!: Replaces specified characters in self with specified replacement characters; returns self if any changes, nil otherwise.

tr\_s!: Replaces specified characters in self with specified replacement characters, removing duplicates from the substrings that were modified; returns self if any changes, nil otherwise.


Encoding

encode!: Returns self with all characters transcoded from one given encoding into another.

unicode\_normalize!: Unicode-normalizes self; returns self.

scrub!: Replaces each invalid byte with a given character; returns self.

force\_encoding: Changes the encoding to a given encoding; returns self.

Deletion

clear: Removes all content, so that self is empty; returns self.

slice!, []=: Removes a substring determined by a given index, start/length, range, regexp, or substring.

squeeze!: Removes contiguous duplicate characters; returns self.

delete!: Removes characters as determined by the intersection of substring arguments.

lstrip!: Removes leading whitespace; returns self if any changes, nil otherwise.

rstrip!: Removes trailing whitespace; returns self if any changes, nil otherwise.

strip!: Removes leading and trailing whitespace; returns self if any changes, nil otherwise.

chomp!: Removes trailing record separator, if found; returns self if any changes, nil otherwise.

chop!: Removes trailing newline characters if found; otherwise removes the last character; returns self if any changes, nil otherwise.



\subsubsection{Methods for converting to new String}
Each of these methods returns a new String based on self, often just a modified copy of self.

Extension

*: Returns the concatenation of multiple copies of self,

+: Returns the concatenation of self and a given other string.

center: Returns a copy of self centered between pad substring.

concat: Returns the concatenation of self with given other strings.

prepend: Returns the concatenation of a given other string with self.

ljust: Returns a copy of self of a given length, right-padded with a given other string.

rjust: Returns a copy of self of a given length, left-padded with a given other string.

Encoding

b: Returns a copy of self with ASCII-8BIT encoding.

scrub: Returns a copy of self with each invalid byte replaced with a given character.

unicode\_normalize: Returns a copy of self with each character Unicode-normalized.


Substitution

dump: Returns a copy of self with all non-printing characters replaced by xHH notation and all special characters escaped.

undump: Returns a copy of self with all \\xNN notation replace by \\uNNNN notation and all escaped characters unescaped.

sub: Returns a copy of self with the first substring matching a given pattern replaced with a given replacement string;.

gsub: Returns a copy of self with each substring that matches a given pattern replaced with a given replacement string.

succ, next: Returns the string that is the successor to self.

\textbf{reverse} zwraca wszystkie znaki od tyłu (podobnie jak dla tablic).
Ma wersję z wykrzyknikiem.
\begin{minted}{ruby}
'stressed'.reverse # => "desserts"
\end{minted}

tr: Returns a copy of self with specified characters replaced with specified replacement characters.

tr\_s: Returns a copy of self with specified characters replaced with specified replacement characters, removing duplicates from the substrings that were modified.

%: Returns the string resulting from formatting a given object into self

\textbf{capitalize}, \textbf{downcase}, \textbf{upcase}, \textbf{swapcase} zmieniają wielkość liter.
Mają wersje z wykrzyknikiem.
\begin{minted}{ruby}
'hello World!'.capitalize # => "Hello world!"
'hello World!'.downcase   # => "hello world!"
'hello World!'.upcase     # => "HELLO WORLD!"
'hello World!'.swapcase   # => "HELLO wORLD!"
\end{minted}

Deletion

delete: Returns a copy of self with characters removed

delete\_prefix: Returns a copy of self with a given prefix removed.

delete\_suffix: Returns a copy of self with a given suffix removed.

lstrip: Returns a copy of self with leading whitespace removed.

rstrip: Returns a copy of self with trailing whitespace removed.

strip: Returns a copy of self with leading and trailing whitespace removed.

chomp: Returns a copy of self with a trailing record separator removed, if found.

chop: Returns a copy of self with trailing newline characters or the last character removed.

squeeze: Returns a copy of self with contiguous duplicate characters removed.

[], slice: Returns a substring determined by a given index, start/length, or range, or string.

byteslice: Returns a substring determined by a given index, start/length, or range.

chr: Returns the first character.

Duplication

to\_s, to\_str: If self is a subclass of String, returns self copied into a String; otherwise, returns self.




\subsubsection{Methods for converting to non-String}
Each of these methods converts the contents of self to a non-String.

Characters, Bytes, and Clusters

bytes: Returns an array of the bytes in self.

chars: Returns an array of the characters in self.

codepoints: Returns an array of the integer ordinals in self.

getbyte: Returns an integer byte as determined by a given index.

% grapheme_clusters: Returns an array of the grapheme clusters in self.

lines: Returns an array of the lines in self, as determined by a given record separator.

partition: Returns a 3-element array determined by the first substring that matches a given substring or regexp,

rpartition: Returns a 3-element array determined by the last substring that matches a given substring or regexp,

\textbf{split} returns an array of substrings determined by a given delimiter – regexp or string.
\begin{minted}{ruby}
'abc def ghi'.split(' ') # => ["abc", "def", "ghi"]
"abc \n\tdef\t\n  ghi".split(' ') # => ["abc", "def", "ghi"]
'abc  def   ghi'.split(' ') # => ["abc", "def", "ghi"]
''.split(' ') # => []
'abracadabra'.split('ab') # => ["", "racad", "ra"]
'aaabcdaaa'.split('a') # => ["", "", "", "bcd"]
''.split('a') # => []
'3.14159'.split('1') # => ["3.", "4", "59"]
'!@#$%^$&*($)_+'.split('$') # => ["!@#", "%^", "&*(", ")_+"]

# regexp delimeter
'abracadabra'.split(/ab/) # => ["", "racad", "ra"]
'aaabcdaaa'.split(/a/) # => ["", "", "", "bcd"]
'aaabcdaaa'.split(//) # => ["a", "a", "a", "b", "c", "d", "a", "a", "a"]
'1 + 1 == 2'.split(/\W+/) # => ["1", "1", "2"]

# with a limit - a positive integer
'aaabcdaaa'.split('a', 1) # => ["aaabcdaaa"]
'aaabcdaaa'.split('a', 2) # => ["", "aabcdaaa"]
'aaabcdaaa'.split('a', 5) # => ["", "", "", "bcd", "aa"]
'aaabcdaaa'.split('a', 7) # => ["", "", "", "bcd", "", "", ""]
'aaabcdaaa'.split('a', 8) # => ["", "", "", "bcd", "", "", ""]

# with a block
'abc def ghi'.split(' ') {|substring| p substring }
\end{minted}

Matching

scan: Returns an array of substrings matching a given regexp or string, or, if a block given, passes each matching substring to the block.

unpack: Returns an array of substrings extracted from self according to a given format.

unpack1: Returns the first substring extracted from self according to a given format.

Numerics

hex: Returns the integer value of the leading characters, interpreted as hexadecimal digits.

oct: Returns the integer value of the leading characters, interpreted as octal digits.

ord: Returns the integer ordinal of the first character in self.

% to_i: Returns the integer value of leading characters, interpreted as an integer.

% to_f: Returns the floating-point value of leading characters, interpreted as a floating-point number.

Strings and Symbols

inspect: Returns copy of self, enclosed in double-quotes, with special characters escaped.

% to_sym, intern: Returns the symbol corresponding to self.




Methods for Iterating
each\_byte: Calls the given block with each successive byte in self.

each\_char: Calls the given block with each successive character in self.

each\_codepoint: Calls the given block with each successive integer codepoint in self.

each\_grapheme\_cluster: Calls the given block with each successive grapheme cluster in self.

each\_line: Calls the given block with each successive line in self, as determined by a given record separator.

upto: Calls the given block with each string value returned by successive calls to succ.





\newpage

If you check the list of methods on our String above you see that in Ruby we can have methods that end with a question mark ?. What’s up with that?
By convention, in Ruby, these methods return either true or false. For example, we can ask a number if it is even or odd:
These methods are called predicate methods in Ruby. Not quite sure why, maybe because of the historical math context of programming.
Bang methods end with an exlamation mark, and often modify the object they are called on. 


\section{Data types}
\subsection{Numeric types}
There are two main types of numbers: integers and floats.
Basic arithmetical operations are denoted with \mintinline{ruby}{+}, \mintinline{ruby}{-}, \mintinline{ruby}{*}, \mintinline{ruby}{/} (division) and \mintinline{ruby}{**} (exponentiation).
Modular exponentiation of integers (but not floats!) is available as \mintinline{ruby}{.pow}, here modulus is optional.
Other common method is \mintinline{ruby}{.abs} (absolute value).

\textbf{even?} and \textbf{odd?} return true if self is an even (an odd) number, false otherwise.

%

\subsection{Strings}
Strings can be formed with either double or single quotes.
Single quotes disable string interpolation and escape characters:
\begin{minted}{ruby}
what = "le Monde"
puts "Bonjour, #{what}" #=> "Bonjour, le Monde"
puts 'Bonjour, #{what}' #=> "Bonjour, #{name}"
\end{minted}

\textbf{count} returns the total number of characters in self that are specified by the given selectors (compare with count under Arrays):
\begin{minted}{ruby}
a = "hello world"
a.count "lo"                   #=> 5
a.count "lo", "o"              #=> 2
a.count "hello", "^l"          #=> 4
a.count "ej-m"                 #=> 4
"hello^world".count "\\^aeiou" #=> 4
"hello-world".count "a\\-eo"   #=> 4
c = "hello world\\r\\n"
c.count "\\"                   #=> 2
c.count "\\A"                  #=> 0
c.count "X-\\w"                #=> 3
\end{minted}

% String methods:
% \begin{minted}{ruby}
% "hello".capitalize #=> "Hello"
% "hello".include?("lo")  #=> true
% "hello".include?("z")   #=> false
% "hello".upcase  #=> "HELLO"
% "Hello".downcase  #=> "hello"
% "hello".empty?  #=> false
% "".empty?       #=> true
% "hello".length  #=> 5
% "hello".reverse  #=> "olleh"
% "hello world".split  #=> ["hello", "world"]
% "hello".split("")    #=> ["h", "e", "l", "l", "o"]
% " hello, world   ".strip  #=> "hello, world"
% \end{minted}


% \begin{minted}{ruby}
% "Hello " + "world!"      # concatenation
% "Hello " << "world!"     # shovel operator
% "Hello ".concat("world!)
% \end{minted}

% Substrings:
% \begin{minted}{ruby}
% "bonjour"[0]    # "b"
% "bonjour"[0..1] # "bo"
% "bonjour"[0, 4] # "bonj"
% "bonjour"[-1]   # "r"
% \end{minted}

% Interpolation:

\textbf{delete} returns a copy of self with characters specified by selectors removed.
% "hello".delete "l","lo"        #=> "heo"
% "hello".delete "lo"            #=> "he"
% "hello".delete "aeiou", "^e"   #=> "hell"
% "hello".delete "ej-m"          #=> "ho"

\textbf{each\_char} calls the given block with each successive character:

\begin{minted}{ruby}
'hello'.each_char {|char| print char, ' ' }
\end{minted}

\textbf{downcase} and \textbf{upcase} return a lowercase and uppercase of characters in string:
\begin{minted}{ruby}
'HeLlO wOrLd!'.downcase # => "hello world!"
'HeLlO wOrLd!'.upcase   # => "HELLO WORLD!"
\end{minted}

\textbf{chars} returns an array of the characters in self:
\begin{minted}{ruby}
'hello'.chars # => ["h", "e", "l", "l", "o"]
\end{minted}

% str.end_with?(ending)


%
\subsection{Booleans}
Booleans: true and false.
Nil represents ''nothing'' and is needed because every method in Ruby always returns exactly one object.

\subsection{Symbols}
Symbols are unique identifiers that are considered code, not data. 
\begin{minted}{ruby}
:symbol
\end{minted}
% "string" == "string"  #=> true
% "string".object_id == "string".object_id  #=> false
% :symbol.object_id == :symbol.object_id    #=> true



\subsection{Arrays}
Arrays are zero-indexed, can contain all kinds of objects and have a defined order.
% \begin{minted}{ruby}
% words = ["one", "two", "three"]
% words << "four"
% puts words[10] # retrieving an element that does not exist gives nil
% [1, 2] + [5, 6] # [1, 2, 5, 6]
% ["A", "Be"] * 2 # ["A", "Be", "A", "Be"]
% & # intersection
% # vvv methods vvv
% .first .last .length .sort .compact .index(...) .rotate(...), .transpose
% \end{minted}

\textbf{count} returns a count of specified elements.
With no argument and no block, returns the count of all elements.
With argument \mintinline{ruby}{obj}, returns the count of elements \mintinline{ruby}{==} to \mintinline{ruby}{obj}.
With no argument and a block given, calls the block with each element; returns the count of elements for which the block returns a truthy value:
\begin{minted}{ruby}
[0, 1, 2].count # => 3
[0, 1, 2, 0.0].count(0) # => 2
[0, 1, 2, 3].count {|element| element > 1} # => 2
\end{minted}

\textbf{sum}.
When a block is given, it is called with each element and the block’s return value (instead of the element itself) is used as the addend.
When no block is given, returns the object equivalent to:
\begin{minted}{ruby}
sum = init
array.each {|element| sum += element }
\end{minted}
Note: \mintinline{ruby}{join} (\mintinline{ruby}{flatten}) may be faster for an array of strings (of arrays).

\textbf{uniq} returns a new Array containing those elements from self that are not duplicates, the first occurrence always being retained.
With no block given, identifies and omits duplicates using method \mintinline{ruby}{eql?} to compare:
With a block given, calls the block for each element; identifies (using method eql?) and omits duplicate values, that is, those elements for which the block returns the same value:
\begin{minted}{ruby}
['a', 'aa', 'aaa', 'b', 'bb', 'bbb'].uniq {|element| element.size } # => ["a", "aa", "aaa"]
[0, 0, 1, 1, 2, 2].uniq # => [0, 1, 2]
\end{minted}

% return true if [2,3,5,7,11,17].include? num

\textbf{Hashes} are associative key-value dictionaries with efficient access by key.

% \begin{minted}{ruby}
%     dictionary = { "one" => "eins", "two" => "zwei", "three" => "drei" }
%     dictionary["zero"] = "null"
%     puts dictionary["one"]
%     > { "one" => "eins" }.merge({ "two" => "zwei" })
%     => { "one" => "eins", "two" => "zwei" }
%     > dictionary = { "one" => "eins" }
%     > dictionary.fetch("one")
%     => "eins"
%     > dictionary.fetch("two")
%     KeyError: key not found: "two"
%     [] gives nil
%     .length
%     .size
%     { one: "eins", two: "zwei", three: "drei" } # new syntax
%     { :one => "eins", :two => "zwei", :three => "drei" } # old
%     \end{minted}

\section{Require ...}
Sets are syntactic sugar on top of Hash provided by the standard library:
\begin{minted}{ruby}
require 'set'
set = Set.new([1, 2, 1])
set.size # => 2
\end{minted}

% DOES NOT REQUIRE! Math.sqrt(16)

\section{Control flow}
Ruby has if, unless, ternary operators and case (which uses \mintinline{ruby}{#===}).

\begin{multicols}{3}
\begin{minted}{ruby}
if x >= 1
  "Very big"
elsif x >= 0
  "Big"
else
  "Small"
end
\end{minted}
\columnbreak

\begin{minted}{ruby}
unless failures == 0
  raise "Failures detected"
end
\end{minted}
\columnbreak

\begin{minted}{ruby}
case object
when Integer
  :int
when String
  :string
else
  :no_idea
end
\end{minted}
\end{multicols}

\subsection{Loops}
Ruby has for loops (but you should almost never use them), while/until loops and an infinite \mintinline{ruby}{loop} loop (which is rarely used).
\mintinline{ruby}{break} and \mintinline{ruby}{next} have usual meaning.
\mintinline{ruby}{redo} goes to the start of current iteration and does it again.

\begin{multicols}{3}
\begin{minted}{ruby}
while applies?
  #...
  break
end

do_something while applies?
\end{minted}
\columnbreak

\begin{minted}{ruby}
until applies?
  do_something
  break
end

do_something until applies?
\end{minted}
\columnbreak

\begin{minted}{ruby}
loop do
  do_something
  break
end
\end{minted}
\end{multicols}

(Nie wiem, jak to się nazywa)
\begin{minted}{ruby}
[1, 2, 3].each do |num|
  do_something
end

[1, 2, 3].each.with_index do |num, idx|
  do_something
end
\end{minted}
Instead of giving a definition of what iterator is, we present several examples:
\begin{itemize}
  \item \mintinline{ruby}{5.times { puts "Hi!" }}
  \item \mintinline{ruby}{1.upto(5) { puts "Hi!" }}
  \item \mintinline{ruby}{5.down­to(1) { puts "Hi!" }}
  \item \mintinline{ruby}{(1..5).each { puts "Hi!" }}
  \item \mintinline{ruby}{xs.each { |x| puts 1+x }}
\end{itemize}  



\section{To be done later}
\begin{minted}{ruby}
[*1..n].inject(:*)
(1..n).reduce(1,:*)


return false if num <= 1 || num % 2 == 0 || num % 3 == 0 
str1.chars.last(str2.size) == str2.chars
str.reverse!
size_compare = ending.size
str.split("").last(ending.length).join("") == ending
s.chars.each_with_index.map{ |c, i| c.upcase + c.downcase * i }.join('-')
arr.select{|x| x > 0}.reduce(0, :+)

number.even? ? "Even" : "Odd"
str.delete('aeiouAEIOU')

s.chars.find {|i| s.downcase.count(i)==1 || s.upcase.count(i)==1} || ""

numbers.split.map(&:to_i).minmax.reverse.join(' ')

(start...finish).step(2).reduce(:+)


require 'prime'
def primeFactors(n)
  n.prime_division.map {|factor, times| times == 1 ? "(#{factor})" : "(#{factor}**#{times})"}.join
end


def consecutive_fibonacci_numbers
  Enumerator.new do |yielder|
    a, b = 1, 1
    
    loop do
      yielder.yield([a, b])
      a, b = b, a + b
    end
  end
end

print vs puts, gets, gets.chomp

if, elsif, else
if vs unless
% || && ! 

5 == 5 true
5.eql?(5.0) false (diff types)
There is also .equal?

spaceship operator <=>

case (vs if)

ternary operator

3.times do ...


begin rescue end


  numbers = numbers.split.map{|word| word.to_i}
  "#{numbers.max} #{numbers.min}"
  # minmax
num.digits.map { |d| d*d } .reverse.join.to_i
arr.select{|x| x > 0}.reduce(0, :+)
l.select{|word| word.is_a? Numeric}
l.reject { |x| x.is_a? String }
l.grep(Numeric) # I knew. :D http://ruby-doc.org/core-2.3.1/Enumerable.html#method-i-grep
s[(s.size-1)/2..s.size/2] # same as length ?
s.chars.map.with_index { |s, i| (s * (i+1)).capitalize}.join("-")


\section{Built-in types}
We start with common built-in data types.

Comparison operators are \mintinline{ruby}{<}, \mintinline{ruby}{<=}, \mintinline{ruby}{<=>}, \mintinline{ruby}{==} (aliased as \mintinline{ruby}{===}), \mintinline{ruby}{>}, \mintinline{ruby}{>=}.

\mintinline{ruby}{.ceil} and \mintinline{ruby}{.floor} are functions known in math as $\lfloor x \rfloor$ and $\lceil x \rceil$ respectively.
There are \mintinline{ruby}{.round} and  \mintinline{ruby}{.truncate} as well.
Both classes support \mintinline{ruby}{.divmod} (quotient and reminder), \mintinline{ruby}{.fdiv} (float result of division) and \mintinline{ruby}{%} (modulo).
Only integers support \mintinline{ruby}{.div} and \mintinline{ruby}{remainder}.

\mintinline{ruby}{.digits} is an array of (base-10, by default) digits of an Integer.
\mintinline{ruby}{.odd?} and \mintinline{ruby}{.even?} are self explanatory.
\mintinline{ruby}{.gcd} stands for greatest common divisor, \mintinline{ruby}{.lcm} for lowest common multiple (and there is \mintinline{ruby}{.gcdlcm} too).
They are all Integer specific.

\mintinline{ruby}{&}, \mintinline{ruby}{|}, \mintinline{ruby}{^} are bitwise AND, OR, XOR;
\mintinline{ruby}{<<}, \mintinline{ruby}{>>} denote bit-shifts.

\mintinline{ruby}{.size} gives the number of bytes in the machine representation, depends on the system and makes no sense for floats.

It's possible to convert between types with \mintinline{ruby}{.to_i}, \mintinline{ruby}{.to_f}, \mintinline{ruby}{.to_s} (to string).

\subsection{Numeric types -- rationals}
To convert float, use \mintinline{ruby}{.to_r} method

Assignment operators = and shorthand +=, -= *= /=.

Variable names should always be lowercase, and multiple words that make up a variable name should be split by an underscore. This is known as snake case.

https://www.theodinproject.com/lessons/ruby-variables
http://ruby-for-beginners.rubymonstas.org/writing_classes/attribute_readers.html
https://launchschool.com/books/ruby/read/basics
\end{minted}

\end{document}




\end{minted}
\end{document}


\documentclass{a5charun}
\title{Ruby 3.3.0 notes, version 0.1.0}
\author{Leon Suwalski}

\setlength{\multicolsep}{6.0pt plus 1.0pt minus 1.0pt}% 50% of original values

\usepackage{xcolor}
% copied from https://www.color-hex.com/color-palette/85828
\definecolor{rubycolor}{HTML}{ffecec}
\definecolor{rubycolor2}{HTML}{990000}

% copied from https://tex.stackexchange.com/questions/132849/how-can-i-change-the-font-size-of-the-number-in-minted-environment
\BeforeBeginEnvironment{minted}{\begin{tcolorbox}[breakable,boxsep=5pt,left=0pt,right=0pt,top=0pt,bottom=0pt,boxrule=0.5pt,arc=0pt,outer arc=0pt,colback=rubycolor,colframe=rubycolor2]\small}%
\AfterEndEnvironment{minted}{\end{tcolorbox}}%

\begin{document}
\maketitle

If you check the list of methods on our String above you see that in Ruby we can have methods that end with a question mark ?. What’s up with that?
By convention, in Ruby, these methods return either true or false. For example, we can ask a number if it is even or odd:
These methods are called predicate methods in Ruby. Not quite sure why, maybe because of the historical math context of programming.
Bang methods end with an exlamation mark, and often modify the object they are called on. 


\section{Data types}
\subsection{Numeric types}
There are two main types of numbers: integers and floats.
Basic arithmetical operations are denoted with \mintinline{ruby}{+}, \mintinline{ruby}{-}, \mintinline{ruby}{*}, \mintinline{ruby}{/} (division) and \mintinline{ruby}{**} (exponentiation).
Modular exponentiation of integers (but not floats!) is available as \mintinline{ruby}{.pow}, here modulus is optional.
Other common method is \mintinline{ruby}{.abs} (absolute value).

\textbf{even?} and \textbf{odd?} return true if self is an even (an odd) number, false otherwise.

%

\subsection{Strings}
Strings can be formed with either double or single quotes.
Single quotes disable string interpolation and escape characters:
\begin{minted}{ruby}
what = "le Monde"
puts "Bonjour, #{what}" #=> "Bonjour, le Monde"
puts 'Bonjour, #{what}' #=> "Bonjour, #{name}"
\end{minted}

\textbf{count} returns the total number of characters in self that are specified by the given selectors (compare with count under Arrays):
\begin{minted}{ruby}
a = "hello world"
a.count "lo"                   #=> 5
a.count "lo", "o"              #=> 2
a.count "hello", "^l"          #=> 4
a.count "ej-m"                 #=> 4
"hello^world".count "\\^aeiou" #=> 4
"hello-world".count "a\\-eo"   #=> 4
c = "hello world\\r\\n"
c.count "\\"                   #=> 2
c.count "\\A"                  #=> 0
c.count "X-\\w"                #=> 3
\end{minted}

% String methods:
% \begin{minted}{ruby}
% "hello".capitalize #=> "Hello"
% "hello".include?("lo")  #=> true
% "hello".include?("z")   #=> false
% "hello".upcase  #=> "HELLO"
% "Hello".downcase  #=> "hello"
% "hello".empty?  #=> false
% "".empty?       #=> true
% "hello".length  #=> 5
% "hello".reverse  #=> "olleh"
% "hello world".split  #=> ["hello", "world"]
% "hello".split("")    #=> ["h", "e", "l", "l", "o"]
% " hello, world   ".strip  #=> "hello, world"
% \end{minted}


% \begin{minted}{ruby}
% "Hello " + "world!"      # concatenation
% "Hello " << "world!"     # shovel operator
% "Hello ".concat("world!)
% \end{minted}

% Substrings:
% \begin{minted}{ruby}
% "bonjour"[0]    # "b"
% "bonjour"[0..1] # "bo"
% "bonjour"[0, 4] # "bonj"
% "bonjour"[-1]   # "r"
% \end{minted}

% Interpolation:

\textbf{delete} returns a copy of self with characters specified by selectors removed.
% "hello".delete "l","lo"        #=> "heo"
% "hello".delete "lo"            #=> "he"
% "hello".delete "aeiou", "^e"   #=> "hell"
% "hello".delete "ej-m"          #=> "ho"

\textbf{each\_char} calls the given block with each successive character:

\begin{minted}{ruby}
'hello'.each_char {|char| print char, ' ' }
\end{minted}

\textbf{downcase} and \textbf{upcase} return a lowercase and uppercase of characters in string:
\begin{minted}{ruby}
'HeLlO wOrLd!'.downcase # => "hello world!"
'HeLlO wOrLd!'.upcase   # => "HELLO WORLD!"
\end{minted}

\textbf{chars} returns an array of the characters in self:
\begin{minted}{ruby}
'hello'.chars # => ["h", "e", "l", "l", "o"]
\end{minted}

% str.end_with?(ending)


%
\subsection{Booleans}
Booleans: true and false.
Nil represents ''nothing'' and is needed because every method in Ruby always returns exactly one object.

\subsection{Symbols}
Symbols are unique identifiers that are considered code, not data. 
\begin{minted}{ruby}
:symbol
\end{minted}
% "string" == "string"  #=> true
% "string".object_id == "string".object_id  #=> false
% :symbol.object_id == :symbol.object_id    #=> true



\subsection{Arrays}
Arrays are zero-indexed, can contain all kinds of objects and have a defined order.
% \begin{minted}{ruby}
% words = ["one", "two", "three"]
% words << "four"
% puts words[10] # retrieving an element that does not exist gives nil
% [1, 2] + [5, 6] # [1, 2, 5, 6]
% ["A", "Be"] * 2 # ["A", "Be", "A", "Be"]
% & # intersection
% # vvv methods vvv
% .first .last .length .sort .compact .index(...) .rotate(...), .transpose
% \end{minted}

\textbf{count} returns a count of specified elements.
With no argument and no block, returns the count of all elements.
With argument \mintinline{ruby}{obj}, returns the count of elements \mintinline{ruby}{==} to \mintinline{ruby}{obj}.
With no argument and a block given, calls the block with each element; returns the count of elements for which the block returns a truthy value:
\begin{minted}{ruby}
[0, 1, 2].count # => 3
[0, 1, 2, 0.0].count(0) # => 2
[0, 1, 2, 3].count {|element| element > 1} # => 2
\end{minted}

\textbf{sum}.
When a block is given, it is called with each element and the block’s return value (instead of the element itself) is used as the addend.
When no block is given, returns the object equivalent to:
\begin{minted}{ruby}
sum = init
array.each {|element| sum += element }
\end{minted}
Note: \mintinline{ruby}{join} (\mintinline{ruby}{flatten}) may be faster for an array of strings (of arrays).

\textbf{uniq} returns a new Array containing those elements from self that are not duplicates, the first occurrence always being retained.
With no block given, identifies and omits duplicates using method \mintinline{ruby}{eql?} to compare:
With a block given, calls the block for each element; identifies (using method eql?) and omits duplicate values, that is, those elements for which the block returns the same value:
\begin{minted}{ruby}
['a', 'aa', 'aaa', 'b', 'bb', 'bbb'].uniq {|element| element.size } # => ["a", "aa", "aaa"]
[0, 0, 1, 1, 2, 2].uniq # => [0, 1, 2]
\end{minted}

% return true if [2,3,5,7,11,17].include? num

\textbf{Hashes} are associative key-value dictionaries with efficient access by key.

% \begin{minted}{ruby}
%     dictionary = { "one" => "eins", "two" => "zwei", "three" => "drei" }
%     dictionary["zero"] = "null"
%     puts dictionary["one"]
%     > { "one" => "eins" }.merge({ "two" => "zwei" })
%     => { "one" => "eins", "two" => "zwei" }
%     > dictionary = { "one" => "eins" }
%     > dictionary.fetch("one")
%     => "eins"
%     > dictionary.fetch("two")
%     KeyError: key not found: "two"
%     [] gives nil
%     .length
%     .size
%     { one: "eins", two: "zwei", three: "drei" } # new syntax
%     { :one => "eins", :two => "zwei", :three => "drei" } # old
%     \end{minted}

\section{Require ...}
Sets are syntactic sugar on top of Hash provided by the standard library:
\begin{minted}{ruby}
require 'set'
set = Set.new([1, 2, 1])
set.size # => 2
\end{minted}

% DOES NOT REQUIRE! Math.sqrt(16)


\section{Control flow}
Ruby has if, unless, ternary operators and case (which uses \mintinline{ruby}{#===}).

\begin{multicols}{3}
\begin{minted}{ruby}
if x >= 1
  "Very big"
elsif x >= 0
  "Big"
else
  "Small"
end
\end{minted}
\columnbreak

\begin{minted}{ruby}
unless failures == 0
  raise "Failures detected"
end
\end{minted}
\columnbreak

\begin{minted}{ruby}
case object
when Integer
  :int
when String
  :string
else
  :no_idea
end
\end{minted}
\end{multicols}

\subsection{Loops}
Ruby has for loops (but you should almost never use them), while/until loops and an infinite \mintinline{ruby}{loop} loop (which is rarely used).
\mintinline{ruby}{break} and \mintinline{ruby}{next} have usual meaning.
\mintinline{ruby}{redo} goes to the start of current iteration and does it again.

\begin{multicols}{3}
\begin{minted}{ruby}
while applies?
  #...
  break
end

do_something while applies?
\end{minted}
\columnbreak

\begin{minted}{ruby}
until applies?
  do_something
  break
end

do_something until applies?
\end{minted}
\columnbreak

\begin{minted}{ruby}
loop do
  do_something
  break
end
\end{minted}
\end{multicols}

(Nie wiem, jak to się nazywa)
\begin{minted}{ruby}
[1, 2, 3].each do |num|
  do_something
end

[1, 2, 3].each.with_index do |num, idx|
  do_something
end
\end{minted}
Instead of giving a definition of what iterator is, we present several examples:
\begin{itemize}
  \item \mintinline{ruby}{5.times { puts "Hi!" }}
  \item \mintinline{ruby}{1.upto(5) { puts "Hi!" }}
  \item \mintinline{ruby}{5.down­to(1) { puts "Hi!" }}
  \item \mintinline{ruby}{(1..5).each { puts "Hi!" }}
  \item \mintinline{ruby}{xs.each { |x| puts 1+x }}
\end{itemize}  

\end{document}

return false if num <= 1 || num % 2 == 0 || num % 3 == 0 
str1.chars.last(str2.size) == str2.chars
str.reverse!
size_compare = ending.size
str.split("").last(ending.length).join("") == ending
s.chars.each_with_index.map{ |c, i| c.upcase + c.downcase * i }.join('-')
arr.select{|x| x > 0}.reduce(0, :+)

number.even? ? "Even" : "Odd"
str.delete('aeiouAEIOU')

s.chars.find {|i| s.downcase.count(i)==1 || s.upcase.count(i)==1} || ""

numbers.split.map(&:to_i).minmax.reverse.join(' ')

(start...finish).step(2).reduce(:+)


require 'prime'
def primeFactors(n)
  n.prime_division.map {|factor, times| times == 1 ? "(#{factor})" : "(#{factor}**#{times})"}.join
end


def consecutive_fibonacci_numbers
  Enumerator.new do |yielder|
    a, b = 1, 1
    
    loop do
      yielder.yield([a, b])
      a, b = b, a + b
    end
  end
end

\section{Built-in types}
We start with common built-in data types.

Comparison operators are \mintinline{ruby}{<}, \mintinline{ruby}{<=}, \mintinline{ruby}{<=>}, \mintinline{ruby}{==} (aliased as \mintinline{ruby}{===}), \mintinline{ruby}{>}, \mintinline{ruby}{>=}.

\mintinline{ruby}{.ceil} and \mintinline{ruby}{.floor} are functions known in math as $\lfloor x \rfloor$ and $\lceil x \rceil$ respectively.
There are \mintinline{ruby}{.round} and  \mintinline{ruby}{.truncate} as well.
Both classes support \mintinline{ruby}{.divmod} (quotient and reminder), \mintinline{ruby}{.fdiv} (float result of division) and \mintinline{ruby}{%} (modulo).
Only integers support \mintinline{ruby}{.div} and \mintinline{ruby}{remainder}.

\mintinline{ruby}{.digits} is an array of (base-10, by default) digits of an Integer.
\mintinline{ruby}{.odd?} and \mintinline{ruby}{.even?} are self explanatory.
\mintinline{ruby}{.gcd} stands for greatest common divisor, \mintinline{ruby}{.lcm} for lowest common multiple (and there is \mintinline{ruby}{.gcdlcm} too).
They are all Integer specific.

\mintinline{ruby}{&}, \mintinline{ruby}{|}, \mintinline{ruby}{^} are bitwise AND, OR, XOR;
\mintinline{ruby}{<<}, \mintinline{ruby}{>>} denote bit-shifts.

\mintinline{ruby}{.size} gives the number of bytes in the machine representation, depends on the system and makes no sense for floats.

It's possible to convert between types with \mintinline{ruby}{.to_i}, \mintinline{ruby}{.to_f}, \mintinline{ruby}{.to_s} (to string).

\subsection{Numeric types -- rationals}
To convert float, use \mintinline{ruby}{.to_r} method

Assignment operators = and shorthand +=, -= *= /=.

Variable names should always be lowercase, and multiple words that make up a variable name should be split by an underscore. This is known as snake case.

print vs puts, gets, gets.chomp

if, elsif, else
if vs unless
% || && ! 

5 == 5 true
5.eql?(5.0) false (diff types)
There is also .equal?

spaceship operator <=>

case (vs if)

ternary operator

3.times do ...


begin rescue end


  numbers = numbers.split.map{|word| word.to_i}
  "#{numbers.max} #{numbers.min}"
  # minmax
num.digits.map { |d| d*d } .reverse.join.to_i
arr.select{|x| x > 0}.reduce(0, :+)
l.select{|word| word.is_a? Numeric}
l.reject { |x| x.is_a? String }
l.grep(Numeric) # I knew. :D http://ruby-doc.org/core-2.3.1/Enumerable.html#method-i-grep
s[(s.size-1)/2..s.size/2] # same as length ?
s.chars.map.with_index { |s, i| (s * (i+1)).capitalize}.join("-")

\end{minted}
\end{document}

https://www.theodinproject.com/lessons/ruby-variables
http://ruby-for-beginners.rubymonstas.org/writing_classes/attribute_readers.html
https://launchschool.com/books/ruby/read/basics
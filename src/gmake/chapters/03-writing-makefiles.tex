\section{Writing Makefiles}
Normally you should call your makefile either makefile or Makefile.
If you want to use a nonstandard name for your makefile, you can specify the makefile name with the \texttt{-f} or \texttt{-{}-file} option. 

The include directive tells make to suspend reading the current makefile and read one or more other makefiles before continuing.

\begin{minted}{make}
include foo *.baz $(bar)
\end{minted}

\color{gray}
\subsection{NEEDS TO BE DONE: The Variable MAKEFILES}
Lorem ipsum...
\color{black}

\color{gray}
\subsection{NEEDS TO BE DONE: How Makefiles Are Remade}
Lorem ipsum...
\color{black}

\color{gray}
\subsection{NEEDS TO BE DONE: Overriding Part of Another Makefile}
Lorem ipsum...
\color{black}

\color{gray}
\subsection{NEEDS TO BE DONE: How make Reads a Makefile}
Lorem ipsum...
\color{black}

\color{gray}
\subsection{NEEDS TO BE DONE: How Makefiles Are Parsed}
Lorem ipsum...
\color{black}

\color{gray}
\subsection{NEEDS TO BE DONE: Secondary Expansion}
Lorem ipsum...
\color{black}
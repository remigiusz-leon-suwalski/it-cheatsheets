\section{Run a command in a new container}
\important{Start a container} (process with its own file system, its own networking, and its own isolated process tree):
\begin{bashcode}
$ docker run busybox
\end{bashcode}

(Docker generates names for containers from notable scientists and hackers.)
Some of the useful switches:
\begin{compactitem}
    \item [d] \textbf{\mintinline{bash}{--detach}} runs in background,
    \item [i] \textbf{\mintinline{bash}{--interactive}}: keeps standard input open, even if not attached,
    \item \textbf{\mintinline{bash}{--name=...}} assigns a custom name,
    \item [p] \textbf{\mintinline{bash}{--publish hostPort:containerPort}} publishes a single port (or range of ports) to the host,
    \item [P] \textbf{\mintinline{bash}{--publish-all}} publishes exposed ports to random ports on the host interfaces,
    \item \textbf{\mintinline{bash}{--rm}} automatically removes the container when it exits,
    \item [t] \textbf{\mintinline{bash}{--tty}} allocate a pseudo-TTY,
    \item [v] \textbf{\mintinline{bash}{--volume hostDir:containerDir}} creates a bind mount.
\end{compactitem}

Example.
Applications can be made more secure by running them in read-only mode.
Read only containers may still need to write temporary data:
\begin{bashcode}
$ docker run --read-only --tmpfs /run --tmpfs /tmp \
>     -it fedora /bin/bash
\end{bashcode}

Example.
If you want messages that are logged in your container to show up in the host's syslog/journal:
\begin{bashcode}
$ docker run -v /dev/log:/dev/log -it fedora /bin/bash
(bash) # logger "Hello from my container"; exit

$ journalctl -b | grep Hellp
\end{bashcode}

%
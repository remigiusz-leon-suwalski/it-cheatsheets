%

\section{Create and run new containers}
\textbf{docker run} runs a process in a new container, with its own file system, its own networking, and its own isolated process tree.
If the IMAGE is not already loaded, it will be pulled with dependencies first.
\begin{bashcode}
$ docker run --rm hello-world
# Auto-remove the container when it exits.

$ docker run --name 'my-busy-box' busybox:1.36
# Assigns a name to the container. When not present, Docker
# generates a name from notable scientists and hackers like
# 'quirky_heyrovsky' or 'condescending_hofstadter'.

$ docker run -it ubuntu /bin/bash
# -i keeps stdin open (same as --interactive)
# -t allocates a pseudo-TTY (same as --tty)

$ docker run -p 8888:5000 my_user/my_image
# -p publishes a port/range of ports to the host,
# -P exposes ports to random ports on the host interfaces.
# -p = --publish, -P = --publish-all
\end{bashcode}

{\color{red}

Useful short options:
\begin{compactitem}
    \item [\texttt{-d}] runs in background (same as \mintinline{bash}{--detach}),
    \item [\texttt{-v}] creates a bind mount (same as \mintinline{bash}{--volume}).
\end{compactitem}

Example.
Applications can be made more secure by running them in read-only mode.
Read only containers may still need to write temporary data:
\begin{bashcode}
$ docker run --read-only --tmpfs /run --tmpfs /tmp \
>     -it fedora /bin/bash
\end{bashcode}

Example.
If you want messages that are logged in your container to show up in the host's syslog/journal:
\begin{bashcode}
$ docker run -v /dev/log:/dev/log -it fedora /bin/bash
(bash) # logger "Hello from my container"; exit

$ journalctl -b | grep Hello
# maj 20 09:03:20 plavi-pc root[8831]: Hello from my container
\end{bashcode}
}

%
\section{Manage containers}
\subsection{Running containers}
Run a container with random name (an adjective with surname of a notable scientist or hacker):
\begin{bashcode}
$ docker run hello-world # = docker container run ...
Unable to find image 'hello-world:latest' locally
latest: Pulling from library/hello-world
\end{bashcode}

Assigns custom name to the container:
\begin{bashcode}
$ docker run --name='cobra' python
\end{bashcode}

Useful switches include:
\begin{bashcode}
# -d == --detach=true runs in the background
# `docker container attach` attaches local stdin & stdout

# -i == --interactive keeps STDIN open
# -t == --tty allocaes a pseudo-TTY
\end{bashcode}

\subsection{Listing containers}
List running containers:
\begin{bashcode}
$ docker ps = # docker container ls
CONTAINER ID IMAGE   COMMAND CREATED    STATUS    PORTS NAMES
fdde5a2d9731 busybox "sh"    1 hour ago Up 1 hour       great_g
\end{bashcode}

Useful switches include:
\begin{bashcode}
# -a == --all lists all, not just running containers
# -q == --quiet lists only container IDs
# -s == --size lists total file sizes
\end{bashcode}

List port mappings:
\begin{bashcode}
$ docker port static-site # = docker container port ...
\end{bashcode}

List logs of a container that was started with the json-file or journald logging driver:
\begin{bashcode}
$ docker run --name test -d busybox sh -c \
    "while true; do $(echo date); sleep 1; done"
$ docker logs -f --until=2s test # = docker container logs ...
# -f == --follow works like in `tail` command
# --until only logs since timestamp (like 2013-01-02T13:23:37Z)
#         or relative (like 42m),
# --since only logs until a timestamp or relative (see above).
\end{bashcode}

\subsection{Starting, pausing, stopping containers}
Pause/unpause all processes within containers:
\begin{bashcode}
$ docker pause cobra   # = docker container pause ...
$ docker unpause cobra # = docker container unpause ...
\end{bashcode}

Start/stop/kill:
\begin{bashcode}
$ docker start fdde5a2d9731 # = docker container start ...
$ docker stop fdde5a2d9731  # = docker container stop ...
# --time: seconds to wait before stopping
$ docker kill fdde5a2d9731  # = docker container kill ...
\end{bashcode}

\subsection{Running commands inside containers}
Execute a command inside the container:

\begin{bashcode}
$ docker exec -it ubuntu bash # = docker container exec ...
\end{bashcode}

By default it's executed in the working directory that was set during container creation.
Does not work for paused containers!
Useful switches:
\begin{bashcode}
# --workdir: custom working directory inside the container.
\end{bashcode}
    
\subsection{Removing containers}
Remove a stopped container:
\begin{bashcode}
$ docker rm fdde5a2d9731 # = docker container rm ...
\end{bashcode}

Useful switches include:
\begin{bashcode}
# -f == --force forces the removal (uses SIGKILL).
# -v == --volumes removes associated anonymous volumes
\end{bashcode}

Remove all stopped containers:
\begin{bashcode}
$ docker container prune                        # requires API >= 1.25
$ docker rm $(docker ps -a -q -f status=exited) # does the same
\end{bashcode}

\subsection{Work in progress}
\color{red}
\important{Create container}, don't start it:
\begin{bashcode}
$ docker container create ...
\end{bashcode}

\important{Rename container}
\begin{bashcode}
$ docker container rename ...
\end{bashcode}

Display running processes of a container:
\begin{bashcode}
$ docker container top bold_cerf
UID  PID   PPID  C STIME TTY   TIME     CMD
root 16433 16412 0 00:42 pts/0 00:00:00 bash
\end{bashcode}

Fetch the logs of a container:

\important{Run a command} in a running container:
\begin{bashcode}
$ docker container exec -it bold_cerf bash
\end{bashcode}

\important{Kill a cointainer}
\begin{bashcode}
$ docker container kill bold_cerf
\end{bashcode}

You can kill them all too:
\begin{bashcode}
$ docker kill $(docker ps -q)
\end{bashcode}

Useful switches:
\begin{compactitem}
    \item \mintinline{bash}{--force}: uses SIGKILL,
\end{compactitem}
\color{black}

%
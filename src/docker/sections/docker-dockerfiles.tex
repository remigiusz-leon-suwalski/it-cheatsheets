%

\section{Manage Dockerfiles}
A Dockerfile is a text document that contains all the commands a user could call on the command line to assemble an image.
\begin{bashcode}
$ cat hello.sh
#!/bin/sh
echo "Hello, docker!"
$ cat Dockerfile
FROM alpine:3.13
WORKDIR /usr/src/app
COPY hello.sh .
# RUN chmod +x hello.sh
CMD ./hello.sh
\end{bashcode}
To build:
\begin{bashcode}
$ docker build . -t hello-dokker
[+] Building 0.3s (8/8) FINISHED                                                                                             docker:default
 => [internal] load build definition from Dockerfile             0.0s
 => => transferring dockerfile: 544B                             0.0s
 => [internal] load .dockerignore                                0.0s
 => => transferring context: 2B                                  0.0s
 => [internal] load metadata for docker.io/library/alpine:3.13   0.0s
 => [1/3] FROM docker.io/library/alpine:3.13                     0.0s
 => [internal] load build context                                0.0s
 => => transferring context: 68B                                 0.0s
 => [2/3] WORKDIR /usr/src/app                                   0.1s
 => [3/3] COPY hello.sh .                                        0.0s
 => exporting to image                                           0.1s
 => => exporting layers                                          0.1s
 => => writing image sha256:2dd...b83                            0.0s
 => => naming to docker.io/library/hello-dokker                  0.0s
\end{bashcode}

It's possible but not recommended to modify exisitng image:
\begin{bashcode}
$ docker run -it hello-dokker sh
# in second terminal
$ docker ps
$ touch extra.txt
$ docker cp ./extra.txt priceless_proskuriakova:/usr/src/app/
$ docker diff priceless_proskuriakova
C /usr
C /usr/src
C /usr/src/app
A /usr/src/app/add
C /root
A /root/.ash_history
$ docker commit proskuriakova goodbye-dokker
\end{bashcode}

%
\section{Build an image from a Dockerfile}
\important{Build an image from a Dockerfile}:
\begin{bashcode}
$ docker build -t my_image .
\end{bashcode}

Last argument is path to the directory that contains the Dockerfile.

Useful switches:
\begin{compactitem}
    \item \mintinline{bash}{--tag}: applies a repository name (and optionally tag) to the resulting image.
    \item \mintinline{bash}{–no-cache}: \textbf{TO BE DONE}
\end{compactitem}

Sample content of the Dockerfile:
\begin{bashcode}
FROM python:3.8
WORKDIR /usr/src/app
COPY . .
RUN pip install --no-cache-dir -r requirements.txt
EXPOSE 5000
CMD ["python", "./app.py"]
\end{bashcode}

From \url{https://dmitryfrank.com/projects/docker-quick-ref} you can obtain a  nice printable quick reference for Dockerfile syntax, available on the MIT license from Dmitry Frank.

To \important{upload an image} to the Docker Hub, authenticate first and then push:
\begin{bashcode}
$ docker login
$ docker push <name>:<tag>
\end{bashcode}

%
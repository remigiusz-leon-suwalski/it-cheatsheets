%

\subsection{Set types}
A set object is an unordered collection of distinct hashable objects. 
There are currently two built-in set types, set (mutable) and frozenset (immutable).

\begin{minted}{python3}
len(s)
a.copy() # shallow copy

x in s
x not in s # tests for membership
s1.isdisjoint(s2) # True if intersection of s1, s2 is empty

# Comparison of a set and a frozenset is based on their members
s1 <= s2 # same as s1.issubset(s2)
s1 < s2
s1 >= s2 # same as s1.issuperset(s2)
s1 > s2

# All return the same type as the first operand
s1 | s2 | ... # same as s1.union([s2, ...])
s1 & s2 & ... # same as s1.intersection([s2, ...])
s1 - s2 - ... # same as s1.difference([s2, ...])
s1 ^ s2       # same as s1.symmetric_difference(s2)
\end{minted}

The following table lists operations available for set that do not apply to immutable instances of frozenset:
\begin{minted}{python3}
s1 |= s2 | ... # same as s1.update([s2, ...])
s1 &= s2 & ... # same as s1.intersection_update([s2, ...])
s1 -= s2 | ... # same as s1.difference_update([s2, ...])
s1 ^= s2       # same as s1.symmetric_difference_update(s2)

s1.add(x)
s1.remove(x)  # raises KeyError if x is not contained in the set.
s1.discard(x) # removes x from the set if it is present.
s1.pop()      # removes and returns an arbitrary element,
              # raises KeyError if the set is empty.
s1.clear()    # removes all elements from the set.
\end{minted}
%
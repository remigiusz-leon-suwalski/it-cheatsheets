% \input{libraries/19-program-frameworks/turtle}
% \input{libraries/19-program-frameworks/cmd}
% \input{libraries/19-program-frameworks/shlex}

%

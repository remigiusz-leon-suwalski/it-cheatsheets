%

\subsection{re -- regular expression operations}
\textbf{re} provides regular expression matching operations similar to those found in Perl.
Most important functions are: \mintinline{python}{re.match}, \mintinline{python}{re.search}, \mintinline{python}{re.findall}, \mintinline{python}{re.sub}, \mintinline{python}{re.split}.
Since regexes often use backslash character, it's a good practice to write them using raw string (prefixing with r).

See also \textbf{fnmatch} in subsection \ref{subsection_fnmatch} for shell globbing.

Both \mintinline{python}{re.match} and \mintinline{python}{re.search} check if a string matches pattern.
The difference between them is that \mintinline{python}{re.match} is anchored at the beginning of the string while \mintinline{python}{re.search} searches the entire string.

\begin{minted}{python}
def display_match(m):
    if m is None:
        return None
    return f"match {m.group()} groups {m.groups()}"

m = re.match(pattern, string)
# Python compiles and caches regexes. Therefore we don't
# get worse performance by compiling regex first, using
# it later. Above code is therefore equivalent to:
regex = re.compile(pattern)
m = regex.match(string)

valid = re.compile(r"^[akjqt2-9]{5}$")
display_match(valid.match("akt5q"))
# 'match akt5q groups ()'
display_match(valid.match("akt5r"))
# None
letters_numbers = re.compile(r'([a-z]*)-([0-9]+)')
display_match(letters_numbers.match('alfa-5009'))
# "match alfa-5009 groups ('alfa', '5009')"
\end{minted}

\mintinline{python}{re.findall} finds all non-overlapping matches of pattern in string, as a list of strings or tuples (if there are multiple capturing groups).
\begin{minted}{python}
re.findall(r'\bf[a-z]*', 'which foot or hand fell faster')
# ['foot', 'fell', 'faster']
re.findall(r'(\w+)=(\d+)', 'set width=20 and height=10')
# [('width', '20'), ('height', '10')]
\end{minted}

\mintinline{python}{re.sub(pattern, replacement, string)} finds and replaces leftmost non-overlapping occurrences of a pattern.
\begin{minted}{python}
re.sub(r'(\d+)', r'digits-\1', 'alfa-3141-5926')
# 'alfa-digits-3141-digits-5926'
re.sub(r'aba', 'efe', 'abababa')
# 'efebefe'
\end{minted}

\mintinline{python}{re.split(pattern, string)} splits a string by the occurrences of a pattern.
\begin{minted}{python}
re.split('[a-f]+', '0a3B9', flags=re.IGNORECASE)
['0', '3', '9']
\end{minted}
%

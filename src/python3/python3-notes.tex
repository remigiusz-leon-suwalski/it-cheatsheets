\documentclass{charun}
\title{Python 3.11.3 notes, version 0.23.0}
\author{Leon Suwalski}

\usepackage{xcolor}
% copied from https://www.color-hex.com/color-palette/85828
\definecolor{pythoncolor}{HTML}{daffeb}
\definecolor{pythoncolor2}{HTML}{009947}

% copied from https://tex.stackexchange.com/questions/132849/how-can-i-change-the-font-size-of-the-number-in-minted-environment
\BeforeBeginEnvironment{minted}{\begin{tcolorbox}[breakable,boxsep=5pt,left=0pt,right=0pt,top=0pt,bottom=0pt,boxrule=0.5pt,arc=0pt,outer arc=0pt,colback=pythoncolor,colframe=pythoncolor2]\small}%
\AfterEndEnvironment{minted}{\end{tcolorbox}}%

\begin{document}
\begin{multicols*}{2}
\maketitle
\raggedright

\section{Built-in Functions}
To print text, use \mintinline{python}{print()}.

Logical operations:
\begin{minted}{python}
all() # False if any element is False
any() # False if all elements are False or iterable is empty
\end{minted}

Mathematical operations:
\begin{minted}{python}
abs()    # absolute value of a number
divmod() # quotient, remainder of non-complex numbers division
min()    # smallest element
max()    # largest element
pow()    # exponentiation, with optional parameter for modulus
round()  # rounds to given precision (to integer by default)
\end{minted}

\mintinline{python}{type()} returns the type of an object.
Converting between data types:
\begin{minted}{python}
bool() # anything not listed below: True
# empty sequences, collections, zeroes, None, False: False

# hex, oct, bin require an integer
hex() # to a lowercase hexadecimal string prefixed with "0x"
oct() # to an octal string prefixed with "0o"
bin() # to a binary string prefixed with "0b"
       
# int, float require a number or string
int() # converts a number or string
float() # converts a number or string

complex() # converts a pair of numbers or string
# complex("3+4j") == complex(3,4) == (3+4j)

chr() # converts integer to character, chr(97) == 'a'
ord() # converts character to integer, ord('a') == 97

tuple()
list()
dict() # dictionary
frozenset()
range(stop)
range(start, stop, step=1)
set()
str() # string
\end{minted}

Functions for sequences/iterators:
\begin{minted}{python}
enumerate() # returns tuples containing count
            # and consecutive values of iterator
len()       # number of items, or length
reversed()  # a reverse iterator

sorted()
# with custom comparison function:
sorted(xs, key=lambda x:x[-1])
# as if each comparison were reversed:
sorted(xs, reverse=True)

sum(xs, start=0)  # sums start and the items
                  # start can't be a string
"".join(xs)       # sum alternative for strings
math.fsum(xs)     # sum alternative for floats
itertools.chain() # sum alternative for a series of iterables

zip() # aggregates (into tuples) elements from each of the iterables
\end{minted}

Functional programming:
\begin{minted}{python}
# only those elements for which function is true
filter(function, iterable)

# applies function to every element of iterable
map(function, iterable)
\end{minted}

\mintinline{python}{open()}: opens a file and then closes matching file descriptor.
\begin{minted}{python}
with open("path/to/file") as f:
	for line in f:
		# do something

with open("path/to/file", "w") as f:
	f.write("some string")
\end{minted}

Yet to be described:
\mintinline{python}{ascii()},
\mintinline{python}{aiter()},
\mintinline{python}{anext()},
\mintinline{python}{breakpoint()},
\mintinline{python}{bytearray()},
\mintinline{python}{bytes()},
\mintinline{python}{callable()},
\mintinline{python}{classmethod()},
\mintinline{python}{compile()},
\mintinline{python}{delattr()},
\mintinline{python}{dir()},
\mintinline{python}{eval()},
\mintinline{python}{exec()},
\mintinline{python}{format()},
\mintinline{python}{getattr()},
\mintinline{python}{globals()},
\mintinline{python}{hasattr()},
\mintinline{python}{hash()},
\mintinline{python}{help()},
\mintinline{python}{id()},
\mintinline{python}{input()},
\mintinline{python}{isinstance()},
\mintinline{python}{issubclass()},
\mintinline{python}{iter()},
\mintinline{python}{locals()},
\mintinline{python}{memoryview()},
\mintinline{python}{next()},
\mintinline{python}{object()},
\mintinline{python}{property()},
\mintinline{python}{repr()},
\mintinline{python}{setattr()},
\mintinline{python}{slice()},
\mintinline{python}{staticmethod()},
\mintinline{python}{super()},
\mintinline{python}{vars()}.

%
% Built-in Constants => skipped on purpose
\section{Built-in types}
The principal built-in types are numerics, sequences, mappings, classes, instances and exceptions.

\subsection{Boolean types. Comparisons}
Here are most of the built-in objects considered false:
\begin{itemize}
\item constants defined to be false: \mintinline{python}{None} and \mintinline{python}{False}.
\item zero of any numeric type: \mintinline{python}{0}, \mintinline{python}{0.0}, \mintinline{python}{0j}, \mintinline{python}{Decimal(0)}, \mintinline{python}{Fraction(0, 1)}
\item empty sequences and collections: \mintinline{python}{''}, \mintinline{python}{()}, \mintinline{python}{[]}, \mintinline{python}{{}}, \mintinline{python}{set()}, \mintinline{python}{range(0)}.
\end{itemize}

Warning: \mintinline{python}{not a == b} means \mintinline{python}{not (a == b)}.

Operations ordered by ascending priority (first two are short-circuit: evaluate second argument only if needed):
\begin{itemize}
	\item \mintinline{python}{p or q}: if p is false, then q, else p
	\item \mintinline{python}{p and q}: if p is false, then p, else q
	\item \mintinline{python}{not p}: if p is false, then \mintinline{python}{True}, else \mintinline{python}{False}
\end{itemize}

All eight operations have the same priority (higher than that of the Boolean operations).
Can be chained.
They are: 
\mintinline{python}{<}, 
\mintinline{python}{<=}, 
\mintinline{python}{>}, 
\mintinline{python}{>=}, 
\mintinline{python}{==} (equal), 
\mintinline{python}{!=} (not equal), 
\mintinline{python}{is} (object identity), 
\mintinline{python}{is not}.

For the string and bytes types, \mintinline{python}{x in y} checks if \mintinline{python}{x} is a substring of \mintinline{python}{y}.

%
\subsection{Numeric types}
All numeric types (except complex) support:
\mintinline{python}{x + y},
\mintinline{python}{x - y},
\mintinline{python}{x * y},
\mintinline{python}{x / y} (quotient),
\mintinline{python}{x // y} (floored quotient),
\mintinline{python}{x % y},
\mintinline{python}{abs(x)},
\mintinline{python}{x ** y} (power).

\subsubsection{int -- a numeric type}
Booleans are a subtype of integers.
Integers have unlimited precision.

Bitwise operations make sense for integers: \mintinline{python}{x | y} (bitwise or), \mintinline{python}{x ^ y} (bitwise xor), \mintinline{python}{x & y} (bitwise and), \mintinline{python}{x << n} (left shift), \mintinline{python}{x >> n} (right shift).
% Todo: underscores in integers (requires Python 3.6+).
\subsubsection{float -- a numeric type}
Floating point numbers are usually implemented using double in C; information about the precision is available in \mintinline{python}{sys.float_info}.
\subsubsection{complex -- a numeric type}
Complex numbers have a real and imaginary part, which are each a floating point number.
To extract these parts from a complex number \mintinline{python}{z}, use \mintinline{python}{z.real} and \mintinline{python}{z.imag}.


%
% Iterator types => skipped on purpose
% Generator types => skipped on purpose
\subsection{Sequence types}
Common sequence operations include:
\begin{itemize}
    \item
    \mintinline{python}{x in s} (true iff some item of s is equal to x),
    \mintinline{python}{x not in s},

    \item
    \mintinline{python}{x + s} (concatenation),
    \mintinline{python}{s * n},
    \mintinline{python}{n * s} (iterated concatenation),

    \item
    \mintinline{python}{s[i]},
    \mintinline{python}{s[i:j]} (slice),
    \mintinline{python}{s[i:j:k]} (slice with custom step),

    \item
    \mintinline{python}{len(s)},
    \mintinline{python}{min(s)},
    \mintinline{python}{max(s)},

    \item
    \mintinline{python}{s.index(x[, i[, j]])} index of first occurence of x in s (at or after index i and before index j),

    \item
    \mintinline{python}{s.count(x)} (total number of occurrences).
\end{itemize}

Mutable sequence types implement as well:
\begin{itemize}
    \item
    \mintinline{python}{del s[i:j]},
    \mintinline{python}{s.clear()} (removes all items),
    \mintinline{python}{s.pop([i])} (retrieves and removes element),
    \mintinline{python}{s.remove(x)} (removes element),

    \item
    \mintinline{python}{s.copy()} (creates shallow copy),

    \item
    \mintinline{python}{s.append(x)} (adds one element to the end),
    \mintinline{python}{s.insert(i, x)}, (adds in specified position),
    \mintinline{python}{s.extend(t)} (adds many elements to the end),

    \item
    \mintinline{python}{s.reverse()} (reverses in place).
\end{itemize}

Lists are mutable.
Tuples and ranges are not.

\subsection{str -- a text sequence type}

Don't use printf-style \mintinline{python}{%} syntax.
Good -- string \mintinline{latex}{.format()} method:
\begin{minted}{python}
a = "Hello"
b = "world"
c = f"{a} {b}!"
c = "{1} {0}!".format(b, a)

d = {"name": "John", "age": 37}
"Hello {name}-{age}!".format(name=d["name"], age=d["age"])
"Hello {name}-{age}!".format(**d)
\end{minted}

Better -- formatted string literals (also called f-strings for short).
Require Python 3.6+.
\begin{minted}{python}
f"Pi is approximately {math.pi:.3f}."
f"2 + 2 = {2 + 2}"
\end{minted}
Because f-strings are evaluated at runtime, you can put any and all valid Python expressions in them.
In order to make a brace appear in your string, you must use double braces.

%
\subsection{Text sequence type 'str'}
String literals can be single, double or triple quoted (triple quoted strings may span multiple lines).
Additional methods for strings (besides all of the common sequence operations) are:
\begin{itemize}
    \item
\mintinline{python}{.capitalize()},
\mintinline{python}{.casefold()} (converts \mintinline{python}{"ß"} aggresively to \mintinline{python}{"ss"}),
\mintinline{python}{.lower()},
\mintinline{python}{.swapcase()},
\mintinline{python}{.title()},
\mintinline{python}{.upper},
    \item
    \mintinline{python}{.center(width)},
    \mintinline{python}{.ljust(width)},
    \mintinline{python}{.rjust(width)},
    \mintinline{python}{.zfill(width)} (warning: \mintinline{python}{"-42".zfill(5) == "-0042"}),
    \mintinline{python}{.expandtabs(tabsize=8)},
    \mintinline{python}{.lstrip()},
    \mintinline{python}{.rstrip()},
    \mintinline{python}{.strip()},
    \mintinline{python}{.removeprefix(prefix)},
    \mintinline{python}{.removesuffix(suffix)},
    \item
    \mintinline{python}{.encode(encoding="utf-8")} (string encoded to bytes),
    \item
    \mintinline{python}{.endswith(suffix)},
    \mintinline{python}{.startswith(prefix)},
        \item
    \mintinline{python}{.find(sub)},
    \mintinline{python}{.rfind(sub)},
    \mintinline{python}{.index(sub)} (like find but raises ValueError if string is not found),
    \mintinline{python}{.rindex(sub)},
    \item
    \mintinline{python}{.isalnum()},
    \mintinline{python}{.isalpha()},
    \mintinline{python}{.isascii()},
    \mintinline{python}{.isdecimal()},
    \mintinline{python}{.isdigit()},
    \mintinline{python}{.isnumeric()},
    \mintinline{python}{.islower()},
    \mintinline{python}{.istitle()},
    \mintinline{python}{.isupper()},
    \item
    \mintinline{python}{.join(iterable)},
    \mintinline{python}{.partition(sep)} (part before separator, separator, part after),
    \mintinline{python}{.rpartition(sep)},
    \mintinline{python}{.split()}
    \mintinline{python}{.rsplit()} 
    \item
    \mintinline{python}{.replace(old, new, count=1)}.
\end{itemize}

Don't use printf-style \mintinline{python}{%} syntax; string \mintinline{latex}{.format()} method is better:
\begin{minted}{python}
a = "Hello"
b = "world"
c = f"{a} {b}!"
c = "{1} {0}!".format(b, a)

d = {"name": "John", "age": 37}
"Hello {name}-{age}!".format(name=d["name"], age=d["age"])
"Hello {name}-{age}!".format(**d)
\end{minted}

Even better are formatted string literals (also called f-strings for short).
Require Python 3.6+.
\begin{minted}{python}
f"Pi is approximately {math.pi:.3f}."
f"2 + 2 = {2 + 2}"
\end{minted}
Because f-strings are evaluated at runtime, you can put any and all valid Python expressions in them.
In order to make a brace appear in your string, you must use double braces.
\subsection{Binary sequence types}
%     Binary Sequence Types -- bytes, bytearray, memoryview

%
\subsection{Set types}
%     Set Types -- set, frozenset

\subsection{Mapping types}
\subsubsection{dict -- a mapping type}

To merge two dictionaries:
\begin{minted}{python}
z = x | y      # >= Python 3.9
z = {**x, **y} # >= Python 3.5
\end{minted}

%
%     Context Manager Types
%     Generic Alias Type
%     Other Built-in Types
%     Special Attributes

%

%
\section{Built-in exceptions}
\subsection{Base classes}
To be done...

\subsection{Concrete exceptions}
To be done...

\subsection{Warnings}
To be done...

\subsection{Exception hierarchy}
To be done...

%

\section{Standard library modules}
%

% %

\subsection{string -- common string operations}

%
%

\subsection{re -- regular expression operations}

\mintinline{python}{re.findall(pattern, string)} finds all non-overlapping matches of pattern in string, as a list of strings or tuples (if there are multiple capturing groups).

%

% %

\subsection{difflib -- helpers for computing deltas}

%
% \subsection{textwrap -- text wrapping and filling}
To be done...

%
% %

\subsection{unicodedata -- Unicode Database}

%
% %

\subsection{stringprep -- Internet String Preparation}

%

% \subsection{readline -- GNU readline interface}
To be done...

%

% \subsection{rlcompleter -- completion function for GNU readline}
To be done...

%

%
\section{Binary Data Services}
\subsection{struct -- Interpret bytes as packed binary data}
To be done ....

%

\subsection{rlcompleter -- completion function for GNU readline}
To be done...

%

%
\section{Data Types}
\subsection{datetime -- basic date and time types}
Converting string into datetime objects:
\begin{minted}{python}
from datetime import datetime

datetime_object = datetime.strptime(
    'Jun 1 2005  1:33PM', '%b %d %Y %I:%M%p'
)
\end{minted}

Current date and time: \mintinline{python}{datetime.now()}, just the time: \mintinline{python}{datetime.now().time()}

% date_time_obj = datetime.datetime.strptime(raw_date, '%Y-%m-%d')
% datetime.datetime.now()
% datetime.datetime.now() - date_time_obj).days
% .total_seconds()
% datetime.fromtimestamp
%     zoneinfo -- IANA time zone support
%     calendar -- General calendar-related functions
%

\subsection{collections -- container datatypes}
To count multiple items:
\begin{minted}{python}
from collections import Counter
z = ['blue', 'red', 'blue', 'yellow', 'blue', 'red']
Counter(z) # Counter({'blue': 3, 'red': 2, 'yellow': 1})
\end{minted}

%
%     collections.abc -- Abstract Base Classes for Containers
%     heapq -- Heap queue algorithm
%     bisect -- Array bisection algorithm
%     array -- Efficient arrays of numeric values
%     weakref -- Weak references
%     types -- Dynamic type creation and names for built-in types
%     copy -- Shallow and deep copy operations
%

\subsection{pprint -- data pretty printer}
\begin{minted}{python}
from pprint import pprint
coordinates = [{'name': 'Location 1',
'gps': (29, 111)},
{'name': 'Location 2',
'gps': (40, 44)}]
pprint(coordinates)
# [{'gps': (29, 111), 'name': 'Location 1'},
#  {'gps': (40, 44), 'name': 'Location 2'}]
\end{minted}

%

%     reprlib -- Alternate repr() implementation
%     enum -- Support for enumerations
%     graphlib -- Functionality to operate with graph-like structures

%
\section{Numeric and Mathematical Modules}
%     numbers -- Numeric abstract base classes
%     math -- Mathematical functions
%     cmath -- Mathematical functions for complex numbers
%     decimal -- Decimal fixed point and floating point arithmetic
%     fractions -- Rational numbers
\subsection{random -- generate pseudo-random numbers}
To randomly select an item from a list:
\begin{minted}{python}
import random

foo = random.choice(['a', 'b', 'c', 'd', 'e'])
\end{minted}
For cryptographically secure random choices use \mintinline{python}{secrets.choice()} (requires Python 3.6).

% randint, seed, sample

%     statistics -- Mathematical statistics functions

%
\section{Functional Programming Modules}
%     itertools -- Functions creating iterators for efficient looping
%

\subsection{functools -- higher-order functions and operations on callable objects}

% cmp_to_key

%     operator -- Standard operators as functions

%
%

% \subsection{pathlib -- object-oriented filesystem paths}
To get the file extension (requires Python 3.4):
\begin{minted}{python}
import pathlib

print(pathlib.Path('yourPath.example').suffix)
# '.example'
\end{minted}

%

%     os.path -- Common pathname manipulations
%     fileinput -- Iterate over lines from multiple input streams
%     stat -- Interpreting stat() results
%     filecmp -- File and Directory Comparisons
%     tempfile -- Generate temporary files and directories
%     glob -- Unix style pathname pattern expansion
%

\subsection{fnmatch -- Unix filename pattern matching}
\label{subsection_fnmatch}%
\textbf{fnmatch} module provides support for Unix shell-style wildcards: asterisk (everything), question mark (any single character) and square brackets (any character in given set).

\begin{minted}{python}
if fnmatch.fnmatch(some_file, '*.txt'):
    print("Case insensitive match", some_file)

if fnmatch.fnmatchcase(some_file, '*.txt'):
    print("Case sensitive match", some_file)
\end{minted}

%
%     linecache -- Random access to text lines
%     shutil -- High-level file operations

%
\section{Data Persistence}
%     pickle -- Python object serialization
%     copyreg -- Register pickle support functions
%     shelve -- Python object persistence
%     marshal -- Internal Python object serialization
%     dbm -- Interfaces to Unix “databases”
%     sqlite3 -- DB-API 2.0 interface for SQLite databases

%
\section{Data Compression and Archiving}
%     zlib -- Compression compatible with gzip
%     gzip -- Support for gzip files
%     bz2 -- Support for bzip2 compression
%     lzma -- Compression using the LZMA algorithm
%     zipfile -- Work with ZIP archives
%     tarfile -- Read and write tar archive files

%
\section{File Formats}
%     csv -- CSV File Reading and Writing
%     configparser -- Configuration file parser
%     netrc -- netrc file processing
\subsection{xdrlib -- Encode and decode XDR data}
To be done ....

%

%     plistlib -- Generate and parse Apple .plist files

%
\subsection{hashlib -- Secure hashes and message digests}
\begin{minted}{python}
>>> hashlib.md5("John Doe".encode("utf-8"))
<md5 _hashlib.HASH object @ 0x779f00a45af0>
>>> hashlib.md5("John Doe".encode("utf-8")).hexdigest()
'4c2a904bafba06591225113ad17b5cec'

# This does not work
>>> hashlib.md5("John Doe")
Traceback (most recent call last):
  File "<stdin>", line 1, in <module>
TypeError: Strings must be encoded before hashing
\end{minted}
%

\subsection{hmac -- Keyed-Hashing for Message Authentication}
To be done ....

%

\subsection{secrets -- Generate secure random numbers for managing secrets}
To be done ....

%


%

\subsection{os -- miscellaneous operating system interfaces}
A dict-like object with global variables:
\begin{minted}{python}
import os
os.environ
# environ({'SHELL': '/bin/bash', 'EDITOR': 'vim'})

os.environ["PATH"]
# '/usr/local/bin:/usr/bin:/bin'
\end{minted}

% os.path.isfile(filename):
% os.path.join(..., ...)
% not here - in os.path!

\subsection{io -- Core tools for working with streams}
To be done ....

%

\subsection{time -- time access and conversions}
Wait (in seconds):
\begin{minted}{python}
import time
# in <= Python 3.4: 5 seconds, more or less
# in >= Python 3.5: 5 seconds, not less
time.sleep(5)
\end{minted}

%

\subsection{argparse -- parser for command-line options, arguments and sub-commands}
In brute verterem consectetuer eos, eirmod nusquam pertinax mei te. Aperiam definitiones pri ei, est ut nibh reque animal, dicunt gubergren usu ex. Nibh ipsum vel ex, sit cu purto omnes. Est choro phaedrum id, his affert dissentiunt conclusionemque ei. Nam enim commodo percipitur id, homero praesent mei ea, quo quis probo complectitur ut.

Est quodsi debitis temporibus at, nisl probo cu ius. Quod salutandi gubergren cu mea, fastidii singulis referrentur mei ea. Sit te sumo aliquid percipit, quo errem periculis id. Sea ei dolore placerat assueverit, vim ipsum tantas graeco cu, ea aperiam denique mel. Meis munere ad per, accumsan ullamcorper ad pri, vix et rebum conceptam. Detracto eligendi incorrupte per ea, has ei modo interesset.

\subsection{getopt -- C-style parser for command line options}
To be done ....

%

\subsection{logging -- logging facility for Python}
Default logging levels (from least important):
\begin{minted}{python}
import logging

logging.debug('This is a debug message')
logging.info('This is an info message')
logging.warning('This is a warning message')
logging.error('This is an error message')
logging.critical('This is a critical message')
\end{minted}

Simple configuration:
\begin{minted}{python}
logging.basicConfig(
    level=logging.DEBUG,
    datefmt='%d-%b-%y %H:%M:%S',
    filename='app.log',
    filemode='w',
    format='==> %(message)s'
)
\end{minted}

Available attributes:
\begin{itemize}
	\item \textbf{asctime} human-readable time like ‘2003-07-08 16:49:45,896’
	\item \textbf{relativeCreated} milliseconds, relative to '\mintinline{python}{import logging}' time
	\item \textbf{levelname} text logging level for the message like 'CRITICAL'
	\item \textbf{lineno} source line number if available
	\item \textbf{message} the logged message
	\item \textbf{name} name of the logger
	\item \textbf{process}, \textbf{processName} process ID and name, if available
	\item \textbf{thread}, \textbf{threadName} thread ID and name, if available
\end{itemize}

Capturing the full stack trace:
\begin{minted}{python}
try:
    c = a / b
except Exception as e:
    logging.exception("Exception occurred")
    # if other level is desired than error:
    logging.debug("Exception occurred", exc_info=True)
\end{minted}

Multiple loggers:
\begin{minted}{python}
logger = logging.getLogger('example_logger')
logger.setLevel(logging.INFO)
logger.warning('This is a warning')
# setting format: stackoverflow 11581794 :(
\end{minted}

Todo:
\begin{minted}{python}
.Formatter()
\end{minted}

%     logging.config -- Logging configuration
%     logging.handlers -- Logging handlers
\subsection{getpass -- Portable password input}
To be done ....

%

\subsection{curses -- Terminal handling for character-cell displays}
To be done ....

%

%     curses.textpad -- Text input widget for curses programs
%     curses.ascii -- Utilities for ASCII characters
%     curses.panel -- A panel stack extension for curses
%     platform -- Access to underlying platform’s identifying data
\subsection{errno -- Standard errno system symbols}
To be done ....

%

\subsection{ctypes -- A foreign function library for Python}
To be done ....

%


%

\section{Concurrent Execution}
\subsection{threading -- Thread-based parallelism}
To be done ....

%

\subsection{multiprocessing -- Process-based parallelism}
To be done ....

%

%     multiprocessing.shared_memory -- Provides shared memory for direct access across processes
%     The concurrent package
%     concurrent.futures -- Launching parallel tasks
\subsection{subprocess -- Subprocess management}
Call an external command:
\begin{minted}{python}
import subprocess
subprocess.run(["ls", "-l"])  # >= Python 3.5
subprocess.call(["ls", "-l"]) # <= Python 3.4
\end{minted}

The advantage of subprocess.run over os.system is that it is more flexible: you can get the stdout, stderr, the real status code, better error handling, etc.

%

\subsection{sched -- Event scheduler}
To be done ....

%

\subsection{queue -- A synchronized queue class}
To be done ....

%

\subsection{contextvars -- Context Variables}
To be done ....

%

%     _thread -- Low-level threading API

%     asyncio -- Asynchronous I/O
% \subsection{socket -- Low-level networking interface}
To be done ....

%

%     ssl -- TLS/SSL wrapper for socket objects
%     select -- Waiting for I/O completion
%     selectors -- High-level I/O multiplexing
% \subsection{asyncore -- Asynchronous socket handler}
To be done ....

%

%     asynchat -- Asynchronous socket command/response handler
% \subsection{signal -- Set handlers for asynchronous events}
To be done ....

%

% \subsection{mmap -- Memory-mapped file support}
To be done ....

%


%

% \subsection{email -- An email and MIME handling package}
To be done ....

%

%

\subsection{json -- JSON encoder and decoder}
Converting plaintext to JSON:
\begin{minted}{python}
x = '{"name": "John", "age": 30}'
y = json.loads(x)
with open("response.json", "r") as f:
    z = json.load(f)
\end{minted}

Converting JSON to plaintext:
\begin{minted}{python}
# dump and dumps use the same keyword arguments
x = json.dumps(y, indent=4, sort_keys=True)
with open("response.json", "w") as f:
    json.dump(z, f)
\end{minted}

%
% \subsection{mailcap -- Mailcap file handling}
To be done ....

%

% \subsection{mailbox -- Manipulate mailboxes in various formats}
To be done ....

%

% \subsection{mimetypes -- Map filenames to MIME types}
To be done ....

%

%     base64 -- Base16, Base32, Base64, Base85 Data Encodings
%     binhex -- Encode and decode binhex4 files
% \subsection{binascii -- Convert between binary and ASCII}
To be done ....

%

% \subsection{quopri -- Encode and decode MIME quoted-printable data}
To be done ....

%

% \subsection{uu -- Encode and decode uuencode files}
To be done ....

%


\section{Structured Markup Processing Tools}
\subsection{html -- HyperText Markup Language support}
To be done ....

%

%     html.parser -- Simple HTML and XHTML parser
%     html.entities -- Definitions of HTML general entities
%     XML Processing Modules
%     xml.etree.ElementTree -- The ElementTree XML API
%     xml.dom -- The Document Object Model API
%     xml.dom.minidom -- Minimal DOM implementation
%     xml.dom.pulldom -- Support for building partial DOM trees
%     xml.sax -- Support for SAX2 parsers
%     xml.sax.handler -- Base classes for SAX handlers
%     xml.sax.saxutils -- SAX Utilities
%     xml.sax.xmlreader -- Interface for XML parsers
%     xml.parsers.expat -- Fast XML parsing using Expat
\subsection{xmltodict -- work with XML like with JSON (pypi)}
Duo an gloriatur suscipiantur, eum ut case expetenda temporibus. Pri cu primis option, ex vis posse explicari hendrerit. Nam te soleat sententiae, id putant scripserit vim. Ludus postea est ei, vix an civibus senserit. Solum dicta tempor an vix.

Feugait deserunt ne has, te odio brute tritani duo, vim graece alterum adolescens ea. Vel reque quaerendum te, qui libris verear quaestio eu. Cum apeirian electram ex, ea quo eirmod maiorum appellantur, est bonorum nominavi at. Dicam meliore facilisis qui ea. Sea doming labores te.

% with open(sys.argv[1]) as f:
%     doc = xmltodict.parse(f.read())


%
% \subsection{webbrowser -- Convenient Web-browser controller}
To be done ....

%

% \subsection{cgi -- Common Gateway Interface support}
To be done ....

%

% \subsection{cgitb -- Traceback manager for CGI scripts}
To be done ....

%

% \subsection{wsgiref -- WSGI Utilities and Reference Implementation}
To be done ....

%

% \subsection{urllib -- URL handling modules}
To be done ....

%

%     urllib.request -- Extensible library for opening URLs
%     urllib.response -- Response classes used by urllib
% \subsection{urllib.parse -- parse URLs into components}
Pro ne nemore singulis deseruisse, agam velit expetendis est ne. Stet invenire sea te. Ne perfecto constituto nam. Partem detraxit quo ea, ut doming debitis persequeris eam. Autem docendi menandri eum ut, te mel alia labore prodesset, vel eu integre suavitate. Te facer ludus accusamus sed, quo habeo malis ad.

Cum justo verterem ut. Ut mel eros noster democritum. Per ridens corpora suscipiantur in, veri laudem ei mel, in eripuit meliore docendi sea. Appetere probatus liberavisse nec ne, alii adhuc interpretaris an eos. Est ad doming possit discere, ad primis civibus duo, ad primis dissentias sit.

% quote

%     urllib.error -- Exception classes raised by urllib.request
%     urllib.robotparser -- Parser for robots.txt
% \subsection{requests -- simple yet elegant HTTP library (pypi)}
\begin{minted}{python}
import requests
r = requests.get('https://api.github.com/user',
    auth=('user', 'pass'))
r.status_code
# 200
r.headers['content-type']
# 'application/json; charset=utf8'
r.encoding
# 'utf-8'
r.text
# '{"type":"User"...'
r.json()
# {'disk_usage': 368627, 'private_gists': 484, ...}
\end{minted}
% \subsection{http -- HTTP modules}
To be done ....

%

%     http.client -- HTTP protocol client
% \subsection{ftplib -- FTP protocol client}
To be done ....

%

%     poplib -- POP3 protocol client
%     imaplib -- IMAP4 protocol client
% \subsection{nntplib -- NNTP protocol client}
To be done ....

%

% \subsection{smtplib -- SMTP protocol client}
To be done ....

%

% \subsection{smtpd -- SMTP Server}
To be done ....

%

% \subsection{telnetlib -- Telnet client}
To be done ....

%

%     uuid -- UUID objects according to RFC 4122
% \subsection{socketserver -- A framework for network servers}
To be done ....

%

%     http.server -- HTTP servers
%     http.cookies -- HTTP state management
%     http.cookiejar -- Cookie handling for HTTP clients
% \subsection{xmlrpc -- XMLRPC server and client modules}
To be done ....

%

%     xmlrpc.client -- XML-RPC client access
%     xmlrpc.server -- Basic XML-RPC servers
%     ipaddress -- IPv4/IPv6 manipulation library

%

% \subsection{audioop -- Manipulate raw audio data}
To be done ....

%

% \subsection{aifc -- Read and write AIFF and AIFC files}
To be done ....

%

% \subsection{sunau -- Read and write Sun AU files}
To be done ....

%

% \subsection{wave -- Read and write WAV files}
To be done ....

%

% \subsection{chunk -- Read IFF chunked data}
To be done ....

%

% \subsection{colorsys -- Conversions between color systems}
To be done ....

%

% \subsection{imghdr -- Determine the type of an image}
To be done ....

%

% \subsection{sndhdr -- Determine type of sound file}
To be done ....

%

% \subsection{ossaudiodev -- Access to OSS-compatible audio devices}
To be done ....

%


%

\section{Internationalization}
\subsection{gettext -- Multilingual internationalization services}
To be done ....

%

\subsection{locale -- Internationalization services}
To be done ....

%


%
\section{Program Frameworks}
\subsection{turtle -- Turtle graphics}
To be done ....

%

\subsection{cmd -- Support for line-oriented command interpreters}
To be done ....

%

\subsection{shlex -- Simple lexical analysis}
To be done ....

%


%
%     tkinter -- Python interface to Tcl/Tk
%     tkinter.colorchooser -- Color choosing dialog
%     tkinter.font -- Tkinter font wrapper
%     Tkinter Dialogs
%     tkinter.messagebox -- Tkinter message prompts
%     tkinter.scrolledtext -- Scrolled Text Widget
%     tkinter.dnd -- Drag and drop support
%     tkinter.ttk -- Tk themed widgets
%     tkinter.tix -- Extension widgets for Tk
%     IDLE
%     Other Graphical User Interface Packages

%

% \subsection{typing -- Support for type hints}
Example:

\begin{minted}{python}
def greeting(name: str) -> str:
    return 'Hello ' + name

greeting(2)
Traceback (most recent call last):
  File "<stdin>", line 1, in <module>
  File "<stdin>", line 2, in greeting
TypeError: can only concatenate str (not "int") to str
\end{minted}

%

% \subsection{pydoc -- Documentation generator and online help system}
To be done ....

%

%     Python Development Mode
%     Effects of the Python Development Mode
%     ResourceWarning Example
%     Bad file descriptor error example
% \subsection{doctest -- Test interactive Python examples}
To be done ....

%

% \subsection{unittest -- Unit testing framework}
To be done ....

%

%     unittest.mock -- mock object library
%     unittest.mock -- getting started
%     2to3 - Automated Python 2 to 3 code translation
% \subsection{test -- Regression tests package for Python}
To be done ....

%

%     test.support -- Utilities for the Python test suite
%     test.support.socket_helper -- Utilities for socket tests
%     test.support.script_helper -- Utilities for the Python execution tests
%     test.support.bytecode_helper -- Support tools for testing correct bytecode generation

%

\section{Debugging and Profiling}
%     Audit events table
\subsection{bdb -- Debugger framework}
To be done ....

%

\subsection{faulthandler -- Dump the Python traceback}
To be done ....

%

\subsection{pdb -- The Python Debugger}
To be done ....

%

%     The Python Profilers
\subsection{timeit -- measure execution time of small code snippets}
\begin{minted}{python}
import timeit
timeit.timeit("sum(range(1,10))")
0.2445767649987829
\end{minted}

\subsection{trace -- Trace or track Python statement execution}
To be done ....

%

\subsection{tracemalloc -- Trace memory allocations}
To be done ....

%


%
% \subsection{distutils -- Building and installing Python modules}
To be done ....

%

% \subsection{ensurepip -- Bootstrapping the pip installer}
To be done ....

%

% \subsection{venv -- Creation of virtual environments}
To be done ....

%

% \subsection{zipapp -- Manage executable Python zip archives}
To be done ....

%


%

%

%

\subsection{sys -- system-specific parameters and functions}

\mintinline{python}{sys.argv} is a list of command line arguments, its first item equals the script name (or \mintinline{bash}{-c} or an empty string).

\mintinline{python}{sys.exit([arg])} raises a SystemExit exception, where optional argument can be an integer giving the exit status (where 0 is considered successful termination).

\mintinline{python}{sys.path} is a list of strings that specifies the search path for modules (based on PYTHONPATH environmental variable plus an installation dependent default).

%

%     sysconfig -- Provide access to Python’s configuration information
% %

\subsection{Shell builtins}
\begin{compactenum}
\item [\symbolbash] \commandbash{.}: evaluates a script.

\item [\symbolbash] \commandbash{[}: defines conditionals, \commandbash{test}: checks file types and compares values. Both are discouraged, use \commandbash{[[ ]]} instead.

\item [\symbolbash] \commandbash{alias}: substitutes a string for a word,\\\commandbash{unalias}: removes defined aliases.

\item [\symbolbash] \commandbash{bg}: resumes suspended job in the background; \commandbash{fg}: brings a background or suspended process to the foreground.

\item [\symbolbash] \commandbash{bind}: binds keys to readline function/macro.

\item [\symbolbash] \commandbash{builtin}: executes the specified shell builtin.

\item [\symbolbash] \commandbash{caller}: inspects the call stack.

\item [\symbolbash] \commandbash{cd}: changes the shell working directory.
\item [\texttt{-}] to the previous directory.

\item [\symbolbash] \commandbash{command -v}: tells how shell will invoke the specified command (like '{\small \mintinline{bash}{command -v grep}}').

\item [\symbolbash] \commandbash{compgen}: generates completions for: \texttt{a}~aliases, \texttt{b}~shell builtins, \texttt{c}~all commands, \texttt{d}~directories, \texttt{e}~exported shell variables, \texttt{f}~files, \texttt{g}~groups, \texttt{j}~jobs, \texttt{k}~shell reserved words, \texttt{s}~services, \texttt{u}~users, \texttt{v}~shell variables.
\item [\symbolbash] \commandbash{complete}: handles how commands are completed when pressing Tab.
\item [\symbolbash] \commandbash{compopt}: modifies completion options (?).

\item [\symbolbash] \commandbash{declare}: declares variables and/or gives them attributes: \texttt{a} indexed array, \texttt{A} associative array, \texttt{i} integer, \texttt{l} lowercase chars only.

\item [\symbolbash] \commandbash{dirs}: lists remembered (for popd) directories.

\item [\symbolbash] \commandbash{disown}: removes entries from the active jobs table; they are still connected to the terminal!

\item [\symbolbash] \commandbash{echo}: displays a line of text:
\item [\texttt{e}] enables interpretation of backslash escapes,
\item [\texttt{n}] does not output the trailing newline.

\item [\symbolbash] \commandbash{enable}: enables/disables shell builtins.

\item [\symbolbash] \commandbash{eval}: executes a command by the shell

\item [\symbolbash] \commandbash{exec}: replaces current shell with a command.

\item [\symbolbash] \commandbash{exit}: exits the shell with a status (0..255) or, if omitted, exit code of the last command.

\item [\symbolbash] \commandbash{export}: marks variables to be passed to subshells; without args prints such variables.

\item [\symbolbash] \commandbash{fc}: \dotfill ????

\item [\symbolbash] \commandbash{getopts}: \dotfill ????

\item [\symbolbash] \commandbash{hash}: \dotfill ????

\item [\symbolbash] \commandbash{help}: displays info about shell builtins.

\item [\symbolbash] \commandbash{history}: lists previously ran commands.

\item [\symbolbash] \commandbash{jobs}: \dotfill ????

\item [\symbolbash] \commandbash{kill}: \dotfill ????

\item [\symbolbash] \commandbash{let}: evaluates arithmetic expressions.

\item [\symbolbash] \commandbash{local}: \dotfill ????

\item [\symbolbash] \commandbash{logout}: exits a login shell.

\item [\symbolbash] \commandbash{mapfile}: \dotfill ????

\item [\symbolbash] \commandbash{pushd}: adds a directory to the top of the stack and performs a cd to new top directory.
\item [\symbolbash] \commandbash{popd}: removes entries from the directory stack and performs a cd to new top directory.

\item [\symbolbash] \commandbash{printf}: \dotfill ????

\item [\symbolbash] \commandbash{pwd}: prints name of current directory.

\item [\symbolbash] \commandbash{read}: \dotfill ????

\item [\symbolbash] \commandbash{readarray}: \dotfill ????

\item [\symbolbash] \commandbash{readonly}: \dotfill ????

\item [\symbolbash] \commandbash{return}: stops function and returns a value.

\item [\symbolbash] \commandbash{set}: \dotfill ????

\item [\symbolbash] \commandbash{shift}: renames positional parameters from n+1, n+2, ... to 1, 2, ... (by default, $n = 1$).

\item [\symbolbash] \commandbash{shopt}: \dotfill ????

\item [\symbolbash] \commandbash{source}: read and execute commands from filename in the current shell environmen.

\item [\symbolbash] \commandbash{suspend}: suspends current shell until receiving of a SIGCONT signal.

\item [\symbolbash] \commandbash{times}: \dotfill ????

\item [\symbolbash] \commandbash{trap}: \dotfill ????

\item [\symbolbash] \commandbash{false}/\commandbash{true}: returns an un/successful result.

\item [\symbolbash] \commandbash{type}: \dotfill ????

\item [\symbolbash] \commandbash{typeset}: \dotfill ????

\item [\symbolbash] \commandbash{ulimit}: modifies shell resource limits.

\item [\symbolbash] \commandbash{umask}: gets/sets file mode creation mask.

\item [\symbolbash] \commandbash{unset}: unsets a shell variable, removing it from memory and the shell's exported environment.

\item [\symbolbash] \commandbash{wait}: waits for process to change state.
\end{compactenum}

Useful when writing scripts/functions:

\begin{compactenum}
\item [\symbolbash] \commandbash{break}: exits from a loop.
\item [\symbolbash] \commandbash{continue}: resumes next iteration of a loop.

\item [\symbolbash] \commandbash{local}: restricts variable scope to a function and its children; without operands lists them.

\item [\symbolbash] \commandbash{return}: stops function and returns a value.

\item [\symbolbash] \commandbash{shift}: renames positional parameters from n+1, n+2, ... to 1, 2, ... (by default, $n = 1$).

\item [\symbolbash] \commandbash{true}: does nothing, successfully.
\item [\symbolbash] \commandbash{:} (colon): like true, does nothing successfully.
\item [\symbolbash] \commandbash{false}: does nothing, unsuccessfully.
\end{compactenum}

%
%     __main__ -- Top-level script environment
% \subsection{warnings -- Warning control}
To be done ....

%

% \subsection{dataclasses -- Data Classes}
To be done ....

%

% \subsection{contextlib -- Utilities for with-statement contexts}
To be done ....

%

% \subsection{abc -- Abstract Base Classes}
To be done ....

%

% \subsection{atexit -- Exit handlers}
To be done ....

%

% \subsection{traceback -- Print or retrieve a stack traceback}
To be done ....

%

%     __future__ -- Future statement definitions
% \subsection{gc -- Garbage Collector interface}
To be done ....

%

% \subsection{inspect -- Inspect live objects}
To be done ....

%

% \subsection{site -- Site-specific configuration hook}
To be done ....

%


%

% \subsection{code -- Interpreter base classes}
To be done ....

%

% \subsection{codeop -- Compile Python code}
To be done ....

%


%

\subsection{zipimport -- Import modules from Zip archives}
To be done ....

%

\subsection{pkgutil -- Package extension utility}
To be done ....

%

\subsection{modulefinder -- Find modules used by a script}
To be done ....

%

\subsection{runpy -- Locating and executing Python modules}
To be done ....

%

\subsection{importlib -- The implementation of import}
To be done ....

%

%     Using importlib.metadata

%

\section{Python Language Services}
\subsection{parser -- Access Python parse trees}
To be done ....

%

\subsection{ast -- Abstract Syntax Trees}
To be done ....

%

%     symtable -- Access to the compiler’s symbol tables
\subsection{symbol -- Constants used with Python parse trees}
To be done ....

%

\subsection{token -- Constants used with Python parse trees}
To be done ....

%

\subsection{keyword -- Testing for Python keywords}
To be done ....

%

\subsection{tokenize -- Tokenizer for Python source}
To be done ....

%

\subsection{tabnanny -- Detection of ambiguous indentation}
To be done ....

%

\subsection{pyclbr -- Python module browser support}
To be done ....

%

%     py_compile -- Compile Python source files
\subsection{compileall -- Byte-compile Python libraries}
To be done ....

%

\subsection{dis -- Disassembler for Python bytecode}
To be done ....

%

\subsection{pickletools -- Tools for pickle developers}
To be done ....

%


%
% \subsection{formatter -- Generic output formatting}
To be done ....

%


%

% \subsection{msilib -- Read and write Microsoft Installer files}
To be done ....

%

%     msvcrt -- Useful routines from the MS VC++ runtime
% \subsection{winreg -- Windows registry access}
To be done ....

%

% \subsection{winsound -- Sound-playing interface for Windows}
To be done ....

%


%

\section{Unix Specific Services}
\subsection{posix -- The most common POSIX system calls}
To be done ....

%

\subsection{pwd -- The password database}
To be done ....

%

\subsection{spwd -- The shadow password database}
To be done ....

%

\subsection{grp -- The group database}
To be done ....

%

\subsection{crypt -- Function to check Unix passwords}
To be done ....

%

\subsection{termios -- POSIX style tty control}
To be done ....

%

\subsection{tty -- Terminal control functions}
To be done ....

%

\subsection{pty -- Pseudo-terminal utilities}
To be done ....

%

\subsection{fcntl -- The fcntl and ioctl system calls}
To be done ....

%

\subsection{pipes -- Interface to shell pipelines}
To be done ....

%

\subsection{resource -- Resource usage information}
To be done ....

%

%     nis -- Interface to Sun’s NIS (Yellow Pages)
\subsection{syslog -- Unix syslog library routines}
To be done ....

%


%
\subsection{optparse -- Parser for command line options}
To be done ....

%

\subsection{imp -- Access the import internals}
To be done ....

%


%


\section{To be done}
from collections import deque - append, maxlen, appendleft, pop
import heapq - nlargest, nsmallest, heappop, heapify, heappush
from collections import defaultdict - setdefault
OrderedDict
% naming a slice
from collections import Counter
from itertools import groupby

\end{multicols*}
\end{document}

% if __name__ == "__main__":

% with open(...) as f:
% with open(filename, "w") as ff:
% ff.write(str(message) + "\n")

% map, filter

To check:
https://stackoverflow.com/questions/tagged/python?tab=votes&page=2&pagesize=15

In Python 3, you can use the sep= and end= parameters of the print function:
print("Hello, World!", flush=True)
%

\subsubsection{Chars and strings}
The character type \mintinline{haskell}{Char} is an enumeration whose values represent Unicode code points.
A character literal in Haskell has type Char; a string is a list of characters: \mintinline{haskell}{type String = [Char]}.

\textbf{TO BE DONE!}
To convert a Char to or from the corresponding Int value defined by Unicode, use toEnum and fromEnum from the Enum class respectively (or equivalently ord and chr).

\textbf{TO BE DONE!}
For historical reasons, the \texttt{base} library uses \mintinline{haskell}{String} in a lot of places for the conceptual simplicity, but library code dealing with user-data should use the \texttt{text} package for Unicode text, or the the bytestring package for binary data.

%
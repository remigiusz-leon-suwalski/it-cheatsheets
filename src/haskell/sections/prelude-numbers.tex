%

\subsection{Numbers}
\textbf{TODO!}
Int, Integer, Float, Double, Rational, Word

Numeric types:
\begin{itemize}
\item \textbf{Int} is a fixed-precision integer type with at least the range $-2^{29} .. 2^{29}-1$.
\item \textbf{Integer} represents the entire infinite range of integers.
\item \textbf{Float} are single-precision floating point numbers.
\item \textbf{Double} are double-precision floating point numbers.
\item \textbf{Rational} are arbitrary-precision rational numbers, represented as a ratio of two Integer values.
\item \textbf{Word} is an unsigned integral type, with the same size as Int.
\end{itemize}

%%% Numeric type classes
\textbf{Num} is a basic numeric class.
The Haskell Report defines no laws for Num.
However, (+) and (*) are customarily expected to define a ring.
\begin{minted}{haskell}
class  Num a  where
    (+), (-), (*)       :: a -> a -> a
    negate              :: a -> a
    abs                 :: a -> a
    signum              :: a -> a
    fromInteger         :: Integer -> a
    x - y               = x + negate y
    negate x            = 0 - x
\end{minted}

% class (Num a, Ord a) => Real a where

% class (Real a, Enum a) => Integral a where

% class Num a => Fractional a where

% class Fractional a => Floating a where

% class (Real a, Fractional a) => RealFrac a where

% class (RealFrac a, Floating a) => RealFloat a where

%%% Numeric functions

%
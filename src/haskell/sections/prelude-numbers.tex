%

\subsubsection{Numeric type classes}

%%% Numeric type classes
\textbf{Num} is a basic numeric class.
The Haskell Report defines no laws for Num.
However, (+) and (*) are customarily expected to define a ring.
\begin{minted}{haskell}
class  Num a  where
    (+), (-), (*) :: a -> a -> a
    negate        :: a -> a
    abs           :: a -> a
    signum        :: a -> a
    fromInteger   :: Integer -> a
    x - y         = x + negate y
    negate x      = 0 - x
\end{minted}

\textbf{Real} is a real numbers class:
\begin{minted}{haskell}
class (Num a, Ord a) => Real a where
    toRational :: a -> Rational 
\end{minted}

\textbf{Integral} supports integer division:
\begin{minted}{haskell}
class (Real a, Enum a) => Integral a where 
    quot      :: a -> a -> a
    rem       :: a -> a -> a
    div       :: a -> a -> a
    mod       :: a -> a -> a
    quotRem   :: a -> a -> (a, a) 
    divMod    :: a -> a -> (a, a) 
    toInteger :: a -> Integer 
\end{minted}

\mintinline{haskell}{quot} truncates toward $0$, \mintinline{haskell}{div} toward $-\infty$.
They satisfy:
\begin{minted}{haskell}
(x `quot` y) * y + (x `rem` y) == x
(x `div` y)  * y + (x `mod` y) == x
\end{minted}
    
% class Num a => Fractional a where

% class Fractional a => Floating a where

% class (Real a, Fractional a) => RealFrac a where

% class (RealFrac a, Floating a) => RealFloat a where

%%% Numeric functions
% subtract even odd gcd lcm ^ ^^ fromIntegral realToFrac

%
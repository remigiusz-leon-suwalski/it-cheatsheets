%

\subsubsection{Numeric type classes}

% TODO
% data Int A fixed-precision integer type with at least the range [-2^29 .. 2^29-1]. The exact range for a given implementation can be determined by using minBound and maxBound from the Bounded class.
% data Integer
% data Float Single-precision floating point numbers. It is desirable that this type be at least equal in range and precision to the IEEE single-precision type.
% data Double Double-precision floating point numbers. It is desirable that this type be at least equal in range and precision to the IEEE double-precision type.
% type Rational = Ratio Integer Arbitrary-precision rational numbers, represented as a ratio of two Integer values. A rational number may be constructed using the % operator.
% data Word A Word is an unsigned integral type, with the same size as Int.

%%% Numeric type classes
\textbf{Num} is a basic numeric class.
The Haskell Report defines no laws for Num.
However, (+) and (*) are customarily expected to define a ring.
\begin{minted}{haskell}
class  Num a  where
    (+), (-), (*) :: a -> a -> a
    negate        :: a -> a
    abs           :: a -> a
    signum        :: a -> a
    fromInteger   :: Integer -> a
    x - y         = x + negate y
    negate x      = 0 - x
\end{minted}

\textbf{Real} is a real numbers class:
\begin{minted}{haskell}
class (Num a, Ord a) => Real a where
    toRational :: a -> Rational 
\end{minted}

\textbf{Integral} supports integer division:
\begin{minted}{haskell}
class (Real a, Enum a) => Integral a where
    quot      :: a -> a -> a
    rem       :: a -> a -> a
    div       :: a -> a -> a
    mod       :: a -> a -> a
    quotRem   :: a -> a -> (a, a) 
    divMod    :: a -> a -> (a, a) 
    toInteger :: a -> Integer 
\end{minted}

\mintinline{haskell}{quot} truncates toward $0$, \mintinline{haskell}{div} toward $-\infty$.
They satisfy:
\begin{minted}{haskell}
(x `quot` y) * y + (x `rem` y) == x
(x `div` y)  * y + (x `mod` y) == x
\end{minted}

There are \mintinline{haskell}{quotRem} and \mintinline{haskell}{divMod} that do two things simultaneously.

\textbf{Fractional} supports real division:
\begin{minted}{haskell}
class Num a => Fractional a where
    (/)          :: a -> a -> a
    recip        :: a -> a
    fromRational :: Rational -> a
    recip x      =  1 / x
    x / y        = x * recip y
\end{minted}

\textbf{Floating} has trigonometric and hyperbolic functions.
\begin{minted}{haskell}
class Fractional a => Floating a where
    pi                  :: a
    exp, log, sqrt      :: a -> a
    (**), logBase       :: a -> a -> a
    sin, cos, tan       :: a -> a
    asin, acos, atan    :: a -> a
    sinh, cosh, tanh    :: a -> a
    asinh, acosh, atanh :: a -> a
\end{minted}

\textbf{RealFrac} extracts components of fractions (?).
\begin{minted}{haskell}
class (Real a, Fractional a) => RealFrac a where
    properFraction :: Integral b => a -> (b, a)
    truncate       :: Integral b => a -> b
    round          :: Integral b => a -> b
    ceiling        :: Integral b => a -> b
    floor          :: Integral b => a -> b
\end{minted}

\textbf{RealFloat} gives efficient, machine-independent access to the components of a floating-point number (?).
\begin{minted}{haskell}
class (RealFrac a, Floating a) => RealFloat a where
    ?
\end{minted}

\subsubsection{Numeric functions}
\begin{minted}{haskell}
subtract :: Num a => a -> a -> a
even :: Integral a => a -> Bool
odd :: Integral a => a -> Bool
gcd :: Integral a => a -> a -> a
lcm :: Integral a => a -> a -> a
(^) :: (Num a, Integral b) => a -> b -> a infixr 8
(^^) :: (Fractional a, Integral b) => a -> b -> a infixr 8
fromIntegral :: (Integral a, Num b) => a -> b
realToFrac :: (Real a, Fractional b) => a -> b
\end{minted}
%%% Numeric functions
% subtract even odd gcd lcm ^ ^^ fromIntegral realToFrac

%
%

\subsubsection{Numeric type classes}
\textbf{Num} is a basic numeric class.
The Haskell Report defines no laws for Num.
However, (+) and (*) are customarily expected to define a ring.
\begin{minted}{haskell}
class  Num a  where
    (+), (-), (*) :: a -> a -> a
    negate        :: a -> a
    abs           :: a -> a
    signum        :: a -> a
    fromInteger   :: Integer -> a
    x - y         = x + negate y
    negate x      = 0 - x
\end{minted}

\textbf{Real} is a real numbers class:
\begin{minted}{haskell}
class (Num a, Ord a) => Real a where
    toRational :: a -> Rational 
\end{minted}

\textbf{Integral} supports integer division:
\begin{minted}{haskell}
class (Real a, Enum a) => Integral a where
    quot      :: a -> a -> a
    rem       :: a -> a -> a
    div       :: a -> a -> a
    mod       :: a -> a -> a
    quotRem   :: a -> a -> (a, a) 
    divMod    :: a -> a -> (a, a) 
    toInteger :: a -> Integer 
\end{minted}

\mintinline{haskell}{quot} truncates toward $0$, \mintinline{haskell}{div} toward $-\infty$.
They satisfy:
\begin{minted}{haskell}
(x `quot` y) * y + (x `rem` y) == x
(x `div` y)  * y + (x `mod` y) == x
\end{minted}

There are \mintinline{haskell}{quotRem} and \mintinline{haskell}{divMod} that do two things simultaneously.

\textbf{Fractional} supports real division:
\begin{minted}{haskell}
class Num a => Fractional a where
    (/)          :: a -> a -> a
    recip        :: a -> a
    fromRational :: Rational -> a
    recip x      =  1 / x
    x / y        = x * recip y
\end{minted}

\textbf{Floating} has trigonometric and hyperbolic functions.
\begin{minted}{haskell}
class Fractional a => Floating a where
    pi                  :: a
    exp, log, sqrt      :: a -> a
    (**), logBase       :: a -> a -> a
    sin, cos, tan       :: a -> a
    asin, acos, atan    :: a -> a
    sinh, cosh, tanh    :: a -> a
    asinh, acosh, atanh :: a -> a
\end{minted}

\textbf{RealFrac} extracts components of fractions (?).
\begin{minted}{haskell}
class (Real a, Fractional a) => RealFrac a
where
properFraction :: Integral b => a -> (b, a)
    truncate   :: Integral b => a -> b
    round      :: Integral b => a -> b
    ceiling    :: Integral b => a -> b
    floor      :: Integral b => a -> b
\end{minted}

\textbf{RealFloat} gives efficient, machine-independent access to the components of a floating-point number (?).
\begin{minted}{haskell}
class (RealFrac a, Floating a) =>
RealFloat a where
    floatRadix
    floatDigits
    floatRange
    decodeFloat
    encodeFloat
    exponent
    significand
    scaleFloat
    isNaN, isInfinite, isIEEE,
    isDenormalized, isNegativeZero
    atan2
\end{minted}

\subsubsection{Numeric functions}
\begin{minted}{haskell}
-- n Num, i Integral, f Fractional, r Real
subtract     :: n -> n -> n
even         :: i -> Bool
odd          :: i -> Bool
gcd          :: i -> i -> i
lcm          :: i -> i -> i
(^)          :: n -> i -> n
(^^)         :: f -> i -> f
fromIntegral :: i -> n
realToFrac   :: r -> f
\end{minted}

\subsubsection{Numeric types}
\mintinline{haskell}{data Int} is a fixed precision integer type with at least he range $[-2^{29}, 2^{29}]$.

\mintinline{haskell}{data Integer} :-)

\mintinline{haskell}{data Float} is single-precision floating.

\mintinline{haskell}{data Double} is double-precision floating.

\mintinline{haskell}{type Rational} is arbitrary-prevision rational, represented as ratio of two integers.

\mintinline{haskell}{data Word} is unsigned integral type, with the same size as \mintinline{haskell}{Int}.

%